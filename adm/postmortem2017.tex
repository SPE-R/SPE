\documentstyle[11pt,a4]{article}
\begin{document}

\section*{SPE-2017: postmortem}

Next year. Martyn explores the possibility to organize the course 
 in Lyon in June, 2018, before 21.6., and will report about the 
IARC response by the end of June 2017.
Dates: Thursday 7 to Tuesday 12, or Thursday 14 to Tuesday 19.
Sunday is day off, but Saturday is a whole course day. 
 
Martyn and Daniel have set up the repository SPE-R at GitHub.
Everybody opens a github account and utilizes the repository.
% Esa's username: tilel, password: spe2017

Brief instructions:
\begin{enumerate}
\item Launch GitKraken
\item Pull
\item Go to the (sub)folder containing your own copy of the repository
\item Edit \& save the source file
\item Stage file
\item Commit message: write a comment
\item Commit changes
\item Push
\end{enumerate}

\subsection*{General points}

It seems desirable to address tools in {\tt tidyverse} in some way at some point, 
especially those for data processing:
 {\tt dplyr} package, {\tt data.table} class, \dots

Create an optional exercise on advanced data handling 
in the first day for those already sufficiently familiar with R. -- Janne \& Martyn will
prepare this.

Style of writing the code could be more consistent. Sweave actually would do it,
if the pertinent parameter is not switched off.  

\subsection*{Day 1}

Lecture on history and ecology of R was OK. 

Lecture on language and basic data 
is too detailed for newcomers at this stage of learning; 
impossible to digest and remember all of it. 
One should focus on the most essential items and leave less important ones aside or to be introduced in the exercise sheets.
For instance, give some more time in the lecture on function calls, encouraging the use
of explicit naming of arguments. Emphasize this also in the exercises -- Krista
will develop the lecture along these lines.

Exercises 1.1 and 1.2: Only few participants reached 1.2, and many people could do 1.1 only to about 1.1.10, which is normal. 
In 1.2.1, provide the exact address from which the .zip containing the data sets can be downloaded to one's own computer and finally to the working directory.
In the end of the exercise it would be nice to provide instructions on how to read data files from an url address. -- Martyn.

Exercises 1.3 on tabulation and exercise 1.4 on graphics were generally OK. 

\subsection*{Day 2}

Lecture on Poisson \& logistic regression. 
The presentation of diabetes example needs some editing.
Analysis on the outbreak study was missing. 
The odds ratio OR associated with chocolate mousse cake was actually nearly 30, but the relative risk RR was only about 12, so in this instance OR cannot be interpreted to approximate RR.
-- Janne changes this after getting some numerical results from Esa.

Lecture on splines: Martyn revises the introduction explicating more clearly the aims of it.

Exercise 1.6: Technical problems in finding tilde '$\sim$' for at least the Estonian and Spanish keyboards. 
Add to pre-course instructions: ``Find out, how to get tilde from your local keyboard. 
Consult \verb|https://en.wikipedia.org/wiki/Tilde|, especially the end of it''.
Same problems with the caret symbol! 
Instruct to do the optional subsection 1.6.7 only after you have done the other 
exercise of this session. \textbf{\textit{Write this in bolded font}}.
-- Esa.

Exercise 1.7: Modify some rows.
In logical expressions change {\tt T} and {\tt F} into {\tt TRUE} and {\tt FALSE}!
Introduce itemization  of the tasks as in other practicals.
 -- Janne


Exercise 1.8, item 11: Change at least {\tt strata = gest4} -- Esa

Exercise 1.9: Develop {\tt plotPenSplines.R} to a more general function.
See {\tt Termplot()} and {\tt termplot()}. 

\subsection*{Day 3}

Lecture on graphics: Maybe something more about {\tt ggplot2} 
and less on {\tt lattice} -- Martyn modifies.

Exercise 1.10: Generally OK. Good to modify some tasks, like removing item 3 to be item 8, so that items 8 to 10 are presented as more advanced. -- Martyn will do.

\subsection*{Day 4}

Lecture on survival: In the post-mortem of 2016 it was said: 
``Should be shortened and somewhat reorganized.''. 
Still there are 55 slides out of which only $\sim$ 42 could be covered. 
It would be desirable to have time for the basics of relative survival, too,
during the lecture. Include a brief comment on the principle of
comparing $O$ with $E$ being applied also in SMR.
Either speed up the oral presentation, or cut down the number of slides.
Reduce the amount of mathematics but refer to Bendix's 5 page intro
on concepts; see \verb|http://bendixcarstensen.com/AdvCoh/papers/fundamentals.pdf|.
Also, remove 21 to 23 on lung cancer risk up to high age, and 27 on {\tt Lexis()}.
Update slide 26 on R tools for competing risks; refer
to \verb|https://cran.r-project.org/| \verb|web/views/Survival.html|.
 Perhaps {\tt mstate} does not deserve special attention any more.
Recall that {\tt survfit()} 
in package {\tt survival} computes AJ-estimates when argument {\tt type="mstate"}
is added in the call of {\tt Surv()}, like is done in Exercise 1.11, item 13.
-- Janne does all of this.

Exercise 1.11: Nothing special to be done. 

Lecture on representation of follow-up \& SMR: When explaining SMR = $O/E$
 make a comment
of an analogy with relative survival. -- Bendix. 

Exercise 1.12: Basic calculation of SMR before spline modelling.
Take out models based on unpenalized splines.
 Reduce the amount of typing by providing a decent function
for repeated tasks, or a ``housekeeping'' script. -- Bendix.  

\subsection*{Day 5}

Lecture on NCC \& CC: Shorten the introductory part. Slide 16 is a bit crowded. -- Esa. 

Exercise 1.14 on NCC \& CC: 
For many this was the first time when {\tt merge()} was asked to use. Make a reference to the previous exercise
where it was introduced.
Change logicals ``{\tt T}'' and ``{\tt F}'' into
{\tt TRUE} and {\tt FALSE}. 
Make sure that the right version will be included in {\tt pracs.tex} and {\tt pracs.pdf} in 2018. -- Esa.

Lecture on causal inference: Some of the slides are too crowded. -- Krista.

Exercise 1.13 on causal inference: Seems OK.

\subsection*{Day 6}

\end{document}

\subsection*{Next year: where and when}
SPE-2017 will be held either in Tartu or in Lyon. Martyn will explore whether there is any 
reasonable time window feasible for the course at IARC.
If in Tartu, the preferred dates would be Fri-Wed 9--14 June, 
which would not overlap either the Gene Forum, or the NBBC17 (in Copenhagen).

In February we need to have a look together, what has been done  by then
and what remains to be done in revising the materials.

\subsection*{Location of course material}
We shall continue like was done this year, i.e. the material is stored on the pertinent folders 
\texttt{SPE/data}, and \texttt{SPE/R}, etc. at the course website.

\subsection*{Practicals}
\begin{enumerate}
\item Remove page 1 from the exercises.
\item Change ancient Unix-based commands like
 \texttt{ls()} and \texttt{rm()} in favour of modern aliases like  \texttt{objects()}
and \texttt{remove()}.
\item Use \texttt{TRUE} and \texttt{FALSE} instead of \texttt{T} and \texttt{F}
with logicals.
\item Numbering the items in exercises should not be restarted from 1 when a new subsection
starts. Use \verb|\begin{enumerate}[resume]| and \\ \verb|\end{enumerate}[resume]| 
for enumerating the items, which requires package \texttt{enumitem}
to be loaded. -- 
See one of Bendix's e-mails before the course.
\item Exercise 1.1: Section 1.1.1 should include an encouragement and 
to use R Studio for better organization of the work and project. 
Modify 1.1.8; remove items asking to \texttt{attach()}. -- Martyn will continue
to develop Exercise 1.1 as well as Exercise 1.2.
\item People should be warned about using \texttt{attach()}. Also,
   in the lectures and exercises we should consistently use and promote 
	\texttt{with(data = ...,} \textit{function}\texttt{(...))} and 
	 \textit{function}\texttt{(..., data =)}.	
\item Use \texttt{url}-addresses of the filenames (from Bendix'is homepage), including
first assigning \verb|url <- "http:/bendixcarstensen.com/SPE/data"| and then 
\verb|read.table( paste( url, "melanoma.dat", sep = "/"), header = T)|. 
%% \item Exercise 1.2.  Most students did not reach to the end of the exercise before lunch. 
\item Exercise 1.3 on simulation: Move it to be after 1.5 on basic graphics, also shortening
it. -- Krista promised to do it.
\item Exercise 1.4 on tabulation: No big changes needed. -- Krista.
\item Exercise 1.5 on basic graphics: Modify  the menu bar paths in 1.5.5 and 1.5.6
to be more generic and less independent on platform. -- Krista will revise this. 
\item Exercise 1.6: Item 10 on deviance? -- Janne will revise.
\item Exercise 1.8 on estimation of effects: Move this right after 1.6 before the current 
exercise 1.7. Sections 1.8.8 and 1.8.9 should be moved into exercise 1.9 on curved effects.
One could expand 1.8.5 on stratified effects by more elaborate treatment
of interaction or effect modification and their different parametrizations.
-- Esa will do this.
\item Exercises 1.7 on logistic modelling. Keep \texttt{effx()} and make perhaps minor revisons.
-- Janne does it.
\item Exercise 1.9 on curved effects: Move 1.8.8 and 1.8.9 here.
Make use of Bendix's recap script and develop \texttt{plotPenSplines.R}
into a proper function. -- Esa and/or Martyn.
\item We should only ever use penalized splines in the course.
%%   We do not explain how to choose knots, for instance. 
	\textbf{Splines need more work and learning!}. Read Simon Wood's book and
  become familiar with the tools in \texttt{mgcv}.
\item Exercise 1.10 on graphics meccano: Take out the housekeeping script and substitute by
\texttt{source()}. Section 1.10.3 needs some fixing -- Martyn will do it.
\item Exercise 1.11 on survival: Remove \texttt{mstate}. Move 1.11.5 \& 1.11.9 to the end  of the
exercise for those who might have done all the preceding items. Elaborate more 
the Fine-Gray model and its problems by  showing that it may give illogical predictions for high risk patients, commenting also
the recommendation given by Latouche et al. (2013) in \textit{J Clin Epidemiol}.
Consider introducing package \texttt{rstpm2} for smooth baseline. -- Janne will revise.
\item Exercise 1.12 on time splitting: Convert to penalized splines and  
make it a little more prescriptive than it is now. Could some of the
more complicated items at the end still be simplified or deleted? 
-- Bendix works on this.
%% SMR should be done earlier in 1.12. 
\item Exercise 1.13. Causal inference: Consider adding one or two other
 interesting DAGs between $X-Y-Z$
and perhaps remove the item on interaction. -- Krista
\item Exercise 1.14 on case-control studies: In \texttt{merge()}, use \texttt{id}
as the key in both files. -- Esa.
\item Exercise 1.15 on multistate models: Convert to penalized splines and perhaps
try to simplify even more. --  Bendix.  
\end{enumerate}

\subsection*{Lectures}
\begin{enumerate}
\item Intro (MP): Add a demo about R studio. Comment out the 4th last slide
about global/local -- Martyn.
\item Language: Discourage the use of \texttt{attach()} and instead recommend
 \texttt{with()} and \texttt{data=} in function calls. -- Krista.
\item Poisson-Logistic: Exponential likelihood instead of Poisson one for censored data on times? 
  Explain the meaning of \texttt{offset}. 
	Remove pages 11 and 14.  
		Reduce presentation of likelihood for case-control studies. -- Janne
\item Linear models etc: 30-45 minutes on classic linear models -- Esa. 
Upon that a separate introductory lecture on splines of 30 min. --  Martyn will work on it.
\item More advanced graphics: No great need to touch -- Martyn.
\item Survival: Should be shortened and somewhat reorganized Package \texttt{rstpm2}?. -- Janne.
\item Follow-up \& SMR: Condense the modelling part. -- Bendix.
\item NCC \& CC: Remove slides 28-29. -- Esa.
\item Causal Inference: Add a couple of other relevant DAGs between $X$, $Z$ and $Y$. -- Krista.
\item Multistate models: Reduce the number of slides from 32 onwards. -- Bendix.

\end{enumerate}
\end{document}








