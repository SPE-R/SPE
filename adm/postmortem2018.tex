\documentstyle[11pt,a4]{article}
\begin{document}

\section*{SPE-2018: postmortem}

Next year in Tartu? Either in late May or in late August. 

\subsection*{General points}

% Create an optional exercise on advanced data handling 
% in the first day for those already sufficiently familiar with R. -- Janne \& Martyn will
% prepare this.

Continue encouraging the use
of explicit naming of arguments. Emphasize this also in the exercises -- modify them as necessary.

In future, family=poisson is to be substituted by family=poisson.process -- whatever its name will be.

Change T into TRUE and F into FALSE when these logical constants are needed.

Add some motivational lines in the beginning of lectures and practicals; 
why we think these things are important.

Instruct how to efficiently write, check and run the R code. 

Add more explanation about what the next lines of cryptic looking script are supposed to do for you.

Change the addresses of the data sets and additional housekeeping and other 
scripts from {\tt bendixcarstensen.com} into the github address.

\subsection*{Day 1}

Lecture on history and ecology of R was OK. So was lecture on language and basic data. 

Exercises 1.1 and 1.2 still long but improved from last year. 

Exercises 1.3 on tabulation and exercise 1.4 on graphics were generally OK. 

\subsection*{Day 2}

Lecture on Poisson \& logistic regression. OK but needs revision with regard to family=poisson.process

Exercise 1.6: OK but needs revision with regard to family=poisson.process

Exercise 1.7: Modify some rows -- Janne

In logical expressions change {\tt T} and {\tt F} into {\tt TRUE} and {\tt FALSE}!

Exercise 1.9: Develop {\tt plotPenSplines.R} to a more general function.
See {\tt Termplot()} and {\tt termplot()} -- \texttt{matshade()}

\subsection*{Day 3}

Lecture on graphics: OK.

Exercise 1.10: Error in the housekeeping script in ggdata -- Martyn will do.

\subsection*{Day 4}

Lecture on survival: Consider removing relative survival.

Exercise 1.11: Perhaps add legends to some plots.  

Lecture on representation of follow-up \& SMR: When explaining SMR, make a comment
of an analogy with relative survival. -- Bendix. 

Exercise 1.12: Change exit.status to a 1/0-indicator rather than keeping it as a factor. Refer to the specific 
slides in the lecture handouts 
when asking to compute and tabulate summary measures like D, Y, rate, E, SMR.

\subsection*{Day 5}

Lecture on NCC \& CC: 

Exercise 1.14 on NCC \& CC: Make sure that the right version will be included in {\tt pracs.tex} and {\tt pracs.pdf} in 2018.

Lecture on causal inference:

Exercise 1.13 on causal inference:

\end{document}

