
\documentclass[a4paper,twoside,12pt]{book}
\usepackage[latin1]{inputenc}
\usepackage[english]{babel}
\usepackage{makeidx,Sweave,floatflt,graphicx}%,stata}
\usepackage{amsmath,amsfonts,datetime}
\usepackage{booktabs,longtable,rotating,graphicx,verbatim,fancyhdr,afterpage}
\usepackage[colorlinks,urlcolor=blue]{hyperref}
\usepackage[dvipsnames]{xcolor}

\oddsidemargin 2.1mm
\evensidemargin 2.1mm
\topmargin -5mm
\headheight 5mm
\headsep 5mm
\textheight 245mm
\textwidth 165mm
\footskip 5mm
\parskip 0mm
% \input{c:/util/tex/settings.tex}
% This is a file of useful extra commands snatched from
% Michael Hills, David Clayton, Bendix Carstensen & Esa Laara.
%

% Commands to draw observation lines on follow-up diagrams
%
% Horizontal lines
%
\providecommand{\hfail}[1]{\begin{picture}(250,5)
      \put(0,0){\line(1,0){#1}}
      \put(#1,0){\circle*{5}}
   \end{picture}}

\providecommand{\hcens}[1]{\begin{picture}(250,5)
      \put(0,0){\line(1,0){#1}}
      \put(#1,0){\line(0,1){2.5}}
      \put(#1,0){\line(0,-1){2.5}}
   \end{picture}}

%
% Diagonals for Lexis diagrams
%
\providecommand{\dfail}[1]{\begin{picture}(250,250)
      \put(0,0){\line(1,1){#1}}
      \put(#1,#1){\circle*{5}}
   \end{picture}}

\providecommand{\dcens}[1]{\begin{picture}(250,250)
      \put(0,0){\line(1,1){#1}}
%      \put(#1,#1){\line(0,1){2.5}}
%      \put(#1,#1){\line(0,-1){2.5}}
% BxC Changed this to an open circle instead of a line
      \put(#1,#1){\circle{5}}
   \end{picture}}

%
% Horizontal range diagrams
%
\providecommand{\hrange}[1]{\begin{picture}(200,5)
     \put(0,0){\circle*{5}}
     \put(0,0){\line(1,0){#1}}
     \put(0,0){\line(-1,0){#1}}
   \end{picture}}

%
% Tree drawing
%
\providecommand{\Tree}[3]{\setlength{\unitlength}{#1\unitlength}\begin{picture}(0,0)
   \put(0,0){\line(3, 2){1}}
   \put(0,0){\line(3,-2){1}}
   \put(0.81, 0.54){\makebox(0,0)[br]{\footnotesize #2\ }}
   \put(0.81,-0.54){\makebox(0,0)[tr]{\footnotesize #3\ }}
\end{picture}}

\providecommand{\Wtree}[3]{\setlength{\unitlength}{#1\unitlength}\begin{picture}(0,0)
   \put(0,0){\line(1, 1){1}}
   \put(0,0){\line(1,-1){1}}
   \put(0.8,0.8){\makebox(0,0)[br]{\footnotesize #2\ }}
   \put(0.8,-0.8){\makebox(0,0)[tr]{\footnotesize #3\ }}
\end{picture}}

\providecommand{\Ntree}[3]{\setlength{\unitlength}{#1\unitlength}\begin{picture}(0,0)
   \put(0,0){\line(2, 1){1}}
   \put(0,0){\line(2,-1){1}}
   \put(0.8,0.4){\makebox(0,0)[br]{\footnotesize #2\ }}
   \put(0.8,-0.4){\makebox(0,0)[tr]{\footnotesize #3\ }}
\end{picture}}

\providecommand{\Nutree}[3]{\setlength{\unitlength}{#1\unitlength}\begin{picture}(0,0)
   \put(0,0){\line(2, 1){#1}}
   \put(0,0){\line(2,-1){#1}}
   \put(0.8,0.4){\makebox(0,0)[br]{#2\ }}
   \put(0.8,-0.4){\makebox(0,0)[tr]{#3\ }}
\end{picture}}

%
% Tree drawing
%
\providecommand{\tree}[3]{\setlength{\unitlength}{#1}\begin{picture}(0,0)
   \put(0,0){\line(3,2){1}}
   \put(0,0){\line(3,-2){1}}
   \put(0.81,0.54){\makebox(0,0)[br]{\footnotesize #2\ }}
   \put(0.81,-0.54){\makebox(0,0)[tr]{\footnotesize #3\ }}
\end{picture}}

\providecommand{\wtree}[3]{\setlength{\unitlength}{#1}\begin{picture}(0,0)
   \put(0,0){\line(1,1){1}}
   \put(0,0){\line(1,-1){1}}
   \put(0.8,0.8){\makebox(0,0)[br]{\footnotesize #2\ }}
   \put(0.8,-0.8){\makebox(0,0)[tr]{\footnotesize #3\ }}
\end{picture}}

\providecommand{\ntree}[3]{\setlength{\unitlength}{#1}\begin{picture}(0,0)
   \put(0,0){\line(2,1){1}}
   \put(0,0){\line(2,-1){1}}
   \put(0.8,0.4){\makebox(0,0)[br]{\footnotesize #2\ }}
   \put(0.8,-0.4){\makebox(0,0)[tr]{\footnotesize #3\ }}
\end{picture}}

\providecommand{\nutree}[3]{\begin{picture}(0,0)
   \put(0,0){\line(2,1){#1}}
   \put(0,0){\line(2,-1){#1}}
   \put(0.8,0.4){\makebox(0,0)[br]{#2\ }}
   \put(0.8,-0.4){\makebox(0,0)[tr]{#3\ }}
\end{picture}}

%
% Other commands
%
\providecommand{\prob}[0]{\text{\rm Pr}}
\providecommand{\nhy}[0]{_{\oslash}}
\providecommand{\true}[0]{_{\text{\rm \tiny T}}}
\providecommand{\hyp}[0]{_{\text{\rm \tiny H}}}
% \providecommand{\mpydiv}[0]{\stackrel{\textstyle \times}{\div}}
% Changed to slightly smaller symbols
\providecommand{\mpydiv}[0]{\stackrel{\times}{\scriptstyle \div}}
\providecommand{\mie}[1]{{\it #1}}
\providecommand{\mycircle}[0]{\circle*{5}}
\providecommand{\smcircle}[0]{\circle*{1}}
\providecommand{\corner}[0]{_{\text{\rm \tiny C}}}
\providecommand{\ind}[0]{\hspace{10pt}}
\providecommand{\gap}[0]{\\[5pt]}
\renewcommand{\S}[0]{section~}
\providecommand{\blank}[0]{$\;\,$}
\providecommand{\vone}{\vspace{1cm}}
\providecommand{\ljust}[1]{\multicolumn{1}{l}{#1}}
\providecommand{\cjust}[1]{\multicolumn{1}{c}{#1}}
\providecommand{\Var}{\text{\rm var}}
\providecommand{\cov}{\text{\rm cov}}
\providecommand{\corr}{\text{\rm corr}}
\providecommand{\mean}{\text{\rm mean}}
\providecommand{\median}{\text{\rm median}}
\providecommand{\transpose}{^{\text{\sf T}}}
\providecommand{\histog}[5]{\rule{1mm}{#1mm}\,\rule{1mm}{#2mm}\,\rule{1mm}{#3mm}\,\rule{1mm}{#4mm}\,\rule{1mm}{#5mm}}
\providecommand{\pmiss}{P_{\mbox{\tiny miss}}}

% Below is BxCs commands inserted

\providecommand{\bc}{\begin{center}}
\providecommand{\ec}{\end{center}}
\providecommand{\bd}{\begin{description}}
\providecommand{\ed}{\end{description}}
\providecommand{\bi}{\begin{itemize}}
\providecommand{\ei}{\end{itemize}}
\providecommand{\bn}{\begin{equation}}
\providecommand{\en}{\end{equation}}
\providecommand{\be}{\begin{enumerate}}
\providecommand{\ee}{\end{enumerate}}
\providecommand{\bes}{\begin{eqnarray*}}
\providecommand{\ees}{\end{eqnarray*}}

\providecommand{\p}{{\mathrm p}}
\providecommand{\e}{{\mathrm e}}
\providecommand{\D}{{\mathrm D}}
\providecommand{\dif}{{\,\mathrm d}}
\providecommand{\pmat}[1]{\text{\rm P}\left\{#1\right\}}
\providecommand{\ptxt}[1]{\text{\rm P}\left\{\text{#1}\right\}}
\providecommand{\E}{\text{\rm E}}
\providecommand{\V}{\text{\rm V}}
\providecommand{\BLUP}{\text{\rm BLUP}}
\providecommand{\se}{\text{\rm s.e.}}
\providecommand{\sem}{\text{\rm s.e.m.}}
\providecommand{\std}{\text{\rm std}}
\providecommand{\sd}{\text{\rm s.d.}}
\providecommand{\cv}{\text{\rm c.v.}}
\providecommand{\erf}{\text{\rm erf}}
\providecommand{\ef}{\text{\rm ef}}
\providecommand{\SSD}{\text{\rm SSD}}
\providecommand{\SPD}{\text{\rm SPD}}
\providecommand{\odds}{\text{\rm odds}}
\providecommand{\bin}{\text{\rm binom}}
\providecommand{\half}{\frac{1}{2}}
% \providecommand{\td}[0]{\stackrel{\textstyle \times}{\div}}
% Changed to slightly smaller symbols
\providecommand{\td}[0]{\stackrel{\scriptstyle \times}{\scriptstyle \div}}
\providecommand{\dt}[0]{\stackrel{\scriptstyle \div}{\scriptstyle \times}}
\providecommand{\diag}{\text{\rm diag}}
\providecommand{\spcol}{\text{\rm span}}
\providecommand{\logit}{\text{\rm logit}}
% \providecommand{\link}{\text{\rm link}}
\providecommand{\spn}{\text{\rm span}}
\providecommand{\CI}{\text{\rm CI}}
\providecommand{\IP}{\text{\rm IP}}
\providecommand{\OR}{\text{\rm OR}}
\providecommand{\RR}{\text{\rm RR}}
\providecommand{\ER}{\text{\rm ER}}
\providecommand{\EM}{\text{\rm EM}}
\providecommand{\EF}{\text{\rm EF}}
\providecommand{\RD}{\text{\rm RD}}
\providecommand{\AC}{\text{\rm AC}}
\providecommand{\AF}{\text{\rm AF}}
\providecommand{\PAF}{\text{\rm PAF}}
\providecommand{\AR}{\text{\rm AR}}
\providecommand{\CR}{\text{\rm CR}}
\providecommand{\PAR}{\text{\rm PAR}}
\providecommand{\SD}{\text{\rm SD}}
\providecommand{\SE}{\text{\rm SE}}
\providecommand{\SEM}{\text{\rm SEM}}
\providecommand{\SR}{\text{\rm SR}}
\providecommand{\SMR}{\text{\rm SMR}}
\providecommand{\RSR}{\text{\rm RSR}}
\providecommand{\CMF}{\text{\rm CMF}}
\providecommand{\pvp}{\text{\rm PV$+$}}
\providecommand{\pvn}{\text{\rm PV$-$}}
\providecommand{\R}{\textsf{R}}
\providecommand{\sas}{\textsl{\textbf{SAS}}}
\providecommand{\SAS}{\textsl{\textbf{SAS}}}
%\providecommand{\gap}[0]{\\[5pt]}
%\providecommand{\blank}[0]{$\;\,$}
% Conditional independence sign from Philip Dawid
\providecommand{\cip}{\mbox{$\perp\!\!\!\perp$}}

%%% Commands to comment out parts of the text
\providecommand{\GLEM}[1]{}
\providecommand{\FORGETIT}[1]{}
\providecommand{\OMIT}[1]{}

%%% Insert output from program in small text
%%% (requires package verbatim)
\providecommand{\insout}[1]{
 \scriptsize
 \renewcommand{\baselinestretch}{0.8}
 \verbatiminput{#1}
 \renewcommand{\baselinestretch}{1.0}
 \normalsize
}
\providecommand{\insouttiny}[1]{
\tiny
\renewcommand{\baselinestretch}{0.8}
\verbatiminput{#1}
\renewcommand{\baselinestretch}{1.0}
\normalsize
}

% From Esa:
\providecommand{\T}{\text{\rm \small{T}}}
\providecommand{\id}{\text{\rm id}}
\providecommand{\Dev}{\text{\rm Dev}}
\providecommand{\Bin}{\text{\rm Bin}}
\providecommand{\probit}{\text{\rm probit}}
\providecommand{\cloglog}{\text{\rm cloglog}}

% Special commands to include output from R, Bugs and Stata

\providecommand{\Rin}[2]{
\subsection{\texttt{#1.R}}
#2

\insout{./R/#1.Rout}

}

\providecommand{\Statain}[2]{
\subsection{\texttt{#1.do}}
#2

\insout{./do/#1.log}

}

\providecommand{\Bugsin}[2]{
\subsection{\texttt{#1.bug}}
#2

\insout{./bugs/#1.bug}

}

\newlength{\wdth}
\providecommand{\fxbl}[1]{\settowidth{\wdth}{#1} \makebox[\wdth]{}}

%%% Local Variables:
%%% mode: latex
%%% TeX-master: t
%%% End:

\setcounter{secnumdepth}{0}

\pagestyle{fancy}
\fancyhead[OR]{{}{\quad \bf \thepage}}
\fancyhead[EL]{{\bf \thepage \quad}{}}
\fancyhead[OL]{\sl SPE 2017 program}
\fancyhead[ER]{\sl SPE 2017 program}
\fancyfoot{}
\renewcommand{\headrulewidth}{0.1pt}

\begin{document}

\noindent
\large
\textbf{Course venue:}
\href{https://www.google.dk/maps/place/University+of+Tartu+Faculty+of+Mathematics+and+Computer+Science/@58.3782997,26.7147398,17z/data=!4m5!3m4!1s0x46eb36e18ffb39cd:0xb51b2c14d336ddbf!8m2!3d58.3782519!4d26.7145735?hl=en}{Mathematics
  buiding, University of Tartu, Juhan Liivi 2}\\[1em]
\normalsize
\noindent
\begin{tabular}{r@{ -- }rp{13cm}}
\multicolumn{3}{l}{\bf Daily timetable} \\
 9:00 &  9:30 & Recap of yesterday's practicals \\
 9:30 & 10:30 & Lecture \\
10:30 & 10:50 & Coffee break \\
10:50 & 12:50 & Practical \\
12:50 & 14:00 & Lunch \\
14:00 & 14:30 & Recap of morning's practical \\
14:30 & 15:30 & Lecture \\
15:30 & 16:00 & Tea break \\
16:00 & 18:00 & Practical \\[2em]
\end{tabular}

\noindent
\begin{tabular}{r@{ -- }rp{12cm}}
\multicolumn{3}{l}{\bf Thursday 1 June} \\
 9:00 &  9:15 & Welcome (KF) \\
 9:15 & 10:30 & Introduction to R language and commands reading data (MP) \\
10:30 & 10:50 & Coffeee break \\
10:50 & 12:50 & Practical:
                Practice with basic \R \newline
                Simple reading and data input \\
12:50 & 14:00 & Lunch \\
14:00 & 14:30 & Recap of morning practical \\
14:30 & 15:30 & Language, indexing,
                {\tt subset()}, {\tt ifelse()},
                \texttt{attach(), detach()},
                \texttt{search}. Simple simulation. Simple graphics. (KF)\\
15:30 & 16:00 & Tea break\\
16:00 & 18:00 & Practical: Simple simulation \newline
                Tabulation\newline
                Introduction to graphs in R \\
18:00 & 19:00 & Tour of the genome center before the \\
19:00 & 21:00 & Welcome reception at the \newline
\href{https://www.google.dk/maps/place/Riia+23b,+51010+Tartu,+Estonia/@58.3728901,26.7157088,17z/data=!3m1!4b1!4m5!3m4!1s0x46eb371f8605d0f9:0x66688818ca0e3156!8m2!3d58.3728873!4d26.7178975?hl=en}%
{Estonian Genome Center (Eesti Geenivaramu), Riia 23b.}\\[1em]
\end{tabular}

\noindent
\begin{tabular}{r@{ -- }rp{13cm}}
\multicolumn{3}{l}{\bf Friday 2 June} \\
 9:00 &  9:30 & Recap of yesterday's practicals. \\
 9:30 & 10:30 & Poisson regression for follow-up studies ---
                likelihood for a constant rate \newline
                Logistic regression for cc-studies (JP) \\
10:30 & 10:50 & Coffeee break \\
10:50 & 12:50 & Practical: Rates, rate ratio and rate difference with \texttt{glm}\newline
                Logistic regression with \texttt{glm} \\
12:50 & 14:00 & Lunch \\
14:00 & 14:30 & Recap of morning practical \\
14:30 & 15:45 & Linear and generalized linear models (EL) \newline
                All you ever wanted to know about splines (MP) \\
15:45 & 16:15 & Tea break\\
16:15 & 18:00 & Practical: Simple estimation of effects \newline
                Estimation and reporting of linear and curved effects \\
\end{tabular}

\noindent
\begin{tabular}{r@{ -- }rp{13cm}}
 \multicolumn{3}{l}{\bf Saturday 3 June} \\
 9:00 &  9:30 & Recap of yesterday's practicals \\
 9:30 & 10:30 & More advanced graphics in R, including \texttt{ggplot2} (MP)\\
10:30 & 10:50 & Coffee break. \\
10:50 & 12:50 & Practical: Graphical meccano \\
12:50 & 14:00 & Lunch\\
\multicolumn{2}{l}{Afternoon}
              & Orienteering and visit at the 
\href{https://www.google.dk/maps/place/Estonian+National+Museum/@58.395294,26.7355243,15z/data=!4m5!3m4!1s0x46eb371e0069d991:0x3b484674469f1cea!8m2!3d58.395294!4d26.744279?hl=en}{Estonian National Museum} (optional)\\[2em]
\end{tabular}

\noindent
\begin{tabular}{r@{ -- }rp{13cm}}
\multicolumn{3}{l}{\bf Sunday 4 June} \\
 9:00 &  9:30 & Recap of yesterday's practicals \\
 9:30 & 10:30 & Survival analysis: Kaplan Meier \& simple
                Cox-model. Simple competing risks and relative
                survival. (JP)\\
10:30 & 10:50 & Tea break\\
10:50 & 12:50 & Practical: Survival and competing risks in oral\
                cancer. Relative survival.\\
12:50 & 14:00 & Lunch \\
14:00 & 14:30 & Recap of morning practical \\
14:30 & 15:30 & Dates in R; follow up representation in \texttt{Lexis} objects,
                time-splitting, multistate model and SMR. (BxC)\\
15:30 & 16:00 & Coffee break. \\
16:00 & 18:00 & Practical: Time-splitting and SMR (Danish diabetes patients)\\[1em]
\end{tabular}

\noindent
\begin{tabular}{r@{ -- }rp{13cm}}
 \multicolumn{3}{l}{\bf Monday 5 June} \\
 9:00 &  9:30 & Recap of yesterday's practicals \\
 9:30 & 10:30 & Nested and matched cc-studies \& Case-cohort studies (EL) \\
10:30 & 10:50 & Coffee break. \\
10:50 & 12:50 & Practical: CC study: Risk factors for Coronary heart disease\\
12:50 & 14:00 & Lunch \\
14:00 & 14:30 & Recap of morning practical \\
14:30 & 15:30 & Causal inference. (KF)\\
15:30 & 16:00 & Coffee break. \\
16:00 & 18:00 & Practical: Simulation and causal inference\\
19:00 &       & Course dinner at %
\href{https://www.google.com/maps/place/Vilde+lokaal+ja+kohvik,+catering/@58.3786583,26.7195608,17.25z/data=!4m5!3m4!1s0x46eb36e097301cdb:0x32b45e5712e44ff!8m2!3d58.3781843!4d26.7228678}{Wilde}\\[1em]
\end{tabular}

\noindent
\begin{tabular}{r@{ -- }rp{13cm}}
\multicolumn{3}{l}{\bf Tuesday 6 June} \\
 9:00 &  9:30 & Recap of yesterday's practicals \\
 9:30 & 10:30 & Multistate models, Poisson models for rates and
                simulation of \texttt{Lexis} objects (BxC)\\
10:30 & 10:50 & Coffee break. \\
10:50 & 12:30 & Practical: Multistate-model: Renal complications\\
12:30 & 13:00 & Recap of morning practical \\
13:00 & 13:15 & Wrap-up and farewell.\\
13:15 & 14:15 & Lunch \\
% \multicolumn{2}{l}{Afternoon} & Post-mortem (Faculty only). \\
% \multicolumn{2}{l}{Evening}   & Faculty dinner.\\
\end{tabular}
\vfill
\noindent
Further material will appear at this year's course website:\\ 
\url{http://bendixcarstensen.com/SPE/2017}
\vfill

\end{document}

\newpage
Here is an overview of who is doing what:
\begin{Schunk}
\begin{Sinput}
> options( width=135 )
> wh <- read.table("load.dat", header=T, as.is=T )
> wh
\end{Sinput}
\begin{Soutput}
   length type                                                            topic teacher
1      75    L            Introduction to R language and commands, reading data      MP
2       1    P Reading data, simple tabulation, data frames, with and without $      MP
3      60    L           Language, indexing, Simple simulation, simple graphics      KF
4       1    P                      Tabulation and simple estimation of effects      EL
5      60    L                    Linear models, fitting, ci.lin, simple spline      EL
6       1    P                                                       Using effx      MP
7       1    P           Estimating a linear, quadratic and reporting the graph     BxC
8      30    L                               Logistic regression for cc-studies      JP
9      30    L                         Poisson regression for follow-up studies      JP
10      1    P                 Simple exercise on rates, RR, RD, and 2x2 tables     BxC
11      1    P                                       Simple logistic regression      KF
12     60    L                                      More advanced graphics in R      MP
13      1    P    Graphics meccano: how to build an informative graph / ggplot2      MP
14     60    L    Survival analysis, KM, simple Cox, simple comp risk, rel.surv      JP
15      1    P                 Practical on relative survival / additive hazard      JP
16      1    P                                             Oral cancer survival      EL
17     60    L                 Dates; follow up representation in Lexis objects     BxC
18      1    P                                    Danish diabetes register data     BxC
19     60    L                                                 Causal inference      KF
20      1    P            Simulation for causal inference and linear regression      KF
21     60    L                  Nested, matched cc-studies. Case-cohort studies      EL
22      1    P                                       Matched case-control study      EL
23      1    P                                                Case-cohort study      EL
24     60    L               Multistate models, simulation in multistate models     BxC
25      1    P               Exercise on multistate models: Renal complications     BxC
\end{Soutput}
\begin{Sinput}
> str(wh)
\end{Sinput}
\begin{Soutput}
'data.frame':	25 obs. of  4 variables:
 $ length : int  75 1 60 1 60 1 1 30 30 1 ...
 $ type   : chr  "L" "P" "L" "P" ...
 $ topic  : chr  "Introduction to R language and commands, reading data" "Reading data, simple tabulation, data frames, with and without $" "Language, indexing, Simple simulation, simple graphics" "Tabulation and simple estimation of effects" ...
 $ teacher: chr  "MP" "MP" "KF" "EL" ...
\end{Soutput}
\end{Schunk}
Here is the calculation of the lecturing loads, note there is no
accounting of the actual preparation load (due to earlier lecturing on
the topic) has been taken.
\begin{Schunk}
\begin{Sinput}
> LL <- with( subset(wh,type=="L"), tapply(length, list(topic,teacher), sum ) )
> LL[is.na(LL)] <- 0
> str(LL)
\end{Sinput}
\begin{Soutput}
 num [1:11, 1:5] 0 60 0 0 0 0 0 60 0 0 ...
 - attr(*, "dimnames")=List of 2
  ..$ : chr [1:11] "Causal inference" "Dates; follow up representation in Lexis objects" "Introduction to R language and commands, reading data" "Language, indexing, Simple simulation, simple graphics" ...
  ..$ : chr [1:5] "BxC" "EL" "JP" "KF" ...
\end{Soutput}
\begin{Sinput}
> print.table(addmargins(LL,1),ze=".")
\end{Sinput}
\begin{Soutput}
                                                              BxC  EL  JP  KF  MP
Causal inference                                                .   .   .  60   .
Dates; follow up representation in Lexis objects               60   .   .   .   .
Introduction to R language and commands, reading data           .   .   .   .  75
Language, indexing, Simple simulation, simple graphics          .   .   .  60   .
Linear models, fitting, ci.lin, simple spline                   .  60   .   .   .
Logistic regression for cc-studies                              .   .  30   .   .
More advanced graphics in R                                     .   .   .   .  60
Multistate models, simulation in multistate models             60   .   .   .   .
Nested, matched cc-studies. Case-cohort studies                 .  60   .   .   .
Poisson regression for follow-up studies                        .   .  30   .   .
Survival analysis, KM, simple Cox, simple comp risk, rel.surv   .   .  60   .   .
Sum                                                           120 120 120 120 135
\end{Soutput}
\end{Schunk}
And here is the assignment of practicals:
\begin{Schunk}
\begin{Sinput}
> PP <- with( subset(wh,type=="P"), tapply(length, list(topic,teacher), sum ) )
> PP[is.na(PP)] <- 0
> str(PP)
\end{Sinput}
\begin{Soutput}
 num [1:14, 1:5] 0 1 1 1 0 0 0 0 0 1 ...
 - attr(*, "dimnames")=List of 2
  ..$ : chr [1:14] "Case-cohort study" "Danish diabetes register data" "Estimating a linear, quadratic and reporting the graph" "Exercise on multistate models: Renal complications" ...
  ..$ : chr [1:5] "BxC" "EL" "JP" "KF" ...
\end{Soutput}
\begin{Sinput}
> print.table(addmargins(PP,1),ze=".")
\end{Sinput}
\begin{Soutput}
                                                                 BxC EL JP KF MP
Case-cohort study                                                  .  1  .  .  .
Danish diabetes register data                                      1  .  .  .  .
Estimating a linear, quadratic and reporting the graph             1  .  .  .  .
Exercise on multistate models: Renal complications                 1  .  .  .  .
Graphics meccano: how to build an informative graph / ggplot2      .  .  .  .  1
Matched case-control study                                         .  1  .  .  .
Oral cancer survival                                               .  1  .  .  .
Practical on relative survival / additive hazard                   .  .  1  .  .
Reading data, simple tabulation, data frames, with and without $   .  .  .  .  1
Simple exercise on rates, RR, RD, and 2x2 tables                   1  .  .  .  .
Simple logistic regression                                         .  .  .  1  .
Simulation for causal inference and linear regression              .  .  .  1  .
Tabulation and simple estimation of effects                        .  1  .  .  .
Using effx                                                         .  .  .  .  1
Sum                                                                4  4  1  2  3
\end{Soutput}
\end{Schunk}

\end{document}
