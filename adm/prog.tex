
\documentclass[a4paper,twoside,12pt]{book}
\usepackage[latin1]{inputenc}
\usepackage[english]{babel}
\usepackage{makeidx,Sweave,floatflt,graphicx}%,stata}
\usepackage{amsmath,amsfonts,datetime}
\usepackage{booktabs,longtable,rotating,graphicx,verbatim,fancyhdr,afterpage}
\usepackage[colorlinks,urlcolor=blue]{hyperref}
\usepackage[dvipsnames]{xcolor}

\oddsidemargin 2.1mm
\evensidemargin 2.1mm
\topmargin -5mm
\headheight 5mm
\headsep 5mm
\textheight 245mm
\textwidth 165mm
\footskip 5mm
\parskip 0mm
% \input{c:/util/tex/settings.tex}
% This is a file of useful extra commands snatched from
% Michael Hills, David Clayton, Bendix Carstensen & Esa Laara.
%

% Commands to draw observation lines on follow-up diagrams
%
% Horizontal lines
%

% exit time with failure, bullet
\newcommand{\hfail}[1]{\begin{picture}(250,5)
       \put(0,0){\line(0,1){2.5}}
      \put(0,0){\line(0,-1){2.5}}
      \put(0,0){\line(1,0){#1}}
      \put(#1,0){\circle*{5}}
   \end{picture}}

% exit time with censoring, open circle
\newcommand{\hcens}[1]{\begin{picture}(250,5)
         \put(0,0){\line(0,1){2.5}}
      \put(0,0){\line(0,-1){2.5}}
      \put(0,0){\line(1,0){#1}}
%      \put(#1,0){\line(0,1){2.5}}
%      \put(#1,0){\line(0,-1){2.5}}
% BxC Changed this to an open circle instead of a line
      \put(#1,0){\circle{5}}
   \end{picture}}

%
% Diagonals for Lexis diagrams
%
\newcommand{\dfail}[1]{\begin{picture}(250,250)
      \put(0,0){\line(1,1){#1}}
      \put(#1,#1){\circle*{5}}
   \end{picture}}

\newcommand{\dcens}[1]{\begin{picture}(250,250)
      \put(0,0){\line(1,1){#1}}
%      \put(#1,#1){\line(0,1){2.5}}
%      \put(#1,#1){\line(0,-1){2.5}}
% BxC Changed this to an open circle instead of a line
      \put(#1,#1){\circle{5}}
   \end{picture}}

%
% Horizontal range diagrams
%
\newcommand{\hrange}[1]{\begin{picture}(200,5)
     \put(0,0){\circle*{5}}
     \put(0,0){\line(1,0){#1}}
     \put(0,0){\line(-1,0){#1}}
   \end{picture}}

%
% Tree drawing
%
\newcommand{\tree}[3]{\setlength{\unitlength}{#1}\begin{picture}(0,0)
   \put(0,0){\line(3,2){1}}
   \put(0,0){\line(3,-2){1}}
   \put(0.81,0.54){\makebox(0,0)[br]{\footnotesize #2\ }}
   \put(0.81,-0.54){\makebox(0,0)[tr]{\footnotesize #3\ }}
\end{picture}}

\newcommand{\wtree}[3]{\setlength{\unitlength}{#1}\begin{picture}(0,0)
   \put(0,0){\line(1,1){1}}
   \put(0,0){\line(1,-1){1}}
   \put(0.8,0.8){\makebox(0,0)[br]{\footnotesize #2\ }}
   \put(0.8,-0.8){\makebox(0,0)[tr]{\footnotesize #3\ }}
\end{picture}}

\newcommand{\ntree}[3]{\setlength{\unitlength}{#1}\begin{picture}(0,0)
   \put(0,0){\line(2,1){1}}
   \put(0,0){\line(2,-1){1}}
   \put(0.8,0.4){\makebox(0,0)[br]{\footnotesize #2\ }}
   \put(0.8,-0.4){\makebox(0,0)[tr]{\footnotesize #3\ }}
\end{picture}}

%
% Other commands
%
\newcommand{\T}{\scriptsize\text T}
\newcommand{\prob}[0]{\text{\rm Pr}}
\newcommand{\nhy}[0]{_{\oslash}}
\newcommand{\true}[0]{_{\text{\rm \tiny T}}}
\newcommand{\hyp}[0]{_{\text{\rm \tiny H}}}
% \newcommand{\mpydiv}[0]{\stackrel{\textstyle \times}{\div}}
% Changed to slightly smaller symbols
\newcommand{\mpydiv}[0]{\stackrel{\times}{\scriptstyle \div}}
\newcommand{\mie}[1]{{\it #1}}
\newcommand{\mycircle}[0]{\circle*{5}}
\newcommand{\smcircle}[0]{\circle*{1}}
\newcommand{\corner}[0]{_{\text{\rm \tiny C}}}
\newcommand{\ind}[0]{\hspace{10pt}}
\newcommand{\gap}[0]{\\[5pt]}
\renewcommand{\S}[0]{section~}
\newcommand{\blank}[0]{$\;\,$}
\newcommand{\vone}{\vspace{1cm}}
\newcommand{\ljust}[1]{\multicolumn{1}{l}{#1}}
\newcommand{\cjust}[1]{\multicolumn{1}{c}{#1}}
\newcommand{\mean}{\text{\rm Mean}}
\newcommand{\transpose}{^{\mbox{\tiny T}}}
\newcommand{\histog}[5]{\rule{1mm}{#1mm}\,\rule{1mm}{#2mm}\,\rule{1mm}{#3mm}\,\rule{1mm}{#4mm}\,\rule{1mm}{#5mm}}
\newcommand{\pmiss}{P_{\mbox{\tiny miss}}}
%
% Below is BxCs commands inserted
%
\newcommand{\bc}{\begin{center}}
\newcommand{\ec}{\end{center}}

\newcommand{\bd}{\setlength{\parskip}{1ex} \begin{description}}
\newcommand{\ed}{\end{description} \setlength{\parskip}{2ex}}
\newcommand{\bdx}{\begin{description}} % Bendix' description macros
\newcommand{\edx}{\end{description}}

\newcommand{\bix}{\begin{itemize}}  % these are Bendix' itemizing macros
\newcommand{\eix}{\end{itemize}}
\newcommand{\bi}{\setlength{\parskip}{1ex} \begin{itemize}} % Esa's item macros 
\newcommand{\ei}{\end{itemize} \setlength{\parskip}{2ex}} 

\newcommand{\bn}{\begin{equation}}
\newcommand{\en}{\end{equation}}
\newcommand{\be}{\begin{enumerate}}
\newcommand{\ee}{\end{enumerate}}
\newcommand{\bes}{\begin{eqnarray*}}
\newcommand{\ees}{\end{eqnarray*}}
\newcommand{\p}{\text{\rm P}}
\newcommand{\pmat}[1]{\text{\rm P}\left\{#1\right\}}
\newcommand{\ptxt}[1]{\text{\rm P}\left\{\text{\rm #1}\right\}}
\newcommand{\E}{\text{\rm E}}
\newcommand{\V}{\text{\rm V}}
\newcommand{\BLUP}{\text{\rm BLUP}}

% \newcommand{\var}{\mbox{Var}} changed by Esa to
\newcommand{\var}{\mbox{var}}
% \newcommand{\cov}{\mbox{Cov}} changed by Esa to
\newcommand{\cov}{\mbox{cov}}
% \newcommand{\corr}{\mbox{Corr}} changed by Esa to
\newcommand{\corr}{\mbox{corr}} 

%\newcommand{\var}{\text{\rm var}}
%\newcommand{\cov}{\text{\rm cov}}
%\newcommand{\corr}{\text{\rm corr}}
\newcommand{\se}{\text{\rm s.e.}}
\newcommand{\sd}{\text{\rm std}}
\newcommand{\erf}{\text{\rm erf}}
\newcommand{\odds}{\text{\rm odds}}
\newcommand{\bin}{\text{\rm binom}}
\newcommand{\half}[1]{\frac{1}{#1}}
% \newcommand{\td}[0]{\stackrel{\textstyle \times}{\div}}
% Changed to slightly smaller symbols
\newcommand{\td}[0]{\stackrel{\scriptstyle \times}{\scriptstyle \div}}
\newcommand{\logit}{\text{\rm logit}}
\newcommand{\link}{\text{\rm link}}
\newcommand{\spn}{\text{\rm span}}
\newcommand{\OR}{\text{\rm OR}}
\newcommand{\RR}{\text{\rm RR}}
\newcommand{\ER}{\text{\rm ER}}
\newcommand{\RD}{\text{\rm RD}}
\newcommand{\AC}{\text{\rm AC}}
\newcommand{\AF}{\text{\rm AF}}
\newcommand{\PAF}{\text{\rm PAF}}
\newcommand{\SR}{\text{\rm SR}}
\newcommand{\SMR}{\text{\rm SMR}}
\newcommand{\CMF}{\text{\rm CMF}}
\newcommand{\pvp}{\text{\rm PV}$+$}
\newcommand{\pvn}{\text{\rm PV}$-$}
\newcommand{\R}{\textsf{R}}
%\newcommand{\gap}[0]{\\[5pt]} 
%\newcommand{\blank}[0]{$\;\,$}
% Conditional independence sign from Philip Dawid
\newcommand{\cip}{\mbox{$\perp\!\!\!\perp$}}

%%% Commands to comment out parts of the text
\newcommand{\GLEM}[1]{}
\newcommand{\FORGETIT}[1]{}
\newcommand{\OMIT}[1]{}

%%% Insert output from program in small text 
%%% (requires package verbatim)

\newcommand{\insout}[1]{
\scriptsize
\renewcommand{\baselinestretch}{0.8}
\verbatiminput{#1}
\renewcommand{\baselinestretch}{1.0}
\normalsize
}

% From Esa:        
%\newcommand{\T}{\text{\rm \small{T}}}
\newcommand{\id}{\text{\rm id}}
\newcommand{\Dev}{\text{\rm Dev}}
\newcommand{\Bin}{\text{\rm Bin}}
\newcommand{\probit}{\text{\rm probit}}
\newcommand{\cloglog}{\text{\rm cloglog}}
\newcommand{\EF}{\text{\rm EF}}
\newcommand{\SE}{\text{\rm SE}}
\newcommand{\IP}{\text{\rm IP}}



\setcounter{secnumdepth}{0}

\pagestyle{fancy}
\fancyhead[OR]{{}{\quad \bf \thepage}}
\fancyhead[EL]{{\bf \thepage \quad}{}}
\fancyhead[OL]{\sl SPE 2017 program}
\fancyhead[ER]{\sl SPE 2017 program}
\fancyfoot{}
\renewcommand{\headrulewidth}{0.1pt}

\begin{document}

\noindent
\large
\textbf{Course venue:}
\href{https://www.google.dk/maps/place/University+of+Tartu+Faculty+of+Mathematics+and+Computer+Science/@58.3782997,26.7147398,17z/data=!4m5!3m4!1s0x46eb36e18ffb39cd:0xb51b2c14d336ddbf!8m2!3d58.3782519!4d26.7145735?hl=en}{Mathematics
  buiding, University of Tartu, Juhan Liivi 2}\\[1em]
\normalsize
\noindent
\begin{tabular}{r@{ -- }rp{13cm}}
\multicolumn{3}{l}{\bf Daily timetable} \\
 9:00 &  9:30 & Recap of yesterday's practicals \\
 9:30 & 10:30 & Lecture \\
10:30 & 11:00 & Coffee break \\
11:00 & 13:00 & Practical \\
13:00 & 14:00 & Lunch \\
14:00 & 14:30 & Recap of morning's practical \\
14:30 & 15:30 & Lecture \\
15:30 & 16:00 & Tea break \\
16:00 & 18:00 & Practical \\[2em]
\end{tabular}

\noindent
\begin{tabular}{r@{ -- }rp{13cm}}
\multicolumn{3}{l}{\bf Friday 2 June -- Day 1} \\
 8:30 &  9:00 & Registration \\
 9:00 &  9:15 & Welcome (KF) \\
 9:15 & 10:15 & R History and Ecology\\% (MP, 15 min shorter than before, remote mode) \\ 
10:15 & 10:30 & Data manipulation with \texttt{tidyverse} (DG) \\
10:30 & 11:00 & Coffeee break \\
11:00 & 13:00 & Practical---choose one: \newline
                1: Practice with basic \R\ \&  Reading data into \R\ \newline
                2: Data manipulation with \texttt{tidyverse} \\

% Students can choose:  \newline EITHER -- Practice with basic \R\ \& Simple data input (MP, 
% based on old exercises 1.1-1.2 to be trimmed according to the postmortem of 2019) for new users of \R\, \newline 
% OR -- Data manipulation with \texttt{dplyr} (DG, based on old 1.4) for \newline experienced \R\ users
% (to avoid these people getting bored in the beginning). \\
  
13:00 & 14:00 & Lunch \\
14:00 & 14:30 & Recap of morning practical \\
14:30 & 15:30 & Practical: Tabulation \& Graphics in \R\ (KF)\\ %(KF, based on old 1.3, 1.5 \newline -- especially the latter part sufficiently shortened from last time)  \\
15:30 & 16:00 & Tea break\\
16:00 & 16:30 & Simple Poisson and binary regression (JP)\\ %(JP, 30 min shorter than previously 
% $\Rightarrow$ max 15 slides! Leave at least case-controlling out?) \\
<<<<<<< HEAD
16:30 & 18:00 & Practical: Incidence rates and proportions and their contrasts by \texttt{glm} \\ 
=======
16:30 & 18:00 & Practical: Analysis of hazard rates, their ratios and differences and binary regression\\
>>>>>>> 5791fd7f29f5ff10cf0a981642ba6fefd796a94c
                % (JP, old 1.7 to be shortened but add a couple of simple items on binary regression
                % of \texttt{lowbw} on \texttt{hyp} in \texttt{births} data?) \\ 
18:00 & 21:00 & Welcome reception at Delta terrace ($4^\text{th}$ floor) \\[1em]
\end{tabular}

\noindent
\begin{tabular}{r@{ -- }rp{13cm}}
\multicolumn{3}{l}{\bf Saturday 3 June -- Day 2} \\
 9:00 &  9:30 & Recap of yesterday's practicals. \\
 9:30 & 10:00 & Introduction to splines (MP)\\ % (MP, 15 min shorter than previously, remote mode) \\
10:00 & 10:45 & Linear and generalized linear models (EL)\\ % EL, 15 min longer than before; \newline add a few things on GLMs) \\
10:45 & 11:15 & Coffeee break \\
<<<<<<< HEAD
11:15 & 13:00 & Practical: Linear, logistic \& Poisson regression  
                with estimation and reporting of linear and curved
                effects\\ %(EL, old exercises 1.9-1.10 to be revised) \\
=======
11:15 & 13:00 & Practical:  Estimation of effects: simple and more complex \& Poisson regression \& analysis of curved effects\\ %(EL, old exercises 1.9-1.10 to be revised) \\
>>>>>>> 5791fd7f29f5ff10cf0a981642ba6fefd796a94c
13:00 & 14:00 & Lunch \\
14:00 & 14:30 & Recap of morning practical \\
14:30 & 15:30 & Causal inference 1: basic concepts\\ % (KF, to be revised) \\
15:30 & 16:00 & Tea break\\
16:00 & 18:00 & Practical: Causal inference\\[1em] % (KF, old 1.14 to be revised) \\        
\end{tabular}

\noindent
\begin{tabular}{r@{ -- }rp{13cm}}
 \multicolumn{3}{l}{\bf Sunday 4 June -- Day 3} \\
 9:00 &  9:30 & Recap of yesterday's practicals \\
 9:30 & 10:30 & More advanced graphics in R, including
                \texttt{ggplot2} (MP)\\ % (MP, remote mode \newline -- or DG on the spot?)\\
10:30 & 11:00 & Coffee break. \\
11:00 & 13:00 & Practical: Graphics meccano \\% (MP, old 1.11) \\
15:00 & 17:00 & Old Town Adventure \newline \url{https://360.ee/en/team-building-games/old-town-adventure-3/}\\[1em]
\end{tabular}

\noindent
\begin{tabular}{r@{ -- }rp{13cm}}
\multicolumn{3}{l}{\bf Monday 5 June -- Day 4} \\
 9:00 &  9:30 & Recap of yesterday's practicals \\
 9:30 & 10:30 & Survival analysis: Kaplan Meier \& basic
                Cox-model, and basic analysis of competing risks (JP)\\ % (JP, perhaps some revision?)\\
10:30 & 11:00 & Coffee break\\
11:00 & 13:00 & Practical: Survival analysis with competing risks: Oral cancer patients\\ % (JP, old 1.12 to be somewhat revised)\\
13:00 & 14:00 & Lunch \\
14:00 & 14:30 & Recap of morning practical \\
14:30 & 15:30 & Dates in R; follow-up representation in \texttt{Lexis} objects,
                time-splitting,  and SMR (BxC)\\
15:30 & 16:00 & Tea break. \\
16:00 & 18:00 & Practical: Time-splitting, time-scales and SMR\\[1em] % (old exercise 1.13 to be revised, BxC)\\
% 18:30 &       & Tour of the Estonian Genome Centre(?)\\[1em]
\end{tabular}

\noindent
\begin{tabular}{r@{ -- }rp{13cm}}
\multicolumn{3}{l}{\bf Tuesday 6 June -- Day 5} \\
 9:00 &  9:30 & Recap of yesterday's practicals \\
 9:30 & 10:30 & Nested and matched cc-studies \& Case-cohort studies (KF)\\
  %(KF -- EL gave this lecture before) \\
10:30 & 11:00 & Coffee break. \\
11:00 & 13:00 & Practical: Nested case-control study and case-cohort study: \\ % (KF) old exercise 1.15 prepared by EL perhaps to be slightly revised)\\
13:00 & 14:00 & Lunch \\
14:00 & 14:30 & Recap of morning practical \\
14:30 & 15:30 & Causal inference 2: model-based estimation of causal
                contrasts (EL)\\ % \newline -- a totally new lecture)\\
15:30 & 16:00 & Tea break. \\
16:00 & 18:00 & Practical: Causal inference 2: Model-based estimation of causal estimands \\ %\newline -- a totally new exercise)\\
19:00 & 22:00 & Course dinner, Vilde \& Vine\\[1em]
\end{tabular}

\noindent
\begin{tabular}{r@{ -- }rp{13cm}}
\multicolumn{3}{l}{\bf Wednesday 7 June -- Day 6} \\
 9:00 &  9:30 & Recap of yesterday's practicals \\
 9:30 & 10:30 & Multistate models, Poisson models for rates and
                simulation of \texttt{Lexis} objects (BxC)\\
10:30 & 11:00 & Coffee break. \\
11:00 & 12:15 & Practical: Time-dependent variables and multiple states\\ % (BxC, old 1.16 to be revised)\\
12:15 & 12:45 & Recap of morning practical \\
12:45 & 13:00 & Wrap-up and farewell.\\
13:00 & 14:00 & Lunch \\
% \multicolumn{2}{l}{Afternoon} & Post-mortem (Faculty only). \\
% \multicolumn{2}{l}{Evening}   & Faculty dinner.\\
\end{tabular}
\vfill
\noindent
Further material will appear at this year's course website:\\ 
\url{http://bendixcarstensen.com/SPE}

\vfill

\end{document}

\newpage
Here is an overview of who is doing what:
\begin{Schunk}
\begin{Sinput}
> options( width=135 )
> wh <- read.table("load.dat", header=T, as.is=T )
> wh
\end{Sinput}
\begin{Soutput}
   length type                                                            topic teacher
1      75    L            Introduction to R language and commands, reading data      MP
2       1    P Reading data, simple tabulation, data frames, with and without $      MP
3      60    L           Language, indexing, Simple simulation, simple graphics      KF
4       1    P                      Tabulation and simple estimation of effects      EL
5      60    L                    Linear models, fitting, ci.lin, simple spline      EL
6       1    P                                                       Using effx      MP
7       1    P           Estimating a linear, quadratic and reporting the graph     BxC
8      30    L                               Logistic regression for cc-studies      JP
9      30    L                         Poisson regression for follow-up studies      JP
10      1    P                 Simple exercise on rates, RR, RD, and 2x2 tables     BxC
11      1    P                                       Simple logistic regression      KF
12     60    L                                      More advanced graphics in R      MP
13      1    P    Graphics meccano: how to build an informative graph / ggplot2      MP
14     60    L    Survival analysis, KM, simple Cox, simple comp risk, rel.surv      JP
15      1    P                 Practical on relative survival / additive hazard      JP
16      1    P                                             Oral cancer survival      EL
17     60    L                 Dates; follow up representation in Lexis objects     BxC
18      1    P                                    Danish diabetes register data     BxC
19     60    L                                                 Causal inference      KF
20      1    P            Simulation for causal inference and linear regression      KF
21     60    L                  Nested, matched cc-studies. Case-cohort studies      EL
22      1    P                                       Matched case-control study      EL
23      1    P                                                Case-cohort study      EL
24     60    L               Multistate models, simulation in multistate models     BxC
25      1    P               Exercise on multistate models: Renal complications     BxC
\end{Soutput}
\begin{Sinput}
> str(wh)
\end{Sinput}
\begin{Soutput}
'data.frame':	25 obs. of  4 variables:
 $ length : int  75 1 60 1 60 1 1 30 30 1 ...
 $ type   : chr  "L" "P" "L" "P" ...
 $ topic  : chr  "Introduction to R language and commands, reading data" "Reading data, simple tabulation, data frames, with and without $" "Language, indexing, Simple simulation, simple graphics" "Tabulation and simple estimation of effects" ...
 $ teacher: chr  "MP" "MP" "KF" "EL" ...
\end{Soutput}
\end{Schunk}
Here is the calculation of the lecturing loads, note there is no
accounting of the actual preparation load (due to earlier lecturing on
the topic) has been taken.
\begin{Schunk}
\begin{Sinput}
> LL <- with( subset(wh,type=="L"), tapply(length, list(topic,teacher), sum ) )
> LL[is.na(LL)] <- 0
> str(LL)
\end{Sinput}
\begin{Soutput}
 num [1:11, 1:5] 0 60 0 0 0 0 0 60 0 0 ...
 - attr(*, "dimnames")=List of 2
  ..$ : chr [1:11] "Causal inference" "Dates; follow up representation in Lexis objects" "Introduction to R language and commands, reading data" "Language, indexing, Simple simulation, simple graphics" ...
  ..$ : chr [1:5] "BxC" "EL" "JP" "KF" ...
\end{Soutput}
\begin{Sinput}
> print.table(addmargins(LL,1),ze=".")
\end{Sinput}
\begin{Soutput}
                                                              BxC  EL  JP  KF  MP
Causal inference                                                .   .   .  60   .
Dates; follow up representation in Lexis objects               60   .   .   .   .
Introduction to R language and commands, reading data           .   .   .   .  75
Language, indexing, Simple simulation, simple graphics          .   .   .  60   .
Linear models, fitting, ci.lin, simple spline                   .  60   .   .   .
Logistic regression for cc-studies                              .   .  30   .   .
More advanced graphics in R                                     .   .   .   .  60
Multistate models, simulation in multistate models             60   .   .   .   .
Nested, matched cc-studies. Case-cohort studies                 .  60   .   .   .
Poisson regression for follow-up studies                        .   .  30   .   .
Survival analysis, KM, simple Cox, simple comp risk, rel.surv   .   .  60   .   .
Sum                                                           120 120 120 120 135
\end{Soutput}
\end{Schunk}
And here is the assignment of practicals:
\begin{Schunk}
\begin{Sinput}
> PP <- with( subset(wh,type=="P"), tapply(length, list(topic,teacher), sum ) )
> PP[is.na(PP)] <- 0
> str(PP)
\end{Sinput}
\begin{Soutput}
 num [1:14, 1:5] 0 1 1 1 0 0 0 0 0 1 ...
 - attr(*, "dimnames")=List of 2
  ..$ : chr [1:14] "Case-cohort study" "Danish diabetes register data" "Estimating a linear, quadratic and reporting the graph" "Exercise on multistate models: Renal complications" ...
  ..$ : chr [1:5] "BxC" "EL" "JP" "KF" ...
\end{Soutput}
\begin{Sinput}
> print.table(addmargins(PP,1),ze=".")
\end{Sinput}
\begin{Soutput}
                                                                 BxC EL JP KF MP
Case-cohort study                                                  .  1  .  .  .
Danish diabetes register data                                      1  .  .  .  .
Estimating a linear, quadratic and reporting the graph             1  .  .  .  .
Exercise on multistate models: Renal complications                 1  .  .  .  .
Graphics meccano: how to build an informative graph / ggplot2      .  .  .  .  1
Matched case-control study                                         .  1  .  .  .
Oral cancer survival                                               .  1  .  .  .
Practical on relative survival / additive hazard                   .  .  1  .  .
Reading data, simple tabulation, data frames, with and without $   .  .  .  .  1
Simple exercise on rates, RR, RD, and 2x2 tables                   1  .  .  .  .
Simple logistic regression                                         .  .  .  1  .
Simulation for causal inference and linear regression              .  .  .  1  .
Tabulation and simple estimation of effects                        .  1  .  .  .
Using effx                                                         .  .  .  .  1
Sum                                                                4  4  1  2  3
\end{Soutput}
\end{Schunk}

\end{document}
