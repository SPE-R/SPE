%% \documentclass[12pt,t  , handout  % Uncomment ",handout" when doing handouts
%% ]{beamer}

\documentclass[12pt,dvipsnames,t,handout% (Un)comment ",handout" when (not) doing handouts
,aspectratio=169% Comment out if you want 4 by 3 (default)
]{beamer}

\newcommand{\toggle}[1]{%
\addtocounter{framenumber}{-0}#1 % Comment this out if extra slides should be omitted
}
\newcommand{\toggleafter}[1]{%
#1\addtocounter{framenumber}{-0} % Comment this out if extra slides should be omitted
}

% \usepackage[latin1]{inputenc}
% \usepackage[danish]{babel}
% \usepackage{booktabs,amsmath,amsbsy}

\usepackage[latin1]{inputenc}
\usepackage{babel}
%\usepackage{graphicx,rotating,booktabs,verbatim,amsmath,amsbsy}
\usepackage{graphicx,rotating,booktabs,verbatim,amsmath}

%----------------------------------------------------------------------
% The general look of things:

% Theme with navigation bar on the right; [width=0em] removes it
\mode<presentation>{\usetheme[width=0em]{Goettingen}}

% Omit the navigation symbols --- I have pg-up and pg-dn anyway
\setbeamertemplate{navigation symbols}{}

% Pagenumbering at the far right bottom corner
\setbeamertemplate{footline}{
  \usebeamercolor[fg]{frametitle}
  \hspace*{3ex}\currentlecture % This is for inserting title of the lecture
  \hfill \bf \insertframenumber / \inserttotalframenumber
  \rule[-2ex]{0pt}{5ex} \hspace*{3ex}}

% How visible should the uncovered items be? 0 corresponds to not at all.
\setbeamercovered{transparent=0}

% Get the fonts to look right in mathematics parts
\usefonttheme[onlymath]{serif}
% % This is a file of redefinitions of the mathematical functions
% so that they can be typeset in serif font in Beamer
\renewcommand{\arccos}[0]{\text{\rm arccos}}
\renewcommand{\arcsin}[0]{\text{\rm arcsin}}
\renewcommand{\arctan}[0]{\text{\rm arctan}}
\renewcommand{\arg}[0]{\text{\rm arg}}
\renewcommand{\cos}[0]{\text{\rm cos}}
\renewcommand{\cosh}[0]{\text{\rm cosh}}
\renewcommand{\cot}[0]{\text{\rm cot}}
\renewcommand{\coth}[0]{\text{\rm coth}}
\renewcommand{\csc}[0]{\text{\rm csc}}
\renewcommand{\deg}[0]{\text{\rm deg}}
\renewcommand{\det}[0]{\text{\rm det}}
\renewcommand{\dim}[0]{\text{\rm dim}}
\renewcommand{\exp}[0]{\text{\rm exp}}
\renewcommand{\gcd}[0]{\text{\rm gcd}}
\renewcommand{\hom}[0]{\text{\rm hom}}
\renewcommand{\inf}[0]{\text{\rm inf}}
\renewcommand{\ker}[0]{\text{\rm ker}}
\renewcommand{\lg}[0]{\text{\rm lg}}
\renewcommand{\lim}[0]{\text{\rm lim}}
\renewcommand{\liminf}[0]{\text{\rm liminf}}
\renewcommand{\limsup}[0]{\text{\rm limsup}}
\renewcommand{\ln}[0]{\text{\rm ln}}
\renewcommand{\log}[0]{\text{\rm log}}
\renewcommand{\max}[0]{\text{\rm max}}
\renewcommand{\min}[0]{\text{\rm min}}
\renewcommand{\Pr}[0]{\text{\rm Pr}}
\renewcommand{\sec}[0]{\text{\rm sec}}
\renewcommand{\sin}[0]{\text{\rm sin}}
\renewcommand{\sinh}[0]{\text{\rm sinh}}
\renewcommand{\sup}[0]{\text{\rm sup}}
\renewcommand{\tan}[0]{\text{\rm tan}}
\renewcommand{\tanh}[0]{\text{\rm tanh}}

% Extra functions in math style
\newcommand{\pr}[0]{\text{\rm Pr}}
\newcommand{\var}{\text{\rm var}}
\newcommand{\cov}{\text{\rm cov}}
\newcommand{\corr}{\text{\rm corr}}
\newcommand{\mean}{\text{\rm mean}}
\newcommand{\median}{\text{\rm median}}
\newcommand{\p}{{\mathrm p}}
\newcommand{\e}{{\mathrm e}}
\newcommand{\D}{{\mathrm D}}
\newcommand{\dif}{{\,\mathrm d}}
% \newcommand{\Pp}{P}
\newcommand{\pmat}[1]{\Pp\left\{#1\right\}}
\newcommand{\ptxt}[1]{\Pp\left\{\text{#1}\right\}}
% \newcommand{\pmat}[1]{\text{\rm P}\left\{#1\right\}}
% \newcommand{\ptxt}[1]{\text{\rm P}\left\{\text{#1}\right\}}
\newcommand{\E}{\text{\rm E}}
\newcommand{\V}{\text{\rm V}}
\newcommand{\BLUP}{\text{\rm BLUP}}
\newcommand{\se}{\text{\rm s.e.}}
\newcommand{\sem}{\text{\rm s.e.m.}}
\newcommand{\std}{\text{\rm std}}
\newcommand{\sd}{\text{\rm s.d.}}
\newcommand{\cv}{\text{\rm c.v.}}
\newcommand{\CV}{\text{\rm CV}}
\newcommand{\erf}{\text{\rm erf}}
\newcommand{\ef}{\text{\rm ef}}
\newcommand{\SSD}{\text{\rm SSD}}
\newcommand{\SPD}{\text{\rm SPD}}
\newcommand{\odds}{\text{\rm odds}}
\newcommand{\bin}{\text{\rm binom}}
\newcommand{\diag}{\text{\rm diag}}
\newcommand{\spcol}{\text{\rm span}}
\newcommand{\logit}{\text{\rm logit}}
% Intentional typo to avoid name conflict
\newcommand{\lnik}{\text{\rm link}}
\newcommand{\spn}{\text{\rm span}}
\newcommand{\CI}{\text{\rm CI}}
\newcommand{\IP}{\text{\rm IP}}
\newcommand{\OR}{\text{\rm OR}}
\newcommand{\RR}{\text{\rm RR}}
\newcommand{\ER}{\text{\rm ER}}
\newcommand{\EM}{\text{\rm EM}}
\newcommand{\EF}{\text{\rm EF}}
\newcommand{\RD}{\text{\rm RD}}
\newcommand{\AC}{\text{\rm AC}}
\newcommand{\AF}{\text{\rm AF}}
\newcommand{\PAF}{\text{\rm PAF}}
\newcommand{\AR}{\text{\rm AR}}
\newcommand{\CR}{\text{\rm CR}}
\newcommand{\PAR}{\text{\rm PAR}}
\newcommand{\SD}{\text{\rm SD}}
\newcommand{\SE}{\text{\rm SE}}
\newcommand{\SEM}{\text{\rm SEM}}
\newcommand{\SR}{\text{\rm SR}}
\newcommand{\SMR}{\text{\rm SMR}}
\newcommand{\RSR}{\text{\rm RSR}}
\newcommand{\CMF}{\text{\rm CMF}}
\newcommand{\pvp}{\text{\rm PV$+$}}
\newcommand{\pvn}{\text{\rm PV$-$}}
\newcommand{\T}{\text{\rm \small{T}}}
\newcommand{\id}{\text{\rm id}}
\newcommand{\Dev}{\text{\rm Dev}}
\newcommand{\Bin}{\text{\rm Bin}}
\newcommand{\probit}{\text{\rm probit}}
\newcommand{\cloglog}{\text{\rm cloglog}}

% Conditional independence sign from Philip Dawid
\newcommand{\cip}{\mbox{$\perp\!\!\!\perp$}}
\newcommand{\half}{\frac{1}{2}}
% Multoply / division
\newcommand{\mpydiv}[0]{\stackrel{\scriptstyle \times}{\scriptstyle \div}}
\newcommand{\td}[0]{\stackrel{\scriptstyle \times}{\scriptstyle \div}}
\newcommand{\dt}[0]{\stackrel{\scriptstyle \div}{\scriptstyle \times}}
 % A re-definition of all math
                        % commands (and some more) that
                        % makes them appear in serif font

% This is a file of useful extra commands snatched from
% Michael Hills, David Clayton, Bendix Carstensen & Esa Laara.
%

% Commands to draw observation lines on follow-up diagrams
%
% Horizontal lines
%

% exit time with failure, bullet
\newcommand{\hfail}[1]{\begin{picture}(250,5)
       \put(0,0){\line(0,1){2.5}}
      \put(0,0){\line(0,-1){2.5}}
      \put(0,0){\line(1,0){#1}}
      \put(#1,0){\circle*{5}}
   \end{picture}}

% exit time with censoring, open circle
\newcommand{\hcens}[1]{\begin{picture}(250,5)
         \put(0,0){\line(0,1){2.5}}
      \put(0,0){\line(0,-1){2.5}}
      \put(0,0){\line(1,0){#1}}
%      \put(#1,0){\line(0,1){2.5}}
%      \put(#1,0){\line(0,-1){2.5}}
% BxC Changed this to an open circle instead of a line
      \put(#1,0){\circle{5}}
   \end{picture}}

%
% Diagonals for Lexis diagrams
%
\newcommand{\dfail}[1]{\begin{picture}(250,250)
      \put(0,0){\line(1,1){#1}}
      \put(#1,#1){\circle*{5}}
   \end{picture}}

\newcommand{\dcens}[1]{\begin{picture}(250,250)
      \put(0,0){\line(1,1){#1}}
%      \put(#1,#1){\line(0,1){2.5}}
%      \put(#1,#1){\line(0,-1){2.5}}
% BxC Changed this to an open circle instead of a line
      \put(#1,#1){\circle{5}}
   \end{picture}}

%
% Horizontal range diagrams
%
\newcommand{\hrange}[1]{\begin{picture}(200,5)
     \put(0,0){\circle*{5}}
     \put(0,0){\line(1,0){#1}}
     \put(0,0){\line(-1,0){#1}}
   \end{picture}}

%
% Tree drawing
%
\newcommand{\tree}[3]{\setlength{\unitlength}{#1}\begin{picture}(0,0)
   \put(0,0){\line(3,2){1}}
   \put(0,0){\line(3,-2){1}}
   \put(0.81,0.54){\makebox(0,0)[br]{\footnotesize #2\ }}
   \put(0.81,-0.54){\makebox(0,0)[tr]{\footnotesize #3\ }}
\end{picture}}

\newcommand{\wtree}[3]{\setlength{\unitlength}{#1}\begin{picture}(0,0)
   \put(0,0){\line(1,1){1}}
   \put(0,0){\line(1,-1){1}}
   \put(0.8,0.8){\makebox(0,0)[br]{\footnotesize #2\ }}
   \put(0.8,-0.8){\makebox(0,0)[tr]{\footnotesize #3\ }}
\end{picture}}

\newcommand{\ntree}[3]{\setlength{\unitlength}{#1}\begin{picture}(0,0)
   \put(0,0){\line(2,1){1}}
   \put(0,0){\line(2,-1){1}}
   \put(0.8,0.4){\makebox(0,0)[br]{\footnotesize #2\ }}
   \put(0.8,-0.4){\makebox(0,0)[tr]{\footnotesize #3\ }}
\end{picture}}

%
% Other commands
%
\newcommand{\T}{\scriptsize\text T}
\newcommand{\prob}[0]{\text{\rm Pr}}
\newcommand{\nhy}[0]{_{\oslash}}
\newcommand{\true}[0]{_{\text{\rm \tiny T}}}
\newcommand{\hyp}[0]{_{\text{\rm \tiny H}}}
% \newcommand{\mpydiv}[0]{\stackrel{\textstyle \times}{\div}}
% Changed to slightly smaller symbols
\newcommand{\mpydiv}[0]{\stackrel{\times}{\scriptstyle \div}}
\newcommand{\mie}[1]{{\it #1}}
\newcommand{\mycircle}[0]{\circle*{5}}
\newcommand{\smcircle}[0]{\circle*{1}}
\newcommand{\corner}[0]{_{\text{\rm \tiny C}}}
\newcommand{\ind}[0]{\hspace{10pt}}
\newcommand{\gap}[0]{\\[5pt]}
\renewcommand{\S}[0]{section~}
\newcommand{\blank}[0]{$\;\,$}
\newcommand{\vone}{\vspace{1cm}}
\newcommand{\ljust}[1]{\multicolumn{1}{l}{#1}}
\newcommand{\cjust}[1]{\multicolumn{1}{c}{#1}}
\newcommand{\mean}{\text{\rm Mean}}
\newcommand{\transpose}{^{\mbox{\tiny T}}}
\newcommand{\histog}[5]{\rule{1mm}{#1mm}\,\rule{1mm}{#2mm}\,\rule{1mm}{#3mm}\,\rule{1mm}{#4mm}\,\rule{1mm}{#5mm}}
\newcommand{\pmiss}{P_{\mbox{\tiny miss}}}
%
% Below is BxCs commands inserted
%
\newcommand{\bc}{\begin{center}}
\newcommand{\ec}{\end{center}}

\newcommand{\bd}{\setlength{\parskip}{1ex} \begin{description}}
\newcommand{\ed}{\end{description} \setlength{\parskip}{2ex}}
\newcommand{\bdx}{\begin{description}} % Bendix' description macros
\newcommand{\edx}{\end{description}}

\newcommand{\bix}{\begin{itemize}}  % these are Bendix' itemizing macros
\newcommand{\eix}{\end{itemize}}
\newcommand{\bi}{\setlength{\parskip}{1ex} \begin{itemize}} % Esa's item macros 
\newcommand{\ei}{\end{itemize} \setlength{\parskip}{2ex}} 

\newcommand{\bn}{\begin{equation}}
\newcommand{\en}{\end{equation}}
\newcommand{\be}{\begin{enumerate}}
\newcommand{\ee}{\end{enumerate}}
\newcommand{\bes}{\begin{eqnarray*}}
\newcommand{\ees}{\end{eqnarray*}}
\newcommand{\p}{\text{\rm P}}
\newcommand{\pmat}[1]{\text{\rm P}\left\{#1\right\}}
\newcommand{\ptxt}[1]{\text{\rm P}\left\{\text{\rm #1}\right\}}
\newcommand{\E}{\text{\rm E}}
\newcommand{\V}{\text{\rm V}}
\newcommand{\BLUP}{\text{\rm BLUP}}

% \newcommand{\var}{\mbox{Var}} changed by Esa to
\newcommand{\var}{\mbox{var}}
% \newcommand{\cov}{\mbox{Cov}} changed by Esa to
\newcommand{\cov}{\mbox{cov}}
% \newcommand{\corr}{\mbox{Corr}} changed by Esa to
\newcommand{\corr}{\mbox{corr}} 

%\newcommand{\var}{\text{\rm var}}
%\newcommand{\cov}{\text{\rm cov}}
%\newcommand{\corr}{\text{\rm corr}}
\newcommand{\se}{\text{\rm s.e.}}
\newcommand{\sd}{\text{\rm std}}
\newcommand{\erf}{\text{\rm erf}}
\newcommand{\odds}{\text{\rm odds}}
\newcommand{\bin}{\text{\rm binom}}
\newcommand{\half}[1]{\frac{1}{#1}}
% \newcommand{\td}[0]{\stackrel{\textstyle \times}{\div}}
% Changed to slightly smaller symbols
\newcommand{\td}[0]{\stackrel{\scriptstyle \times}{\scriptstyle \div}}
\newcommand{\logit}{\text{\rm logit}}
\newcommand{\link}{\text{\rm link}}
\newcommand{\spn}{\text{\rm span}}
\newcommand{\OR}{\text{\rm OR}}
\newcommand{\RR}{\text{\rm RR}}
\newcommand{\ER}{\text{\rm ER}}
\newcommand{\RD}{\text{\rm RD}}
\newcommand{\AC}{\text{\rm AC}}
\newcommand{\AF}{\text{\rm AF}}
\newcommand{\PAF}{\text{\rm PAF}}
\newcommand{\SR}{\text{\rm SR}}
\newcommand{\SMR}{\text{\rm SMR}}
\newcommand{\CMF}{\text{\rm CMF}}
\newcommand{\pvp}{\text{\rm PV}$+$}
\newcommand{\pvn}{\text{\rm PV}$-$}
\newcommand{\R}{\textsf{R}}
%\newcommand{\gap}[0]{\\[5pt]} 
%\newcommand{\blank}[0]{$\;\,$}
% Conditional independence sign from Philip Dawid
\newcommand{\cip}{\mbox{$\perp\!\!\!\perp$}}

%%% Commands to comment out parts of the text
\newcommand{\GLEM}[1]{}
\newcommand{\FORGETIT}[1]{}
\newcommand{\OMIT}[1]{}

%%% Insert output from program in small text 
%%% (requires package verbatim)

\newcommand{\insout}[1]{
\scriptsize
\renewcommand{\baselinestretch}{0.8}
\verbatiminput{#1}
\renewcommand{\baselinestretch}{1.0}
\normalsize
}

% From Esa:        
%\newcommand{\T}{\text{\rm \small{T}}}
\newcommand{\id}{\text{\rm id}}
\newcommand{\Dev}{\text{\rm Dev}}
\newcommand{\Bin}{\text{\rm Bin}}
\newcommand{\probit}{\text{\rm probit}}
\newcommand{\cloglog}{\text{\rm cloglog}}
\newcommand{\EF}{\text{\rm EF}}
\newcommand{\SE}{\text{\rm SE}}
\newcommand{\IP}{\text{\rm IP}}




% The heading font is a little too thin to my taste
\setbeamerfont{frametitle}{size=\large,series=\bfseries}

% Use pdf graphs
\DeclareGraphicsExtensions{.pdf,.jpg}

% End of setting up the formal layout of the slides

% Definition of a dummy command so that the above works and so that ALL
% redefinitions can be done by \renewcommand{\currentlecture}
\newcommand{\currentlecture}{}

% A banner page to include and separate lectures

\newcommand{\banner}[3]{
\addtocounter{framenumber}{-0}
\section{#1}
\renewcommand{\currentlecture}{#1}
\begin{frame}[plain]
{\usebeamercolor[fg]{frametitle}
 \LARGE \bf #1\\
 \large \sf #2\\
 \large \bf #3}
\vfill
Statistical Practice in Epidemiology with R\\
%%University of Tartu, Estonia\\
 International Agency for Research on Cancer \\
June, 2024\\
\end{frame}
%\input{#4}
}

% Tree drawing is from useful.tex which won;t load since it produces an error
% when trying to overwrite existing definitions

%
% Tree drawing
%
%\newcommand{\tree}[3]{\setlength{\unitlength}{#1}\begin{picture}(0,0)
 %  \put(0,0){\line(3,2){1}}
 %  \put(0,0){\line(3,-2){1}}
  % \put(0.81,0.54){\makebox(0,0)[br]{\footnotesize #2\ }}
   %\put(0.81,-0.54){\makebox(0,0)[tr]{\footnotesize #3\ }}
%\end{picture}}

%\newcommand{\wtree}[3]{\setlength{\unitlength}{#1}\begin{picture}(0,0)
 %  \put(0,0){\line(1,1){1}}
  % \put(0,0){\line(1,-1){1}}
   %\put(0.8,0.8){\makebox(0,0)[br]{\footnotesize #2\ }}
   %\put(0.8,-0.8){\makebox(0,0)[tr]{\footnotesize #3\ }}
%\end{picture}}

%\newcommand{\ntree}[3]{\setlength{\unitlength}{#1}\begin{picture}(0,0)
 %  \put(0,0){\line(2,1){1}}
  % \put(0,0){\line(2,-1){1}}
 %  \put(0.8,0.4){\makebox(0,0)[br]{\footnotesize #2\ }}
 %  \put(0.8,-0.4){\makebox(0,0)[tr]{\footnotesize #3\ }}
%\end{picture}}

%\newcommand{\nutree}[3]{\begin{picture}(0,0)
 %  \put(0,0){\line(2,1){#1}}
 %  \put(0,0){\line(2,-1){#1}}
 %  \put(0.8,0.4){\makebox(0,0)[br]{#2\ }}
 %  \put(0.8,-0.4){\makebox(0,0)[tr]{#3\ }}
%\end{picture}}

%\newcommand{\bes}{\begin{eqnarray*}}
%\newcommand{\ees}{\end{eqnarray*}}

% Commands to draw observation lines on follow-up diagrams
%
% Horizontal lines
%
%\newcommand{\hfail}[1]{\begin{picture}(250,5)
 %     \put(0,0){\line(1,0){#1}}
  %    \put(#1,0){\circle*{5}}
  % \end{picture}}

%\newcommand{\hcens}[1]{\begin{picture}(250,5)
 %     \put(0,0){\line(1,0){#1}}
  %    \put(#1,0){\line(0,1){2.5}}
   %   \put(#1,0){\line(0,-1){2.5}}
   %\end{picture}}

\definecolor{darkgreen}{rgb}{0,.5,0}

%----------------------------------------------------------------------
\begin{document}
% It is more readable with a little extra space between paragraphs
\parskip 0.8ex

% \AtBeginSubsection[]
% {
%   \begin{frame}<beamer>
%     \frametitle{Outline}
%     \tableofcontents[currentsection,currentsubsection]
%   \end{frame}
% }
% \beamerdefaultoverlayspecification{<+->}

%----------------------------------------------------------------------
% The default titlepage is horribly formatted, so a custom made one:
\addtocounter{framenumber}{-1}
\begin{comment}
\begin{frame}[plain]
{\usebeamercolor[fg]{frametitle}
 \LARGE \bf Statistical Practice in Epidemiology with R}
\vfill
{\footnotesize
 \bf Bendix Carstensen\sf, Steno Diabetes Center, Copenhagen\\%[1ex]
% \bf Peter Dalgaard\sf, Dept. of Biostatistics, 
% University of Copenhagen\\%[1ex]
 \bf Krista Fischer\sf, Dept. of Biostatistics, University of Tartu\\%[1ex]
% \bf Lyle Gurrin\sf, School of Population Health,
% University of Melbourne\\%[1ex]
% \bf Michael Hills\sf, (retired), Highgate, London\\%[1ex]
 \bf Esa L??r?\sf, Unit of Math. Sciences, University of Oulu\\%[1ex]
 \bf Janne Pitk?niemi\sf, Finnish Cancer Registry\\
 \bf Martyn Plummer\sf, IARC, Lyon\\%[1ex]
}
\normalsize
\vfill
2 to 7 June, 2023\\
%% International Agency for Research on Cancer,\\ Lyon, France.
University of Tartu, Estonia
\end{frame}
\end{comment}

%----------------------------------------------------------------------
% Just for the annotation of the sidebar since we do not use \titlepage
\title{Statistical Practice in Epidemiology}
%\author{BxC, KF, EL, JP, MP}

% %----------------------------------------------------------------------
% \begin{frame}
%   \frametitle{Outline}
%   \tableofcontents
%   % You might wish to add the option [pausesections]
% \end{frame}

%----------------------------------------------------------------------
% Here is the input of the individual lectures including the chapter
% headings, timing and lecturer

%%% Tuesday 27
% \banner{Simple Poisson and logistic regression}%
%        {Saturday 25 May 2007, morning}%
%        {Bendix Carstensen}%
%        {.bxc/Poisson-logreg}

%\banner{Parametrization of models}%
%       {Saturday 25 May 2007, morning}%
%       {Bendix Carstensen}
% \input{./bxc/param}     % These files assume that the graphs are in
% \input{./bxc/interact}  % ./bxc/graph (i.e. refer to ./bxc/graph/x.pdf)

%% Tuesday 29
 \banner{Nested case-control and case-cohort studies}%
       {Thursday, 06 June, 2024  }%
        {  Esa L��r� \& Martyn Plummer \& Krista Fischer}
%        {}
%----------------------------------------------------------------------

\begin{frame}
\frametitle{Points to be covered}

\begin{itemize}
\item
Outcome-dependent sampling designs a.k.a. \\ 
\textbf{case-control} studies {\it vs.} 
\textbf{full cohort} design.
\medskip
\item
\textbf{Nested case-control} study (NCC): sampling 
of controls from risk-sets during follow-up of study population.
\medskip
\item
\textbf{Matching} in selection of control subjects in NCC.
\medskip
\item
R tools for NCC: function {\tt ccwc()}  in {\tt Epi} 
for sampling controls, and
 {\tt clogit()} in {\tt survival} 
 for model fitting.
\medskip
\item
\textbf{Case-cohort} study (CC): sampling a subcohort from the
whole cohort as it is at the start of follow-up.
\medskip
\item
R tools for CC model fitting: function {\tt cch()} in \texttt{survival}
 % {\tt coxph()} % but SEs, CIs {\it etc.} with additional tricks. 
 % {\tt robust=T} or using {\tt 'dfbeta'} residuals;
  % in package {\tt survival} 
\end{itemize}
\end{frame}


\begin{frame}
\frametitle{Example: Smoking and cervix cancer}% (Simen Kapeu 2006)}
\small
\pause
Study population,  measurements, follow-up, and sampling design
\begin{itemize}
\item 
Joint cohort of $N \approx$ 500 000 women from 3
Nordic biobanks. 
% (in Norway, Sweden and Finland). \pause %\medskip
% \item Recruitment \& taking blood specimens started 1975. 
% \ei
%
% Study subjects 
% \bi
\pause %\medskip
\item Follow-up: From variable entry times since 1970s till 2000.\pause
% \item 200 incident cases of cervix ca., 
% 596 controls. \pause % \medskip
\item For each of 200 cases, 3 controls were sampled; matched for biobank,  % region (NO), 
  age ($\pm 2$ y), and time of entry ($\pm 2$ mo).\pause % \medskip 
\item Frozen sera of cases and controls 
analyzed for cotinine {\it etc.}
% plus antibodies to   HPV16, 18, 33, and {\it C. trachomatis}.
\end{itemize}
\pause
Main result: Adjusted OR = 1.5 (95\% CI 1.1 to 2.3) for high 
($>$242.6 ng/ml) 
 {\it vs.} low ($<$3.0 ng/ml) cotinine levels.

\bigskip
Simen Kapeu {\it et al.} (2009) {\it Am J Epidemiol}
\normalsize

\end{frame}

\begin{frame}
\frametitle{Example: \emph{USF1} gene and CVD}
 
 \small
Study population,  measurements, follow-up, and sampling design
\begin{itemize}
\item
Two FINRISK cohorts, total $N \approx$ 14000 M \& F, 25-64 y.
% \item 
\item Baseline health exam, questionnaire \& blood specimens
  at recruitment in the 1990s -- Follow-up until the end of 2003.
% taken and stored.
\item Subcohort of 786 subjects sampled.
% Study subjects 
% \bi
\pause
\item 
528 incident cases of CVD; 72 of them in the subcohort. \pause
\item Frozen blood from cases and subchort members 
genotyped. % for \emph{USF1}.
\end{itemize}
\pause
Main result: Female carriers of a high risk haplotype had a \\ 2-fold hazard
 of getting CVD [95\% CI: 1.2 to 3.5] % , adjusted for conventional risk factors.

\bigskip
Komulainen {\it et al.} (2006) \emph{PLoS Genetics}

\normalsize

\end{frame}

\begin{comment}

\begin{frame}
\frametitle{Full cohort design or ``cohort study''}

\pause
\bi
% \item Cohort = closed study population.
\item Cohort defined: eligibility criteria. % of membership.
\pause\medskip
\item
Possible selection of special groups of exposed and/or unexposed.
\pause\medskip
\item Risk factors of interest, 
confounders \& modifiers measured in  all cohort members.
  \pause\medskip
\item Cases of outcome are identified 
% during {\it follow-up} from {\it entry} time 
%  till {\it exit} time. 
  \pause\medskip 
\item Occurrence measures in exposure groups, and
  comparative measures are computed.
\pause\medskip
\item Confounding and modification are controlled by
   stratification \& Mantel--Haenszel methods, or \\
   regression models.
 \ei
\end{frame}


\begin{frame}
  \frametitle{Cohort follow-up and risk sets}
  
 \pause 
  Each member of the cohort provides exposure data, 
%  for all cases, 
  as long as (s)/he is {\bf at risk},
   \pause {\it i.e.} alive, not censored \& free from outcome.
  
  \ \\
\pause
  \begin{center}
  \begin{picture}(200,100)(0,0)

    \thicklines

    %Axes
    \put(0,0){\vector(1,0){200}}
    \put(200,0){Time}
    \put(0,0){\vector(0,1){100}}
    \put(0,105){Subjects}
    \put(0,10){\line(-1,0){3}}
    \put(0,20){\line(-1,0){3}}
    \put(0,30){\line(-1,0){3}}
    \put(0,40){\line(-1,0){3}}
    \put(0,50){\line(-1,0){3}}
    \put(0,60){\line(-1,0){3}}
    \put(0,70){\line(-1,0){3}}
    \put(0,80){\line(-1,0){3}}
    \put(0,90){\line(-1,0){3}}
    \pause    

    %first time step step
    \thinlines
    \put(0,10){\line(1,0){50}}
    \put(0,20){\line(1,0){50}}
    \put(0,30){\line(1,0){50}}
    \put(0,40){\line(1,0){50}}
    \put(0,50){\line(1,0){50}}
    \put(0,60){\line(1,0){30}} \put(30,57.5){\line(0,1){5}}
    \put(0,70){\line(1,0){50}}
    \put(0,80){\line(1,0){50}}
    \put(0,90){\line(1,0){50}}
    % Legend for controls
    \put(30,-10){\line(-1,0){5}}
    \put(30,-12.5){\line(0,1){5}}
    \put(35,-10){\makebox(0,0)[l]{Censored}}
    \pause

    %second time step
    \put(50,10){\line(1,0){50}}
    \put(50,20){\line(1,0){50}}
    \put(50,30){\line(1,0){30}} \put(80,27.5){\line(0,1){5}}
    \put(50,40){\line(1,0){50}}
    \put(50,50){\line(1,0){50}} 
    %Censored at time 80
    \put(50,70){\line(1,0){50}}
    \put(50,80){\line(1,0){50}}
    \put(50,90){\line(1,0){50}}
    \color{red} 
    \put(100,50){\circle*{5}} %case

    % Legend for case
    \put(100,-10){\circle*{5}}
    \color{black}
    \put(105,-10){\makebox(0,0)[l]{Case}}
    \pause
    
    %sampling frame
    \color{darkgreen} 
    \put(100,50){\oval(8,100)} 
    \pause
    
    %sampled controls
    \put(100,90){\circle{5}}
    \put(100,80){\circle{5}}
    \put(100,70){\circle{5}}
    \put(100,40){\circle{5}}
  %  \put(100,30){\circle{5}}
    \put(100,20){\circle{5}}
    \put(100,10){\circle{5}}
    % Legend for controls
    \put(150,-10){\circle{5}}
    \color{black}
    \put(155,-10){\makebox(0,0)[l]{At risk}}
    \pause

    %third time step
    \put(100,10){\line(1,0){50}}
    \put(100,20){\line(1,0){10}} \put(110,17.5){\line(0,1){5}}
    %Censored at time 130
    \put(100,40){\line(1,0){50}}
    %Case at time 150
    %\put(100,60){\line(1,0){50}}
    \put(100,70){\line(1,0){30}} \put(130,67.5){\line(0,1){5}}
    \put(100,80){\line(1,0){50}}
    \put(100,90){\line(1,0){50}}
    \color{red} 
    \put(150,10){\circle*{5}} %second case
    \pause

    %Second sampling frame
    \color{darkgreen} 
    \put(150,50){\oval(8,100)} 
    \pause

    %sampled controls
    \put(150,90){\circle{5}}
    \put(150,80){\circle{5}}
    \put(150,40){\circle{5}}
  %  \color{black}
    \put(105,105){Risk sets}
  \end{picture}
  \end{center}

\end{frame}

\end{comment}

\begin{frame}
\frametitle{Full cohort design \& its simple analysis}

\bi
\item
{\bf Full cohort design}: Data on exposure variables 
obtained for all subjects in a large study population.
\pause
\medskip
\item
Summary data for crude comparison:
\medskip
\begin{center}
\begin{tabular}{lccc}
\hline
          &  Exposed  &  Unexposed & Total\\
\hline
 Cases     &   $D_{1}$ &    $D_{0}$ & $D$ \\
 Non-cases  &  $B_{1}$  &  $B_{0}$  & $B$ \\
\hline    
 Group size at start  &  $N_{1}$  &  $N_{0}$ & $N$ \\
 Follow-up times      &  $Y_1 $ & $Y_0$  & $Y$\\ 
\hline 
\end{tabular}
\end{center}
\pause
\bigskip
\item  
Crude estimation of {\bf hazard ratio} $\rho = \lambda_1/\lambda_0$: \\
{\bf incidence rate ratio} IR, with standard error of log(IR):
% \begin{align*}
$$ \widehat\rho = \text{IR} % = \frac{I_1}{I_0}
   = \frac{ D_1/Y_1}{D_0/Y_0}  \qquad
  \text{SE}[\log(\text{IR})]  =  \sqrt{\frac{1}{D_1} + \frac{1}{D_0} }.
$$  
% \end{align*}
% \item      
% Statistical precision resides in numbers of cases.
\item More refined analyses: Poisson or Cox regression.
\ei

\end{frame}

\begin{comment}

\begin{frame}
  \frametitle{Example. HT and prevention of CVD}
  
 \pause  
 
 Nurses' Health Study (Grodstein {\it et al.} {\it AIM} 2000)
 \pause
 
 \begin{itemize}
 \item Cohort: $N \approx 70000$ postmenopausal women.
 \pause
 \item Use of HT asked biennially by questionnaire.
 \pause
 \item Follow-up 1976-96.
 \pause
 \item 1258 (major) coronary events, 767 strokes.
 \end{itemize} 
 \pause

Main result: IR of coronary events 0.61 [95\% CI: 0.52 to 0.71] for
 current vs. never users of HT, adjusted for known risk factors.
 \pause
\bi
\item[$\bullet$] HT reduces the risk of CVD -- or bias?
\pause
%\item[$\bullet$] Is this result due to bias?
\item[$\bullet$] Results from randomized trials are not quite 
concordant with this observational evidence  
({\it e.g.} WHI study, JAMA 2002).
% Veerus P. {\it et al.} 
% Results from the Estonian postmenopausal hormone therapy trial.
% {\it Maturitas} 2006; {\bf 55}: 162-173.
  
\ei
 
\end{frame}
 
\begin{frame}
 \frametitle{Example.  HT and CVD in NHS (cont'd)}

\pause

Summary of results
\begin{center}
\begin{tabular}{l r r }
\hline
 %              & \multicolumn{2}{c}{Use of HRT}\\
%               \cline{2-3}
							 & current HT & never HT \\
\hline
coronary events        & 259     & 662   \\
person-years   & 265 200   & 358 100 \\
incidence rate ($/10^4$y) & 9.8 & 18.5 \\
\hline
\end{tabular}
\end{center}
 \pause
Crude estimate, estimated
variance, 95\% error factor \& confidence interval of hazard  ratio
$\rho$:
\pause
\bes
   \text{IR} & = &\frac{9.8}{18.5} = 
    \frac{259 / 662 } {265200 / 358100} = 0.53  \\ 
    \pause
     V & = & 1 / 259 + 1 / 662 = 0.073^2  \\  
    \pause
  \text{EF} & = & \text{exp}(1.96 \times 0.073) = 1.1545 \\  
 \pause
95\%\text{CI} & = & [0.53 / 1.154, 0.53 \times 1.154] = [0.46, 0.61]
\ees
\end{frame}

\end{comment}

\begin{frame}
\frametitle{Problems with full cohort design}

% For rare diseases large cohorts needed to obtain
% adequate number of cases.

% \pause
Obtaining exposure and covariate data
\pause
\bi
\item Slow and expensive in a big cohort. \pause
\item Easier with questionnaire and register data, \pause
\item Extremely costly and laborious for {\it e.g.} 
\bi
{\normalsize
\item[--]   measurements from biological specimens, 
    like genotyping, antibody assays, {\it etc.} % \pause
    \medskip
\item[--] dietary diaries \& other manual records
%% \medskip
%% \item[--]  occupational exposure histories in manual records.
}
\ei
\ei
\pause

% \medskip
% Question: % \pause
\emph{ Can we obtain equally valid estimates of hazard ratios
etc. with nearly as good precision by some other
strategies}? 
\pause

\medskip
Yes -- we can!

\end{frame}

\begin{frame}
\frametitle{Estimation of hazard ratio}

% \pause
The incidence rate ratio can be expressed:
\pause
\bes
   \text{IR} =  \frac{ D_1/ D_0 }{ Y_1 / Y_0 }  
    =  \frac{\text{cases: exposed / unexposed}}
				    {\text{person-times: exposed / unexposed}}
				    \\ { } \\ 
	  \pause			    
              =  \frac{\textit{exp're odds} \text{ in cases}}
                 {\textit{exp're odds} \text{ in p-times}}   
             = \textbf{exposure odds ratio} \text{ (EOR)}  
\ees
\pause
= Exposure distribution in cases {\it vs.} that in cohort!

\pause
\bigskip
Implication for more efficient design: 
\pause
\bi 
\item {\it Numerator}: Collect exposure data on all cases. \pause
\item {\it Denominator}: Estimate the ratio of
person-times $Y_1/Y_0$ of the exposure groups in the cohort by
\textbf{sampling}  ``control'' subjects,
 on whom exposure is measured. 
% from members of the cohort at risk.
\ei
\end{frame}  
 

\begin{frame}
\frametitle{Case-control designs}

{General principle:} Sampling of subjects from a given {study population}
is {\it outcome-dependent}. \\ 

\pause
\bigskip
Data on risk factors are collected separately from
\pause
\begin{itemize}
\item[(I)] {\bf Case group}: All (or high \% of) the $D$ subjects 
in the study population (total $N$)
      encountering the outcome event during the follow-up.
	     \pause\bigskip
\item[(II)] {\bf Control group}:
          
 \item Random {\bf sample} (simple or stratified) of \\
 $C$ subjects ($C<<N$) from the population.
                  \pause\medskip
\item Eligible controls must be {bf risk} 
(alive, under follow-up \& free of outcome) at given time(s).   
%			      \pause\medskip
% \item Different \emph{sampling schemes} or {\it designs}  available.
\end{itemize}
\end{frame} 

\begin{frame}
\frametitle{Study population in a case-control study?}

\pause
{Ideally:} The study population comprises subjects who
 \underline{\textit{would be}} included as
	cases, \underline{\textit {if they got}} the outcome in
	the study
\pause
%	= {\it Source population} of cases. 
	\bigskip
	\bi
\item {\it Cohort-based studies}: {\bf cohort} or
 {\bf closed} population of well-identified subjects
  under intensive follow-up for outcomes (\textit{e.g.} biobank cohorts).
% and total mortality.
\pause
\medskip
\item {\it Register-based studies}: {\bf open} or
 {\bf dynamic} population in a region covered by a disease register. 
 % (\textit{e.g.} the melanoma study).	
  \pause
  \medskip
\item {\it Hospital-based} studies: dynamic {\bf catchment} population of cases 
--  may be hard to identify ({\it e.g.} hospitals in US). 
\end{itemize}

\pause
In general, the role of control subjects is to represent the
distribution of person-times by exposure variables 
in the underlying population
from which the cases emerge.

\end{frame} 
%-----------------




%----------------------------------------------------

\begin{frame}
\frametitle{\large Sampling of controls -- alternative frames}

\medskip
Illustrated in a simple longitudinal setting: \\
Follow-up of a cohort over a fixed risk period \& no censoring.

% \pause
\begin{center}
{\normalsize
\setlength{\unitlength}{0.08cm}
\begin{picture}(135,50)
\thicklines
\linethickness{0.3mm}

%mittlere Box

%% \put(40,45){\line(6,-1){55}}
%% \put(40,5){\framebox(55,40){}}
 \put(42.5,45){\line(6,-1){60}}
 \put(42.5,5){\framebox(60,40){}}
%Achsenbeschriftung
\put(50,-1.5){\makebox(25,5){\small Time ($t$)}}
\put(30,-1){\makebox(25,5){\small Start}}
\put(90,-1){\makebox(25,5){\small End}}
\put(75,2){\vector(1,0){20}}

% \pause
%linker Text
\put(0,5){\makebox(40,40){\parbox{7cm}{\begin{center}
						   {\small \bf (B)  Initially at risk}\\
						%	(disease-free)\\ (C)\\
							{\small $(N)$}
							\end{center} }}}
% \pause
\put(45,15){\makebox(55,10){\parbox{11cm}{ \begin{center}
						{\small \bf (C) Currently at risk $(N_t)$}
							\end{center} }}}
\put(72.5,25){\vector(0,1){15}}
\put(72.5,15){\vector(0,-1){9}}							
%rechter Text
% \pause
\put(100,30){\makebox(40,15){\parbox{7cm}{\begin{center}
					{\small		New cases\\of disease\\ $(D)$}
							\end{center} }}}
% \pause							
\put(100,5){\makebox(40,20){\shortstack{\strut\small
       {\bf (A) Still at risk} \\ \strut
						%	(disease-free)\\ \strut 	(A)\\
						 \strut $(N - D)$ }}}


\end{picture}
}
\end{center}

\medskip
{\small Rodrigues, L. \& Kirkwood, B.R. (1990).
 Case-control designs of common diseases \dots 
 {\it Int J Epidemiol} {\bf  19}: 205-13. }

\end{frame} 


%---------------------------------

\begin{frame}
\frametitle{\large Sampling schemes or designs for controls}

\pause
\medskip
{\bf (A) Exclusive} or {\bf traditional, ``case-noncase'' sampling}
           \bi
           \item
           Controls chosen from those $N-D$ 
           subjects still at risk (healthy) \\
    \underline{\it at the end}
				    of the risk period (follow-up). 
	     \ei 
	     \pause  
{\bf (B) Inclusive sampling} or {\bf case-cohort design} (CC)
	     \bi
\item The control group -- {\it subcohort} -- 
           is a random sample of the cohort $(N)$  
%    free from the disease 
\underline{\it at start}. % of the follow-up.
    \ei
    \pause
{\bf (C) Concurrent sampling} or {\bf density sampling}  
          \bi 
         %  \item[--] Also called % {\it density} sampling, 
           % {\it risk-set} s., 
           % {\it time-matched} s., or
         %  {\it concurrent} sampling.
			  %        or {\it dynamic} s.
           \item  
           Controls drawn 
     %      at different times $t$
            \underline{\it during the follow-up}
            % from those
           % $N_t$ 
           % at risk.
           \pause 
           \medskip
           \item {\bf Risk-set} or {\bf time-matched sampling:} \\
         A set of controls is sampled from
         the {\it risk set} 
          %  (alive \& yet uncensored)
	%			  for the outcome 
			  \underline{\it at each time $t$ of diagnosis}
                           of a new case 
         \pause                  
         -- a.k.a. \textbf{nested case-control design} (NCC)                
	%			     i.e. have not yet contracted the disease by then.
	 %\medskip
   %       \item[$\bullet$]  applicable both in cohorts 
   %        and in dynamic populations.
   %        \pause\medskip
   %       \item[$\bullet$] in dynamic populations, 
   %       time-matching not always exact.
	     \ei
    
\end{frame}

\begin{frame}
\frametitle{Nested case-control -- two meanings}

\begin{itemize}
\item
In some epidemiologic books, the term ``nested case-control study''
 (NCC) covers jointly 
all variants of sampling: {\bf (A)}, {\bf (B)}, and {\bf (C)}, from
 a cohort.
 \pause
\bi
\item[ ]
 Rothman \textit{et al.} (2008): 
  \textit{Modern Epidemology, 3rd Ed.}  
\item[ ]  
  Dos Santos Silva (1999): \textit{Cancer Epidemiology.} Ch 8-9 
\ei  
\bigskip
\pause
\item
In biostatistical texts NCC typically refers only to
the variant of concurrent or density sampling {\bf (C)}, in which 
{\it risk-set} or {\it time-matched} sampling is employed.
 \pause
 \medskip 
 \bi
\item[ ]
Borgan \& Samuelsen (2003) in \textit{Norsk Epidemiologi} 
\item[ ]
 Langholz (2005) in \textit{Encyclopedia of Biostatistics.}
 \ei

\pause
\medskip
 \item
We shall follow the biostatisticians!
\end{itemize}

\end{frame}

\begin{frame}
\frametitle{NCC: Risk-set sampling with staggered entry}
\ \\

Sampling frame to select controls for a given case: \\
Members ($\times$) of the {\bf risk set} at $t_k$, {\it i.e.}
the population at risk at the time of diagnosis $t_k$ of case $k$.

% \begin{frame}\frametitle{Choice of controls (II)}
\pause
\bigskip

\begin{center}
{\small
\begin{picture}(150,85)(0,0)
\setlength{\unitlength}{0.023cm}
\put( 75,140){\hfail{110}}
\put(180,140) {\makebox(0,0){$\times$}}
\put(-80,140){\makebox(0,0)[l]{Case}}
\put( 75,120){\hcens{250}}
\put(180,120) {\makebox(0,0){$\times$}}
\put(-80,120){\makebox(0,0)[l]{Healthy until end}}
\put( 75,100){\hcens{90}}
\put(-80,100){\makebox(0,0)[l]{Early censoring}}
% \put(200, 80){\makebox(0,0)[l]{case's dgn}}
\put(120, 80){\hcens{150}}
\put(180, 80) {\makebox(0,0){$\times$}}
\put(-80, 80){\makebox(0,0)[l]{Late entry}}
\put(210, 60){\hcens{115}}
% \put(180, 60) {\makebox(0,0){$\times$}}
\put(-80, 60){\makebox(0,0)[l]{Too late entry}}
\put(100,40){\hfail{50}}
% \put(180,40) {\makebox(0,0){$\times$}}
\put(-80,40){\makebox(0,0)[l]{Early case}}
\put( 100,20){\hfail{110}}
\put(180,20) {\makebox(0,0){$\times$}}
\put(-80,20){\makebox(0,0)[l]{Later case}}
\put(180, 00){\line(0,1){140}}
\put(190, 00){\line(0,1){140}}
\put( 75, 00){\line(1,0){250}}
\put( 75, -10){\makebox(0,0)[l]{Start}}
\put(325, -10){\makebox(0,0)[r]{End}}
\put(200, -25){\makebox(0,0)[c]{Study period}}
\end{picture}
} % end small
\end{center}

\medskip
{\bf Sampled risk set} contains the case
and the control subjects randomly sampled from the 
non-cases in the risk set at $t_k$. 

\end{frame} 



\begin{frame}
\frametitle{\large Use of different sampling schemes}

\pause
\bigskip
 
{\textbf{(A)} Exclusive sampling, or ``textbook'' case-control design}
\bi
% \item traditional view of the case-control study,
% \item effect measure available\\
% $ \ \ \ \ $ {\bf disease odds ratio}, which is the same as \\
% $ \ \ \ \ $ {\bf exposure odds ratio} EOR 
% \item practical in some ,
% \item with prevalent cases and non-cases \\ 
\item Almost exclusively(!) used in studies of epidemics. 
\item (Studies on birth defects with \textit{prevalent} cases.)
\ei

\pause
\textbf{(B)} Inclusive sampling or case-cohort design 
\bi
\item Good esp. for multiple outcomes, if
 measurements of  \\ risk factors from stored material
 remain stable.
\ei

\pause
\textbf{(C)} Concurrent or density sampling  
(without or with time-matching) 
 \pause
\bi
\item The only logical design in an open population.
\medskip
\pause
\item Most popular in chronic diseases (Knol {\it et al.} 2008).
% \pause 
%\medskip
%\item[$\Rightarrow$] Should receive priority in 
% introductory courses and texts!
\ei

\pause
Designs {\bf (B)} and {\bf (C)} allow valid 
estimation of hazard ratios $\rho$ without any
	``rare disease'' assumption.
	
\end{frame} 

\begin{frame}
\frametitle{\large Case-control studies: Textbooks vs. real life}

\pause
\bi
\item
Many epi texts %% in epidemiology teach outdated dogma and myths
%% about outcome-dependent designs.
%% \pause \medskip \item They tend to 
focus on the traditional design: \\
 \textbf{exclusive sampling} of controls, ignoring other designs.
%% from the non-diseased,
\pause
\medskip
\item
Claim: ``\textit{Odds ratio is the only estimable parameter.}'' 
 \pause
\medskip
\item
 Yet, over 60\% of published case-control 
 studies apply \textbf{concurrent sampling}
 or {\bf density sampling} of controls from an \textbf{open} or
 \textbf{dynamic} population. 
 \pause
 \medskip
 \item
 Thus, the parameter most often estimated % in case control studies
 is the \\ {\bf hazard ratio} (HR) or {\bf rate ratio} $\rho$.
\pause
 \medskip
 \item
  Still,  90\% of authors really estimating HR, 
  reported as having estimated an OR ({\it e.g.} Simen Kapeu {\it et al.} 2009)
\ei
\pause

{\small Knol {\it et al.} (2008). What do case-control studies estimate? \\% Survey of methods and assumptions in published case-control research.
{\it Am J Epidemiol} {\bf 168}: 1073-81. }
% \medskip
% see also Langholz (2010). Case-control studies = odds ratios: Blame the retrospective model. {\it Epidemiology} {\bf 21}: 10-12. } 
\end{frame}

\begin{comment}

\begin{frame}
\frametitle{\large Case-control studies -- true essence}

\pause
\medskip
\begin{itemize}
% \item
% ``In the 1970s the concept was advanced that 
% case-control studies should be  viewed as 
% \textit{efficient sampling schemes} of the disease experience of the 
% underlying  \textit{cohorts}'' \\
% \item[ ]
% {\small (MacMahon \& Trichopoulos 1996.
% \textit{Epidemiology})} % : Principles \& Methods})}.
\item
``the modern conception of the case-control study as involving  \textbf{sampling}, on the basis of {\it outcome}, from an ongoing real or imagined  \textit{cohort study}.''\\
{\small (Breslow 2005, in % Case-Control Studies. In
 \textit{Handbook of Epidemiology}}) % , 2005)}
% CC study as outcome-dependent sampling from real or fictitious cohort
% - Cohort definition
% - Paradigm: nested CC study
% - Incidence density sampling
\pause
\medskip
\item[$\Rightarrow$] 
 ``Case-control studies'' not so distinct from  
``cohort studies'' -- rather a subspecies of them?
\pause\bigskip
\item
``The case-control study \dots
 {\bf data missing}  by design'' 
{\small (Wacholder. \textit{Epidemiology} 1996; 7:144-50.)}
\pause
\medskip
\item[$\Rightarrow$] General approaches for analysis: \\
(i) full likelihood methods for incomplete data, \\
(ii) partial or pseudo-likelihood methods, with weighing \& calibration , 
% both were nicely covered in Nathalie's talk!
\end{itemize}
 
\end{frame}



\begin{frame}
\frametitle{\large Only odds ratio & and it $\approx$ ``relative risk''?}

\pause
\bi
\item Textbook dogma: 
  ``Case-control studies: Only odds ratio is estimable, and it
  approximates relative risk with rare diseases.''
\pause
\medskip
\item
What is this ubiquitous ``relative risk''? 
\pause
\bi
\item[--] rate ratio or hazard ratio (HR)? 
\medskip
\item[--] cumulative incidence ratio or risk ratio (RR)?
\medskip
\item[--] prevalence ratio (PR)?
\ei   
\pause 
\medskip
\item
{\bf NB.} 
Sometimes hazard ratio 
= prevalence odds ratio.
 \\
{\small (see e.g. Keiding {\it JRSSA}, 1991; Alho, J. {\it Biometrics}, 1992; and Kari Auranen's talk at NBBC11)}.    
\ei  
% \medskip
\end{frame}

\end{comment}

\begin{frame}
\frametitle{\large Exposure odds ratio -- estimate of what?}

\bi
\item
Crude summary of case-control data
% (ignoring possible matching or stratification in design)

\medskip
\begin{center}
\begin{tabular}{l c c c}
\hline
               & exposed & unexposed & total\\
\hline
cases          & $D_1$     & $D_0$       & $D$\\
controls       & $C_1$     & $C_0$       & $C$\\
% (total)      & $(n_1)$ & $(n_0)$    & $(N)$\\
\hline
\end{tabular}
\end{center}

\pause
\bigskip
\item  Depending on study base \& sampling strategy,   
the {\bf exposure odds ratio} % (EOR) 
$$
  \text{EOR} 
    = \frac{D_1/ D_0}{C_1/C_0}
    = \frac{\text{cases: exposed / unexposed}}
           {\text{controls: exposed / unexposed}}
$$
\pause
is a consistent estimator of  
\bi
{\normalsize 
\item[ ]
 (a) hazard ratio, (b) risk ratio,
(c) risk odds ratio, \pause \medskip
\item[ ] (d) prevalence ratio, or (e) prevalence
odds ratio 
}
\ei
\pause
\item {\bf NB.} In case-cohort studies with variable follow-up times
$C_1/C_0$ is substituted by $\widehat Y_{1}/\widehat Y_{0}$, 
from estimated p-years.
      
 \ei  
\end{frame}



\begin{frame}
\frametitle{Precision and efficiency}

With exclusive {\bf (A)} or concurrent {\bf (C)} sampling 
of controls (unmatched), the estimated variance of log(EOR) is \pause
\bes
 \widehat{\text{var}}\text{[log(EOR)]} 
  & = & \frac{1}{D_1} + \frac{1}{D_0} 
      + \frac{1}{C_1} + \frac{1}{C_0} \\
\pause
 & = & \text{cohort variance} + \text{sampling variance}
\ees 
\pause
\begin{itemize}
\item Depends basically on the numbers of cases, with  
$\geq 4$ controls per case.
\pause
\medskip
\item Is not much bigger than $1/D_1 + 1/D_0$ = variance in a full cohort study with   
 same numbers of cases.
 \pause
 \medskip
\item[$\Rightarrow$] Usually $< 5$ controls per case is enough.
 \pause\medskip
\item
[$\Rightarrow$] {\it These designs are very cost-efficient!}
\end{itemize}
% CC design: Same principle, more complicated formula.
\end{frame} 


\begin{frame}
\frametitle{Estimation in concurrent or density sampling}

\pause
\bi
\item Assume a simple situation: 
Prevalence of exposure % $q(t)$  
in the study population stable over time. % $q$ % over time.
\pause
\medskip
\item[$\Rightarrow$] The exposure odds $C_1/C_0$ among controls \\
 = a consistent estimator of exposure odds $Y_1/Y_0$ of person-times.
%% even if controls sampled at any time from population at risk.
\pause
\medskip
\item
Therefore, the crude EOR = $(D_1/D_0)/(C_1/C_0)$ \\ 
= a consistent estimator of
 {hazard ratio}  $\rho = \lambda_1/\lambda_0$.
\pause
\medskip
\item
Variance of log(EOR) estimated as above.
 \pause
\medskip
\item
Yet, stability of exposure distribution may be unrealistic,
especially in a closed study population or cohort. 

\pause
\medskip
\item
Solution: \textbf{Time-matched} sampling of controls from \\
{\bf risk sets}, {\it i.e.} NCC, \& 
matched EOR to estimate HR.
\ei
 Prentice \& Breslow (1978), Greenland \& Thomas (1982).
\end{frame} 

\begin{comment}
\begin{frame}
\frametitle{Case-cohort design: Estimation of HR}

\medskip
Summary of case-cohort data with variable follow-up times.

\begin{center}
\begin{tabular}{ l c c c}
\hline
               & Exposed & Unexp. & Total\\
\hline
Numbers of cases\\
${}\quad {} $ outside subcohort & $D_{10}$ & $D_{00}$  & $D_{+0}$\\
 ${}\quad {} $  in subcohort  & $D_{11}$ & $D_{01}$  & $D_{+1}$\\
 \hline
 Total                         & $D_1$  & $D_0 $      & $D$  \\
\hline
Person-times\\
${}\quad {}$ cases outside subcohort & $Y_{10D}$ & $Y_{00D}$ & $Y_{+0D}$   \\
${}\quad {} $  cases in subcohort & $Y_{11D}$ & $Y_{01D}$ & $Y_{+1D}$   \\
${}\quad {} $  non-cases in subcohort & $Y_{11B}$  & $Y_{01B}$ & $Y_{+1B}$   \\
${}\quad {} $  (non-cases outside) & ($Y_{10B}$)  & ($Y_{00B}$) & $Y_{+0B}$  \\
\hline
 Total        & $(Y_1)$ & $(Y_0)$    & $ Y $\\
 \hline
\end{tabular}
\end{center}

\medskip
Person-times in parentheses not directly observed
\end{frame}

\begin{frame}
\frametitle{Case-cohort design: Estimation of HR (cont'd)}

\bi
\item
Total person-times in the subcohort\\
-- among exposed: $Y_{11} = Y_{11D} + Y_{11B}$, \\
-- among unexposed $Y_{01} = Y_{01D} + Y_{01B}$.
\pause
\medskip
\item
Exposure odds $Y_{11}/Y_{01}$ in the subcohort 
is consistently estimating odds $Y_1/Y_0$ in the whole cohort.
\pause
\medskip
\item
Consistent estimator of the {\it hazard ratio}
 $\rho = \lambda_1/\lambda_0$ is 
%% the following kind of exposure odds ratio
$$ \text{EOR} = \frac{D_1/D_0}{Y_{11}/Y_{01}} $$
\pause
\item
(Generalization: {\it Prentice's pseudo-likelihood}.)
\pause
 \medskip
\item
Estimated variance of log(EOR): 
$$ V = 1/D_1 + 1/D_0 + \text{ additional terms}, $$ 
the additional terms depending on the precision of 
sampled person-times $Y_{11}$ and $Y_{01}$.


\ei
\end{frame}

%------------------------

\begin{frame}
\frametitle{Case-cohort design: Estimation of HR (cont'd)}

\bi
\item
Information in person-times
of {\it cases outside subcohort} can be utilized, too.
\pause 
\medskip
\item
$f$ = {\bf sampling fraction} of non-cases to subcohort.
\pause
\medskip
\item[$\to$]
Expanded exposure odds in the whole case-cohort data 
$$ \frac{Y_{10D} + Y_{11D} + Y_{11B}/f}
        {Y_{00D} + Y_{01D} + Y_{01B}/f} 
   = \frac{\widehat Y_1}{\widehat Y_0} $$ 
is consistent for $Y_1/Y_0$. 
%% In fact, $\widehat Y_1$ and $\widehat Y_0$ are valid estimates
%% of person-times in the whole cohort.
\pause
 $\to$  Crude absolute rates \\ $\lambda_1, \lambda_0$
 also estimable by $\widehat\lambda_j = D_j/{\widehat Y_j}$, $j=1,0$.
\pause
\medskip
\item
Hazard ratio now efficiently estimated
\begin{center}
$ \widehat\rho = {(D_1/D_0)}/{(\widehat Y_{1}/\widehat Y_{0})} $
\end{center}
\medskip
\pause
\item
(Generalization: {\it Weighted partial likelihood}.)
\pause
\medskip
\item
Estimated variance of log($\widehat\rho$): 
$$ V = 1/D_1 + 1/D_0 + \text{ smaller add'l term}. $$
\ei
\end{frame}

\begin{frame}
\frametitle{Estimation in open population}

\bi
\item
Concurrent sampling -- the only logical sampling scheme!
\medskip
\item
The population at risk ``at the end'', namely, does not necessarily
contain anybody from the population at risk ``in the beginning''
of the period of case collection. 
(\textit{e.g.} infant mortality study during one calendar year)
\medskip
\item
Assuming constant exposure prevalence $q(t) = q$ over time, the
hazard ratio $\rho = \lambda_1/\lambda_0$ is again 
consistently estimated by   EOR = $(D_1/D_0)/(C_1/C_0)$.
\medskip
\item
Stability of exposure distribution may actually be more often realistic
in a dynamic population with short time period compared to
a lengthy follow-up of a cohort.
\medskip
\item
Nevertheless, time-matched sampling may often be employed leading to
matched EOR for estimating $\rho$.

\ei
\end{frame}



\end{comment}

% ----------------------------------------------------------------------

\begin{frame}
\frametitle{Matching in case-control studies}
\pause
% Matching in sampling designs {\it A} and {\it B}:
\begin{itemize}
\item {\bf Stratified sampling} of controls,  {\it e.g.} from 
the \\ same region, sex, and age group as a given case 
%  (with or without \textit{time-matching} on risk set!).  
\pause\medskip
   \item {\bf Frequency matching} or {\bf group matching:} \\
  For cases in a specific stratum 
  ({\it e.g.} same sex and 5-year age-group), % in the relevant risk set
a set of controls from a similar subgroup. % (see melanoma study). 
\\ { } \\
		\pause 
		  \medskip		
\item {\bf Individual matching} (1:1 or 1:m matching):\\
		For each case, choose 1 or more (rarely $>5$) closely similar
		controls (\textit{e.g.} same sex, age within $\pm$1 year. 
%%		same neighbourhood, \textit{etc.}).
		\pause
	\medskip
\item
{\bf NCC}: Sampling from risk-sets 
implies time-matching at least.
Additional matching for other factors possible. 
\pause
	\medskip
\item
{\bf CC}: Subcohort selection involves no matching with cases.	
		\end{itemize}
% \item Creates similar distributions of matching factors
%      for ctrls \& cases $\Rightarrow$ balanced comparisons 

\end{frame}
%      \pause\medskip
\begin{frame}
\frametitle{Virtues of matching}
\pause
\begin{itemize}
 \item Increases {\it efficiency}, 
 if the matching factors are both
 \bi
 {\normalsize
 \item[ ]
   (i) strong
	 \textit{risk factors} of the disease, and 
\item[ ](ii) 
	 \textit{correlated} with the main exposure. 
}
\ei	 
\pause
	-- Major reason for matching.  
	\pause   
\medskip   
\item \textit{Confounding} due to poorly
quantified factors (sibship, 
   neighbourhood, {\it etc.}) may be removed by close matching  
  -- only if properly analyzed. 
	\pause\medskip
\item Biobank studies: Matching
  for storage time, % (depening on the date of baseline measurements), 
  freeze-thaw cycle \& analytic batch improves \textbf{comparability of measurements}
	from frozen specimens 
\bi
\item[$\to$] 
{\normalsize
  Match on the time of baseline measurements within the case's risk set.
  }
\ei   
\end{itemize}

\end{frame} 




\begin{frame}
\frametitle{Warnings for overmatching}

\pause

Matching a case with a control subject is a different issue than matching 
an unexposed subject to an exposed one in a cohort study  --  much trickier! 
\pause
\begin{itemize}
\item Matching on an {\it intermediate} variable between
	exposure  and outcome. \\  	
	\pause\medskip
	$\Rightarrow$ {\it Bias}! % in effect estimation
\pause\medskip
\item Matching on a {\it surrogate} or 
{\it correlate} of exposure, which is not a
	true risk factor. \\ % of outcome 		
%	\pause\medskip
	$\Rightarrow$ {\it Loss of efficiency}. 
	\pause\medskip
% From the latter arises the principle of
\item[$\rightarrow$] {\bf Counter-matching:} \pause
 Choose a control which \\
    \underline{is not
	similar} to the case w.r.t a correlate of exposure. 
	\\	
	\medskip
	$\Rightarrow$ Increases efficiency!
	
	\medskip
	$\bullet \ \ $ Requires appropriate weighting in the analysis.
\end{itemize}
\end{frame} 

\begin{frame}
\frametitle{Sampling matched controls for NCC using R}

\pause
\bi
\item
Suppose key follow-up items are recorded
for all subjects in a cohort, in which a NCC study
is planned.
\pause
\medskip
\item
Function {\tt ccwc()} in package {\tt Epi} can be used for risk-set sampling
of controls. -- Arguments:
\ei
\pause
\begin{tabular}{r l}
{\tt entry} : & Time of entry to follow-up\\  
{\tt exit} : & Time of exit from follow-up \\ 
{\tt fail} : & Status on exit (1 for case, 0 for censored) \\ 
{\tt origin} : & Origin of analysis time scale (e.g. time of birth)\\ 
{\tt controls} : & Number of controls to be selected for each case \\ 
{\tt match} : & List of matching factors\\
{\tt data} : & Cohort data frame containing input variables 
\end{tabular}
\pause 
\bi
\item Creates a data frame for a NCC study,
containing the desired number of matched controls
for each case.
\ei

\end{frame}

\begin{frame}
\frametitle{Analysis of matched studies}

\begin{itemize}
\item Close matching induces a new 
parameter for each matched case-control set or stratum.
\pause
\begin{itemize}
{\normalsize 
\item[$\Rightarrow$] % Methods that ignore matching, like 
 {\bf unconditional logistic regression} breaks down.
}
\end{itemize}
\pause 
\item Matching on well-defined variables (like age, sex) \\ -- 
%% broader strata may be formed {\em post hoc}, and 
include these factors as covariates.
\pause
\medskip
\item Matching  on ``soft'' variables (like sibship) %% cannot be ignored, but this 
can be dealt with \\
{\bf conditional logistic regression}.
\pause
\medskip
\item Same method in matched designs {\bf (A)}, exclusive, and {\bf (C)}, concurrent, but interpretation of % regression coefficients 
$\beta_j$s differs:
\medskip
\begin{itemize}
{\normalsize
\item[]\textbf{(A)} $\beta_j$ = log of risk odds ratio (ROR),
\item[]\textbf{(C)} $\beta_j$ = log of hazard ratio (HR).
}
 \end{itemize} 
\end{itemize}

\end{frame}



\begin{frame}
  \frametitle{Full cohort design: Follow-up \& risk sets}
  
 \pause 
  Each member of the cohort provides exposure data
   for all cases, as long as this member is {at risk}, 
   {\it i.e.} (i) alive, (ii) not censored \& (iii) free from outcome.
  
  \ \\
\pause
  \begin{center}
  \begin{picture}(200,100)(0,0)

    \thicklines

    %Axes
    \put(0,0){\vector(1,0){200}}
    \put(200,0){Time}
    \put(0,0){\vector(0,1){100}}
    \put(0,105){Subjects}
    \put(0,10){\line(-1,0){3}}
    \put(0,20){\line(-1,0){3}}
    \put(0,30){\line(-1,0){3}}
    \put(0,40){\line(-1,0){3}}
    \put(0,50){\line(-1,0){3}}
    \put(0,60){\line(-1,0){3}}
    \put(0,70){\line(-1,0){3}}
    \put(0,80){\line(-1,0){3}}
    \put(0,90){\line(-1,0){3}}
    \pause    

    %first time step step
    \thinlines
    \put(0,10){\line(1,0){50}}
    \put(0,20){\line(1,0){50}}
    \put(0,30){\line(1,0){50}}
    \put(0,40){\line(1,0){50}}
    \put(0,50){\line(1,0){50}}
    \put(0,60){\line(1,0){30}} \put(30,57.5){\line(0,1){5}}
    \put(0,70){\line(1,0){50}}
    \put(0,80){\line(1,0){50}}
    \put(0,90){\line(1,0){50}}
    % Legend for controls
    \put(30,-10){\line(-1,0){5}}
    \put(30,-12.5){\line(0,1){5}}
    \put(35,-10){\makebox(0,0)[l]{Censored}}
    \pause

    %second time step
    \put(50,10){\line(1,0){50}}
    \put(50,20){\line(1,0){50}}
    \put(50,30){\line(1,0){30}} \put(80,27.5){\line(0,1){5}}
    \put(50,40){\line(1,0){50}}
    \put(50,50){\line(1,0){50}} 
    %Censored at time 80
    \put(50,70){\line(1,0){50}}
    \put(50,80){\line(1,0){50}}
    \put(50,90){\line(1,0){50}}
    \color{red} 
    \put(100,50){\circle*{5}} %case

    % Legend for case
    \put(100,-10){\circle*{5}}
    \color{black}
    \put(105,-10){\makebox(0,0)[l]{Case}}
    \pause
    
    %sampling frame
    \color{darkgreen} 
    \put(100,50){\oval(8,100)} 
    \pause
    
    %sampled controls
    \put(100,90){\circle{5}}
    \put(100,80){\circle{5}}
    \put(100,70){\circle{5}}
    \put(100,40){\circle{5}}
  %  \put(100,30){\circle{5}}
    \put(100,20){\circle{5}}
    \put(100,10){\circle{5}}
    % Legend for controls
    \put(150,-10){\circle{5}}
    \color{black}
    \put(155,-10){\makebox(0,0)[l]{At risk}}
    \pause

    %third time step
    \put(100,10){\line(1,0){50}}
    \put(100,20){\line(1,0){10}} \put(110,17.5){\line(0,1){5}}
    %Censored at time 130
    \put(100,40){\line(1,0){50}}
    %Case at time 150
    %\put(100,60){\line(1,0){50}}
    \put(100,70){\line(1,0){30}} \put(130,67.5){\line(0,1){5}}
    \put(100,80){\line(1,0){50}}
    \put(100,90){\line(1,0){50}}
    \color{red} 
    \put(150,10){\circle*{5}} %second case
    \pause

    %Second sampling frame
    \color{darkgreen} 
    \put(150,50){\oval(8,100)} 
    \pause

    %sampled controls
    \put(150,90){\circle{5}}
    \put(150,80){\circle{5}}
    \put(150,40){\circle{5}}
  %  \color{black}
    \put(105,105){Risk sets}
  \end{picture}
  \end{center}
\pause
\medskip
Times of new cases define the {\bf risk-sets}.

\end{frame}

 \begin{frame}
  \frametitle{Nested case-control (NCC) design}

\pause
Whenever a new case occurs, a set of controls 
 (here 2/case)  are \\ sampled
from its {risk set}. % = those at risk at the time of diagnosis.

\ \\
\pause
  \begin{center}
  \begin{picture}(200,100)(0,0)

    \thicklines

    %Axes
    \put(0,0){\vector(1,0){200}}
    \put(200,0){Time}
    \put(0,0){\vector(0,1){100}}
    \put(0,100){Subjects}
    \put(0,10){\line(-1,0){3}}
    \put(0,20){\line(-1,0){3}}
    \put(0,30){\line(-1,0){3}}
    \put(0,40){\line(-1,0){3}}
    \put(0,50){\line(-1,0){3}}
    \put(0,60){\line(-1,0){3}}
    \put(0,70){\line(-1,0){3}}
    \put(0,80){\line(-1,0){3}}
    \put(0,90){\line(-1,0){3}}
    \pause    

    %first time step step
    \thinlines
    \put(0,10){\line(1,0){50}}
    \put(0,20){\line(1,0){50}}
    \put(0,30){\line(1,0){50}}
    \put(0,40){\line(1,0){50}}
    \put(0,50){\line(1,0){50}}
    \put(0,60){\line(1,0){30}} \put(30,57.5){\line(0,1){5}}
    \put(0,70){\line(1,0){50}}
    \put(0,80){\line(1,0){50}}
    \put(0,90){\line(1,0){50}}
    % Legend for controls
    \put(30,-10){\line(-1,0){5}}
    \put(30,-12.5){\line(0,1){5}}
    \put(35,-10){\makebox(0,0)[l]{Censored}}
    \pause

    %second time step
    \put(50,10){\line(1,0){50}}
    \put(50,20){\line(1,0){50}}
    \put(50,30){\line(1,0){30}} \put(80,27.5){\line(0,1){5}}
    \put(50,40){\line(1,0){50}}
    \put(50,50){\line(1,0){50}} 
    %Censored at time 80
    \put(50,70){\line(1,0){50}}
    \put(50,80){\line(1,0){50}}
    \put(50,90){\line(1,0){50}}
    \color{red} 
    \put(100,50){\circle*{5}} %case

    % Legend for case
    \put(100,-10){\circle*{5}}
    \color{black}
    \put(105,-10){\makebox(0,0)[l]{Case}}
    \pause
    \color{darkgreen}  
     \put(105,105){Risk sets}
    %sampling frame
    \color{darkgreen} 
    \put(100,50){\oval(8,100)} 
    \pause
    
    %sampled controls
    \put(100,90){\circle{5}}
    \put(100,20){\circle{5}}
    % Legend for controls
    \put(150,-10){\circle{5}}
    \color{black}
    \put(155,-10){\makebox(0,0)[l]{Control}}
    \pause

    %third time step
    \put(100,10){\line(1,0){50}}
    \put(100,20){\line(1,0){10}} \put(110,17.5){\line(0,1){5}}
    %Censored at time 130
    \put(100,40){\line(1,0){50}}
    %Case at time 150
    %\put(100,60){\line(1,0){50}}
    \put(100,70){\line(1,0){30}} \put(130,67.5){\line(0,1){5}}
    \put(100,80){\line(1,0){50}}
    \put(100,90){\line(1,0){50}}
    \color{red} 
    \put(150,10){\circle*{5}} %second case
    \pause

    %Second sampling frame
    \color{darkgreen} 
    \put(150,50){\oval(8,100)} 
    \pause

    %sampled controls
    \put(150,80){\circle{5}}
    \put(150,40){\circle{5}}
    \color{black}
    
  \end{picture}
  \end{center}
  
\ \\  
\pause
{\bf NB.} A control once selected for some case can be
selected as a control for another case, and 
can later on become a case, too.
\end{frame}

%------------------------



% \section{Case-cohort design}

\begin{frame}
  \frametitle{Case-cohort (CC) design}
  
  \pause
 {\bf Subcohort}: Sample of the whole cohort randomly selected
 at the outset. \\ --  Serves as a reference group for all cases. 
 
 \pause
 \bigskip
 
 \medskip
 
  \begin{center}
  \begin{picture}(200,100)(0,0)

    \thicklines
    %\graphpaper(0,0)(200,100)

    %Axes
    \put(0,0){\vector(1,0){200}}
    \put(200,0){Time}
    \put(0,0){\vector(0,1){100}}
    \put(0,100){Subjects}
    \put(0,10){\line(-1,0){3}}
    \put(0,20){\line(-1,0){3}}
    \put(0,30){\line(-1,0){3}}
    \put(0,40){\line(-1,0){3}}
    \put(0,50){\line(-1,0){3}}
    \put(0,60){\line(-1,0){3}} 
    \put(0,70){\line(-1,0){3}}
    \put(0,80){\line(-1,0){3}}
    \put(0,90){\line(-1,0){3}}
 %   \put(0,100){\line(-1,0){3}}
    % Legend
    \put(30,-10){\line(-1,0){5}}
    \put(30,-12.5){\line(0,1){5}}
    \put(35,-10){\makebox(0,0)[l]{Censored}}

    \color{red}
    \put(100,-10){\circle*{5}}
    \color{black}
    \put(105,-10){\makebox(0,0)[l]{Case}}

    \color{darkgreen}
    \put(150,-10){\circle{5}}
    \color{black}
    \put(155,-10){\makebox(0,0)[l]{Control}}

    \pause    

    %Subcohort
    \color{darkgreen} 
    \put(-3,80){\oval(8,30)[l]} 
    \put(-11,80){\makebox(0,0)[r]{Subcohort}}
    \color{black}
    \pause
    
    %first time step
    \thinlines
    \put(0,10){\line(1,0){50}}
    \put(0,20){\line(1,0){50}}
    \put(0,30){\line(1,0){50}}
    \put(0,40){\line(1,0){50}}
    \put(0,50){\line(1,0){50}}
    \put(0,60){\line(1,0){30}} \put(30,57.5){\line(0,1){5}}
    \color{darkgreen}
    \put(0,70){\line(1,0){50}}
    \put(0,80){\line(1,0){50}}
    \put(0,90){\line(1,0){50}}
 %   \put(0,100){\line(1,0){50}}
    \color{black}
    \pause

    %second time step
    \put(50,10){\line(1,0){50}}
    \put(50,20){\line(1,0){50}}
    \put(50,30){\line(1,0){30}} \put(80,27.5){\line(0,1){5}}
    \put(50,40){\line(1,0){50}}
    \put(50,50){\line(1,0){50}} 
    %Censored at time 80
    \color{darkgreen}
    \put(50,70){\line(1,0){50}}
    \put(50,80){\line(1,0){50}}
    \put(50,90){\line(1,0){50}}
    \color{red} 
    \put(100,50){\circle*{5}} %case
    \pause

   
    %Risk set
    \color{darkgreen} 
    \put(100,80){\oval(8,30)} 
    \put(100,50){\oval(8,15)}  %%%%%%%%%%%%%%%
     \put(90,105){Sampled risk sets}
 %%%%   \pause

    %sampled controls
    \put(100,90){\circle{5}}
    \put(100,80){\circle{5}}
    
    \color{black}
    \put(100,70){\circle{5}}
    \pause

    
    %third time step
    \put(100,10){\line(1,0){50}}
    \put(100,20){\line(1,0){10}} \put(110,17.5){\line(0,1){5}}
    %Censored at time 130
    \put(100,40){\line(1,0){50}}
    %Case at time 150
    %\put(100,60){\line(1,0){50}}
    \color{darkgreen}
    \put(100,70){\line(1,0){30}} \put(130,67.5){\line(0,1){5}}
    \put(100,80){\line(1,0){50}}
    \put(100,90){\line(1,0){50}}
    \color{darkgreen}
    \color{red} 
    \put(150,10){\circle*{5}} %second case
    \pause

    %Second risk set
    \color{darkgreen} 
    \put(150,80){\oval(8,30)} 
    \put(150,10){\oval(8,15)} 
 %%%   \pause

    %sampled controls
    \color{darkgreen}
    \put(150,90){\circle{5}}
    \put(150,80){\circle{5}}
    \color{black}
    
  \end{picture}
  \end{center}
 
 \ \\ 
 \pause
{\bf NB.} A subcohort member can become a case, too.
\end{frame}

%------------------------




\begin{comment}

\begin{frame}
\frametitle{Sampling from the risk sets}

Follow-up of cohort with $n=11$, four cases ($\bullet$). Risk-sets comprise subjects under follow-up at the times of diagnosis. 

 \small
\setlength{\unitlength}{0.7pt}
   \begin{center}
      \begin{picture}(300,230)
\thicklines
      \small
      \put(  0,215){\makebox(0,0)[cl]{Subject}}
      \put(  0, 20){\vector(1,0){300}}
      \put(300, 10){\makebox(0,0)[tr]{Time}}
      \put(  0,195){\makebox(0,0)[cl]{1}}
      \put( 10,195){\hfail{250}}
      \put(260, 15){\line(0,1){185}}
      \put(  0,180){\makebox(0,0)[cl]{2}}
      \put( 30,180){\hcens{100}}
      \put(  0,165){\makebox(0,0)[cl]{3}}
      \put( 80,165){\hcens{120}}
      \put(  0,150){\makebox(0,0)[cl]{4}}
      \put( 20,150){\hcens{250}}
      \put(  0,135){\makebox(0,0)[cl]{5}}
      \put(130,135){\hcens{170}}
      \put(  0,120){\makebox(0,0)[cl]{6}}
      \put( 40,120){\hfail{110}}
      \put(150, 15){\line(0,1){185}}
      \put(  0,105){\makebox(0,0)[cl]{7}}
      \put( 50,105){\hcens{80}}
      \put(  0, 90){\makebox(0,0)[cl]{8}}
      \put( 90, 90){\hcens{200}}
      \put(  0, 75){\makebox(0,0)[cl]{9}}
      \put( 10, 75){\hfail{95}}
      \put(105, 15){\line(0,1){185}}
      \put(  0, 60){\makebox(0,0)[cl]{10}}
      \put(100, 60){\hcens{80}}
      \put(  0, 45){\makebox(0,0)[cl]{11}}
      \put( 60, 45){\hfail{60}}
      \put(120, 15){\line(0,1){185}}
      \end{picture}
   \end{center}
\setlength{\unitlength}{1pt}
 \normalsize
Draw two controls at random from each risk set and
list the resulting matched sets.

\end{frame}

\end{comment}

% %----------------------------------------------------------------------
% \bs{The risk sets}
% Defined at each event time ($\bullet$):

% \renewcommand{\arraystretch}{1.5}
% \begin{tabular}{p{5em}|p{10em}|p{10em}|}
% Event & Risk set & Sample \\
% \hline
%     1 & & \\ \hline
%     2 & & \\ \hline
%     3 & & \\ \hline
%     4 & & \\ \hline
% \end{tabular}
% \renewcommand{\arraystretch}{1}
% \es

% \toggle{
% %----------------------------------------------------------------------
% \bs{The risk sets}
% Defined at each event time ($\bullet$):

% \begin{center}
% \begin{tabular}{r@{\ \ \ }l@{\ \ \ }l}
% Event & Risk set & Sample \\
% \hline
%     1 & 1,2,3,4,6,7,8,\textbf{9},10,11 & 4,1 \\
%     2 & 1,2,3,4,6,7,8,10,\textbf{11} & 2,1 \\
%     3 & 1,3,4,5,\textbf{6},8,10 & 8,3 \\
%     4 & \textbf{1},4,5,8 & 4,5 \\ \hline
% \end{tabular}
% \end{center}

% Note that individuals 4 and 1 are used twice as controls, and that
% individual 1 eventually becomes a case.

% This is perfectly OK, because they are controls (=at risk) at
% the time where they are selected to represent the risk set.

% \es
% }




\begin{comment}


%----------------------------------------------------------------------


\begin{frame}
\frametitle{Statistical modelling of case-base data}
\pause

Most popular approach: fit proportional hazards (PH) model.
{Hazard} % (theoretical incidence of disease) 
at age $t$ for cohort member $i$ with risk factor 
profile $x_i$% = (x_{i1}, \dots, x_{ip})$:
$$ \lambda_i(t, x_i ;\beta) =
  \lambda_0(t ) \exp(x_{i}^{\T}\beta), $$
% where $\lambda_0(t)$ = baseline hazard, 
% $\beta = (\beta_1 \dots, \beta_p)$ are
% parameters: \\ \pause
% $\beta_j$ = log(HR$_j$) for unit change of value in covariate $X_j$.
Estimation of parameters, typical methods based on:\pause
\bi
% \item full cohort: partial likelihood,
\item NCC: partial likelihood $\Leftrightarrow$ cond'l logistic model,
\item CC: weighted partial likelihood, or 
pseudo-likelihood (Samuelsen et al \emph{SJS} 2007).
% \item CC: weighted pseudo-likelihood,
\ei
\pause
Estimation of ``absolute'' risks also feasible by proper weighting
(Langholz \& Borgan \emph{Biometrics} 1997, Borgan \emph{SMMR} 2002)

\pause
Full-likelihood solutions recently developed --  computationally 
challenging: EM algorithm or MCMC data augmentation  \\
(Scheike \& Juul \emph{Biostat's} 2004, Scheike \& Martinussen
\emph{SJS} 2004, Kulathinal \& Arjas \emph{SJS} 2006, 
Saarela {\it et al.} \emph{SiM} 2008)

\end{frame}



% \section{$\bullet\quad$Comparison \& conclusion}

\end{comment}

%----------------------------------------------------------------------

\begin{frame}
\frametitle{Modelling in NCC and other matched studies}

\pause
Cox proportional hazards model: 
% {Hazard}  
% at time (or age) $t$ for subject $i$ with risk factor 
% profile $x_i = (x_{i1}, \dots, x_{ip})$:
$$ \lambda_i(t, x_i ;\beta) =
  \lambda_0(t ) \exp(x_{i1}\beta_1 + \dots + x_{ip} \beta_p ) , $$
 % $\lambda_0(t)$ = baseline hazard, 
% $\beta = (\beta_1 \dots, \beta_p)$ are
% parameters: \\
%$\beta_j$ = log(HR$_j$) for unit change in covariate $X_j$.
\pause
Estimation: partial likelihood $L^P = \prod_k L^P_k$:
\[
 L^P_k  = \exp(\eta_{i_k} ) /  \sum_{i\in \widetilde{R}(t_k)} \exp(\eta_i), 
\]
where $\widetilde{R}(t_k)$ = \textbf{sampled risk set}  
at observed event time $t_k$, containing the \\ case + sampled controls
 $(t_1 < \dots < t_D)$
 \pause
\bi
\item[$\Rightarrow$] Fit stratified Cox model, with
$\widetilde{R}(t_k)$'s as the strata.
 % being defined by the sampled risk sets.
% (\texttt{coxph( suob ~ X + strata(rset)}) 
\pause
\medskip
\item[$\Leftrightarrow$] \textbf{Conditional logistic regression} \\
 -- function \texttt{clogit()} in \texttt{survival}, wrapper
 of {\tt coxph()}.

\ei
% Estimation of ''absolute'' risks also feasible by proper weighting.
% \bigskip
% Full-likelihood solutions recently developed \\
% --  computationally challenging (\textit{e.g.} EM, MCMC).

\end{frame}

\begin{frame}
  \frametitle{Modelling case-cohort data}

Cox's PH model $\lambda_i(t) = \lambda_0(t)\exp(\eta_i)$ again, but \dots
\pause
  \begin{itemize}
  \item Analysis of survival data relies on the theoretical principle that \\
     {\it you can't know the future}.
 \pause
  \medskip  
  \item  Case-cohort sampling breaks this principle: \\ cases are sampled
    based on what {\it is known} to be happening to them \\ during follow-up.
\pause
  \medskip  
  \item The union of cases and subcohort is a mixture 
    \begin{enumerate}
{\normalsize    
    \item  random sample of the population, and
    \medskip
    \item ``high risk'' subjects who are 
    \textit{certain} to become cases.
}    
    \end{enumerate}
\pause
% \medskip   
  \item[$\Rightarrow$] Ordinary Cox partial likelihood is wrong.
\pause
 \medskip
 \item 
  Overrepresentation of cases must be corrected for, by \\ 
  (I) \textbf{weighting}, or (II) \textbf{late entry method}.
\end{itemize}

\end{frame}

\begin{frame}
  \frametitle{Correction method I -- weighting}

The method of \textbf{weighted partial likelihood}
borrows some basics ideas from survey sampling theory.
\pause
\begin{itemize} 
\item
 Sampled risk sets \\ $\widetilde{R}(t_k)$ 
  = \{cases\} $\cup$ \{subcohort members\} at risk at $t_k$.
\pause
 \medskip
 \item
  Weights:
 {\normalsize 
 \begin{itemize}
  \item[$-$]
   $w$ = 1 for all cases (within and outside the subcohort),
\pause   
    \item[$-$]  
   $w$ = $N_{\text{non-cases}}/
   n_{\text{non-cases}}$ = inverse of sampling-fraction \\
  ${}\qquad$ $f$ for selecting a non-case to the subcohort.
  \end{itemize} 
  }
  \pause
  \medskip
  \item
  Function {\tt coxph()} with option {\tt weights = w} would
  provide \\ consistent estimation of $\beta$ parameters.
  \pause
 \medskip
\item
However, the SEs must be corrected! % using either
% robust ``sandwich'' covariance matrix option {\tt robust = TRUE}
% or utilizing the \textbf{dfbeta} residuals.
\pause
\medskip
\item
R solution: Function {\tt cch()} --
a wrapper of {\tt coxph()} -- \\ in package {\tt survival}, 
with {\tt method = "LinYing"}.
\end{itemize}

\end{frame}

\begin{comment}

\begin{frame}
  \frametitle{Correction method II -- late entry}
 
 Principle of \textbf{pseudo-likelihood} or \textbf{late entry} method:
 \pause
  \begin{itemize}
  \item Pretend we \emph{don't know} risk factor values for  
  cases {\it outside subcohort} until the times $t_k$ they become cases.
\pause
  \medskip  
  \item These cases only become ``at risk'' an instant before $t_k$.
  % \small In \textsf{R} late entry may be specified by
 % \texttt{  Surv(entry.time, exit.time, status) }  
\pause
  \medskip
 \item 
   The sampled risk set $\widetilde R(t_k)$
    at $t_k$ thus contains the case and the
   subcohort non-cases still at risk.
   \pause
  \medskip 
  \item At any time, subjects in the current sample are representative of the 
    population at risk.
\pause    
  \medskip
  \item 
However, the risk sets across the event times are dependent $\Rightarrow$
 Must correct the standard errors. 
 \pause    
  \medskip
  \item 
R: Function {\tt cch()} with   
{\tt method = "Prentice"}.  
 \end{itemize}
\end{frame}
  

\begin{frame}
\frametitle{Late entry method applied in R}

Function {\tt coxph()} works again, but with modified data:
\pause
\bi
\item
Non-cases within subcohort: original entry and exit times
\pause
\medskip
\item
Cases {\it outside} subcohort: \\
 entry time = exit time -- ``epsilon''.
\pause
\medskip
\item
Cases {\it within} the subcohort are given two records:
\pause
\bi
{\normalsize
\item[1.]
  as a {\it non-case} until time of outcome event: \\
   ``epsilon'' subtracted from original exit time, and \\
   event indicator = 0.
\pause
\medskip
\item[2.]
  as a {\it case} at the time of outcome: \\
   entry time = exit time -- ``epsilon'',
  and \\ event indicator = 1.
  }
\ei

\ei
\pause
Alternative: Function {\tt cch()} with   
{\tt method = "Prentice"}. \\ 
-- No data modification needed.

\end{frame}

 
\begin{frame}
  \frametitle{How both correction methods work}

\bi
\item
  We estimate the log-hazard ratio parameters $\beta$ by
  solving the \textbf{estimating equations} 
  \[
  S(\widehat{\beta}) = 0,
  \]
  based on the 
  {\bf gradient} or {\bf score function}
   $S = \partial \log L^P/\partial \beta$
\medskip
\item
These methods work in a case-cohort setting,
as it has been shown for both that the estimation
equation is {\it unbiased}, {\it i.e.}
  \[
  \mbox{E}\left(S(\beta)\right) = 0
  \]
\item  
  Unbiasedness of the estimating equation 
  implies that the 
  estimator $\widehat{\beta}$ is {\it consistent}.
\ei
\end{frame}


\begin{frame}
  \frametitle{Score function in the Cox model}

  There is a contribution to the score function every time
  a case occurs.
  \[
  S(\beta) = \sum_i S_i (\beta)
  \]  
  Suppose subject $i$ becomes a case
  \[
  \begin{array}{rcc}
    S_i(\beta) = x_i & - & \mbox{E}(x_{\mbox{case}}) \\
    \pause 
    &   & \downarrow \\
    &   & \frac{\sum_{j \in \mathcal{R}_i} r_j z_j}
	{\sum_{j \in \mathcal{R}_i} r_j}
  \end{array}
  \]
  \pause
  Where
  \begin{eqnarray*}
    x_j & = & \mbox{Vector of covariates} \\
    r_j & = & \mbox{exp}(\beta' z_j) = \mbox{Relative rate function}\\
    \mathcal{R}_i & = & \mbox{Set of subjects at risk when i becomes a case}
  \end{eqnarray*}
  
\end{frame}



\begin{frame}
  \frametitle{Correcting standard errors}
  
  \begin{itemize}
  \item
  SEs  in the output  from a usual Cox 
   model fit in R \\ % using {\tt coxph()} function 
	(or % analogous procedures in
	SAS, Stata, {\it etc.})
  are incorrect for case-cohort data.
  \pause
  \medskip
  \item  Function {\tt coxph()} by default assumes that 
  the partial likelihood contributions    
   at each $t_k$ are \emph{independent}.
  \pause
  \medskip
  \item 
  Yet, with case-cohort data
   we keep reusing the same sampled subcohort as representing
    the whole cohort \\ $\Rightarrow$ \textit{dependence} is induced!
  \pause
  \medskip
  \item Valid error margins are obtained from appropriately estimated 
  covariance matrix cov$(\widehat\beta )$: % of the regression coefficients.
  {\normalsize
  \bi
  \item[(a)] \textbf{Robust ``sandwich'' estimator} of cov$(\widehat\beta)$.% (Barlow 1994), 
  \medskip
  \item[(b)] Estimation of 
  $\text{cov}(\widehat\beta)$ based on \textbf{dfbeta residuals}
  from the subcohort.
%  (Self \& Prentice 1988).
   \ei
   }
  \end{itemize}
	
	{\small See Samuelsen {\it et al.} (2007) {\it Scand J Stat} {\bf 34}: 103-119.}

\end{frame}

\end{comment}
\begin{comment}

\begin{frame}
  \frametitle{Correcting SEs: Robust covariance matrix}
  
  \medskip
  Using the \texttt{robust=TRUE} argument with \texttt{coxph()}
  gives the robust estimate of the covariance matrix.
  
\pause  
  \bigskip
  This is a general-purpose method
  \begin{itemize}
  \item Valid for any unbiased estimating function.
 \pause
  \medskip
  \item Calculates a ``jackknife'' estimate of the variance of the score
    function.
 \pause   
  \medskip  
  \item An approximate ``bootstrap'' method.
  \end{itemize}

\end{frame}

 

\begin{frame}
 \frametitle{Correcting SEs: Use of dfbeta residuals}
  
  \pause
  \begin{itemize}
  \item
  The Self-Prentice variance estimator adds a \\
  \textbf{penalty} to
  cov$(\widehat{\beta})$ for reusing the same sample.
\pause
\medskip
  \item Define \textbf{dfbeta residual} $\Delta\widehat\beta_i$ for subject $i$ as the \\ 
  change in $\widehat{\beta}$ when obs'n   $i$ is deleted,
 \pause
  \medskip  
  \item Calculate  $\Delta\widehat\beta_i$ for all subjects $i$ 
  in subcohort.
  \pause
  \medskip
  \item If we did a new analysis after deleting at random one subcohort member, cov$(\widehat{\beta})$ would increase by
    \mbox{cov}$(\Delta\widehat\beta)$.
 \pause
 \medskip   
  \item Yet, instead of having {\it selected} the subcohort
    at random, \\
     we in a way {\it delete all the rest} in the whole cohort! \\
\pause      
      \medskip 
    $\Rightarrow$ Appropriate decrease in precision.
  \end{itemize}
\end{frame}


\begin{frame}
  \frametitle{Heuristic derivation of the variance penalty}

\bi
\item
  If we start with a larger cohort of size $N_C$, 
  the penalty for deleting a random member is smaller
  \[
  \frac{N_{SC}}{N_C} \mbox{cov}(\Delta\widehat\beta)
  \]
  where $N_{SC}$ is the size of the sub-cohort.
\medskip
\item  
  Apply this penalty for each subject excluded from the subcohort
  \[
  (N_C - N_{SC}) \frac{N_{SC}}{N_C} \mbox{cov}(\Delta\widehat\beta)
  \]
\item  
  This can also be written
  \[
  ( 1 - \alpha) \sum_{i \in SC} \Delta\widehat\beta_i \Delta\widehat\beta_i^T
  \]
  where $\alpha = N_{SC}/N_C$ is the sampling fraction.
  \ei
  
\end{frame}

\end{comment}

\begin{frame}
\frametitle{Comparison of NCC and CC designs}

\pause
\bi
\item Statistical efficiency
\bi
\item[ ] Broadly similar in NCC
and CC with similar numbers of cases and controls.
\ei
\pause
\item Statistical modelling and valid inference \pause
\bi
\item[ ] Straightforward for both designs with appropriate software, \\ 
now  widely available for CC, too 
% fitting
% conditional logistic regression or proportional hazards models.
% \pause
% \medskip
% \item[CC:] Software for Cox PH regression can 
% be used when supplemented with additional tricks.
 \ei
 \pause
 \item Analysis of outcome rates on several time scales?
 \pause
 \bi
 \item[NCC:] Only the time scale used in risk set definition
 can be the \\ time variable $t$ in the baseline hazard 
 of PH model.\pause\medskip
 \item[CC:] Different choices for the basic time  in
    PH model possible, because \\ subcohort members are
    not time-matched to cases. 
 \ei
 \ei
 \end{frame}
 
 \begin{frame}
 \frametitle{Comparison of designs (cont'd)}
 \pause
 \bi
 \item Missing data
 \pause
 \bi \item[NCC:]
   With close 1:1 matching, a case-control pair is lost, \\
  if either of the two has data missing on key exposure(s).
  \pause\medskip
  \item[CC:] Missingness of few data items is less serious.
  \ei 
  \pause
 \item Quality and comparability of biological measurements
 \pause
 \bi
 \item[NCC:] Allows 
 each case  and its controls to be matched also 
 for analytic batch, storage time, freeze-thaw cycle,  
    $\to$ better comparability. \pause\medskip
  \item[CC:] Measurements for subcohort 
  performed at different times than for cases \\ 
	$\to$  differential quality \& misclassification.
 \ei
 \pause
 \item  Possibility for studying many diseases with same controls
 \pause
 \bi
 \item[NCC:] Complicated, but possible if matching is not too refined.
 \medskip
 \item[CC:] Easy, as no subcohort member is ``tied'' with any case.
 \ei
 \ei
 
 \end{frame}
 

\begin{frame}
\frametitle{Conclusion}

\pause
\begin{itemize}
\item
``Case-controlling'' is very cost-effective.
\pause
\medskip
\item
 Case-cohort design is useful especially when several outcomes
are of interest, given that the measurements on stored
materials remain stable during the study.
\pause
\medskip
\item
 Nested case-control design is better suited \textit{e.g.}
 for studies involving biomarkers that can be infuenced by analytic batch,
long-term storage, and freeze-thaw cycles.
\pause
\medskip
\item
 Matching helps in improving effciency and in reducing bias \\
-- but only if properly done.
\pause
\medskip
\item
Handy R tools are available for all designs.

\end{itemize}

\end{frame}

\end{document}

\begin{frame}
\frametitle{Modelling case-control data}

Common paradigm -- {\bf logistic regression model} (LRM)
\bi
\item \textbf{unconditional} LRM for unmatched data, 
\item \textbf{conditional} LRM for matched data,
\item \textbf{weighted} LRM for case-cohort data.
\ei

\textbf{Design A} -- exclusive (case-noncase) sampling
 in both a cross-sectional and longitudinal base: 
\bi
\item
{\it Unmatched} design:
Unconditional LRM to estimate prevalence odds ratios or risk odds ratios. 
\medskip
\item
{\it Matched} design: 
Matching factors need to be allowed for,
 by either conditional LRM or
appropriate incorporation in unconditional LRM.
-- More about this soon.
% including them appropriately
% as categorical, stratifying factors, and/or quantitative terms
% in an unconditional model.
\medskip
\item
No adjustments needed for SEs \& CIs in the output.
\ei

\end{frame}

\begin{frame}
\frametitle{Modelling case-control data (cont'd)}

\textbf{Design C} -- concurrent sampling
 in both closed and open populations, estimation
 of hazard ratios for the $X_j$s.
\bi
\item
{\it Unmatched} design:
Unconditional LRM, as in design A. 
\medskip
\item
{\it Matched} design -- not necessarily involving 
time-matching: Same principles as in design A.
% Conditional LRM, in which intercept parameters for 
% vindividual matched sets are conditioned out.
\medskip
\item
{\it Time-matched}, or {\it nested case-control} (NCC) design: \\
% based on sampling from the {\it risk sets}: 
Cox model stratified by times $t_k$ of the new cases 
$$ \lambda_k(t; x_i) = \lambda_{0k} \exp(x_i^{\small{\text T}} \beta). $$
$\Rightarrow$ conditional LRM. 
\medskip
\item
No further adjustments needed to the output.
\ei


\end{frame}

\begin{frame}
\frametitle{Dealing with matching}

\medskip
\begin{itemize}
\item
Close individual matching on a non-quantifiable factor
 (\textit{e.g.} sibship) induces a separate 
(intercept) parameter $\alpha_i$ for each matched 
case-control set or stratum.
\begin{itemize}
{\normalsize 
\item Unconditional LRM with fixed $\alpha_i$s breaks down.
\medskip
\item Use conditional LRM that throws out $\alpha_i$s. 
\medskip
\item Unconditional LRM with $\alpha_i$s as random terms?
}
\end{itemize} 

\medskip
\item Matching on categorical 
factors with few classes ({\it e.g.} sex, marital status):
{\em Post hoc} pooling into bigger strata $\Rightarrow$
 frequency matching. Use unconditional LRM including the stratification.
% and these factors included as covariates.
\medskip
\item Same principles in designs {\it A} and {\it C} with matching.

\end{itemize}

\end{frame}

\begin{frame}

\frametitle{Matching on quantitative variables $Z$}

\bi
\item
Conditional LRM remains robust in principle.
\medskip
\item
If the tolerance for the difference of $Z$-values
 between cases and controls
is wide ({\it e.g.} $\pm$ 5 years in individual matching, or 
belonging to the same 10-year ageband in frequency matching),
 residual confounding may remain.
\medskip
\item
{\it Post hoc} stratification into finer intervals? \\
 -- May result in too many uninformative strata.
\medskip
\item
Dissolving the matching and inclusion of $Z$ into LRM
as {\it e.g.} linear term? -- May lead to bias in matched data 
even if the assumed model were true in the study base!  
\medskip
\item
Modelling upon stratification: For each $i$ add
difference $Z_i - \bar Z_s$ btw ind'l value 
and stratum mean to LRM.
\ei
See Greenland (1997) \textit{Stat Med} {\bf 16}: 515-526.
\end{frame}

\begin{frame}
\frametitle{Modelling case-cohort data}

\medskip
{\bf Design B} -- inclusive (case-cohort) sampling in cross-sectional and longitudinal base; for the latter first with a fixed risk period and complete follow-up.
\bi
\item
Unconditional LRM to estimate prevalence probability ratios or risk ratios. 
\medskip
\item
Non-cases in subcohort weighted by $1/f$, where \\
$f = C/N$ is the sampling fraction to the subcohort. 
\medskip
\item
Cases in subcohort have two rows in the data: \\
 (1) as a case, and (2) as a cohort member. 
\medskip
\item
SEs need to be corrected by using {\it e.g.} \\
 robust ``sandwich'' covariance matrix.
\ei
Schouten {\it et al.} (1993) {\it Stat Med} {\bf 12}: 1733-1745.

\end{frame}




\begin{comment}

\begin{frame}
  \frametitle{Analysis of case-cohort data}

Cox's PH model $\lambda_i(t) = \lambda_0(t)\exp(\eta_i)$ again, but ...
  \begin{itemize}
  \item Analysis of survival data relies on the theore- tical principle that
    you {can't know the future}.
  \medskip  
  \item  Case-cohort sampling breaks this principle: cases are sampled
    based on what {\bf will} happen to them during follow-up.
  \medskip  
  \item The union of cases and subcohort is a mixture
    \begin{enumerate}
    \item  random sample of the population, and
    \item ``high risk'' subjects who are \textit{certain} to become cases.
    \end{enumerate}
  \item[$\Rightarrow$] Ordinary Cox partial likelihood is wrong.
  \end{itemize}

\end{frame}

\section{Estimates}

\end{comment}


 

\begin{frame}

\item Different variants of case-control studies exist, and the
terminology varies.
\medskip
\item Unmatched studies with few ``large'' strata: --
\textit{Unconditional logistic regression} using \texttt{glm()},
and including the factors determining
 stratified sampling of the controls to the model.
\medskip
\item Matched studies on ``sparse'' data with
many thin strata: -- \textit{Conditional logistic regression}
 using function
\texttt{clogit()}, and defining the variable distinguishing
matched sets as a \texttt{strata()} term in the model formula.
\end{itemize}
\end{frame} 


\end{document}



%----------------------------------------------------------------------

% \renewcommand{\currentlecture}{References}
% %----------------------------------------------------------------------
% \begin{frame}[plain,allowframebreaks]
% \frametitle{References}
% \footnotesize
% \bibliographystyle{plain}
% \bibliography{SPE}
% \end{frame}

\end{document}
