



\renewcommand{\rwpre}{./graph/DMDK}
\section{Time-splitting, time-scales and SMR}

This exercise is about mortaity among Danish Diabetes patients. It is
based on the dataset \texttt{DMlate}, a random sample of 10,000
patients from the Danish Diabetes Register (scrambeled dates), all
with date of diagnosis after 1994.

\begin{enumerate}


\item First, we load the \texttt{Epi} package (and two other packages
  we will need later) and the dataset, and take a look at it:
\begin{Schunk}
\begin{Sinput}
 options( width=90 )
 library( Epi )
 library( popEpi )
 library( mgcv )
 data( DMlate )
 str( DMlate )
\end{Sinput}
\begin{Soutput}
'data.frame':	10000 obs. of  7 variables:
 $ sex  : Factor w/ 2 levels "M","F": 2 1 2 2 1 2 1 1 2 1 ...
 $ dobth: num  1940 1939 1918 1965 1933 ...
 $ dodm : num  1999 2003 2005 2009 2009 ...
 $ dodth: num  NA NA NA NA NA ...
 $ dooad: num  NA 2007 NA NA NA ...
 $ doins: num  NA NA NA NA NA NA NA NA NA NA ...
 $ dox  : num  2010 2010 2010 2010 2010 ...
\end{Soutput}
\begin{Sinput}
 head( DMlate )
\end{Sinput}
\begin{Soutput}
       sex    dobth     dodm    dodth    dooad doins      dox
50185    F 1940.256 1998.917       NA       NA    NA 2009.997
307563   M 1939.218 2003.309       NA 2007.446    NA 2009.997
294104   F 1918.301 2004.552       NA       NA    NA 2009.997
336439   F 1965.225 2009.261       NA       NA    NA 2009.997
245651   M 1932.877 2008.653       NA       NA    NA 2009.997
216824   F 1927.870 2007.886 2009.923       NA    NA 2009.923
\end{Soutput}
\begin{Sinput}
 summary( DMlate )
\end{Sinput}
\begin{Soutput}
 sex          dobth           dodm          dodth          dooad          doins     
 M:5185   Min.   :1898   Min.   :1995   Min.   :1995   Min.   :1995   Min.   :1995  
 F:4815   1st Qu.:1930   1st Qu.:2000   1st Qu.:2002   1st Qu.:2001   1st Qu.:2001  
          Median :1941   Median :2004   Median :2005   Median :2004   Median :2005  
          Mean   :1942   Mean   :2003   Mean   :2005   Mean   :2004   Mean   :2004  
          3rd Qu.:1951   3rd Qu.:2007   3rd Qu.:2008   3rd Qu.:2007   3rd Qu.:2007  
          Max.   :2008   Max.   :2010   Max.   :2010   Max.   :2010   Max.   :2010  
                                        NA's   :7497   NA's   :4503   NA's   :8209  
      dox      
 Min.   :1995  
 1st Qu.:2010  
 Median :2010  
 Mean   :2009  
 3rd Qu.:2010  
 Max.   :2010  
\end{Soutput}
\end{Schunk}


\item We then set up the dataset as a \texttt{Lexis} object with age, calendar
  time and duration of diabetes as timescales, and date of death as
  event.

  In the dataset we have a date of exit \texttt{dox} which is either
  the day of censoring or the date of death:
\begin{Schunk}
\begin{Sinput}
 with( DMlate, table( dead=!is.na(dodth),
                      same=(dodth==dox), exclude=NULL ) )
\end{Sinput}
\begin{Soutput}
       same
dead    TRUE <NA>
  FALSE    0 7497
  TRUE  2503    0
\end{Soutput}
\end{Schunk}
  So we can set up the \texttt{Lexis} object by specifying the
  timescales and the exit status via \texttt{!is.na(dodth)}:
\begin{Schunk}
\begin{Sinput}
 LL <- Lexis( entry = list( A = dodm-dobth,
                            P = dodm,
                          dur = 0 ),
               exit = list( P = dox ),
        exit.status = factor( !is.na(dodth),
                              labels=c("Alive","Dead") ),
               data = DMlate )
\end{Sinput}
\begin{Soutput}
NOTE: entry.status has been set to "Alive" for all.
\end{Soutput}
\end{Schunk}
Note that we made sure the the \texttt{first} level of
\texttt{exit.status} is ``Alive'', because the default is to use the
first level as entry status when \texttt{entry.status} is not given as
argument to \texttt{Lexis}.

  The 4 persons are persons that have identical date of diabetes and
  date of death; they can be found by using \texttt{keep.dropped=TRUE}:
\begin{Schunk}
\begin{Sinput}
 LL <- Lexis( entry = list( A = dodm-dobth,
                            P = dodm,
                          dur = 0 ),
               exit = list( P = dox ),
        exit.status = factor( !is.na(dodth),
                              labels=c("Alive","Dead") ),
               data = DMlate,
       keep.dropped = TRUE )
\end{Sinput}
\begin{Soutput}
NOTE: entry.status has been set to "Alive" for all.
\end{Soutput}
\end{Schunk}
  The dropped persons are:
\begin{Schunk}
\begin{Sinput}
 attr( LL, 'dropped' )
\end{Sinput}
\begin{Soutput}
       sex    dobth     dodm    dodth dooad doins      dox
173047   M 1917.863 1999.457 1999.457    NA    NA 1999.457
361856   F 1936.067 1996.984 1996.984    NA    NA 1996.984
245324   M 1917.877 1999.906 1999.906    NA    NA 1999.906
318694   F 1919.060 2006.794 2006.794    NA    NA 2006.794
\end{Soutput}
\end{Schunk}
  We can get an overview of the data by using the \texttt{summary}
  function on the object:
\begin{Schunk}
\begin{Sinput}
 summary( LL )
\end{Sinput}
\begin{Soutput}
Transitions:
     To
From    Alive Dead  Records:  Events: Risk time:  Persons:
  Alive  7497 2499      9996     2499   54273.27      9996
\end{Soutput}
\begin{Sinput}
 head( LL )
\end{Sinput}
\begin{Soutput}
              A        P dur    lex.dur lex.Cst lex.Xst lex.id sex    dobth     dodm
50185  58.66119 1998.917   0 11.0800821   Alive   Alive      1   F 1940.256 1998.917
307563 64.09035 2003.309   0  6.6885695   Alive   Alive      2   M 1939.218 2003.309
294104 86.25051 2004.552   0  5.4455852   Alive   Alive      3   F 1918.301 2004.552
336439 44.03559 2009.261   0  0.7364819   Alive   Alive      4   F 1965.225 2009.261
245651 75.77550 2008.653   0  1.3442847   Alive   Alive      5   M 1932.877 2008.653
216824 80.01643 2007.886   0  2.0369610   Alive    Dead      6   F 1927.870 2007.886
          dodth    dooad doins      dox
50185        NA       NA    NA 2009.997
307563       NA 2007.446    NA 2009.997
294104       NA       NA    NA 2009.997
336439       NA       NA    NA 2009.997
245651       NA       NA    NA 2009.997
216824 2009.923       NA    NA 2009.923
\end{Soutput}
\end{Schunk}


\item A crude picture of the mortality by sex can be obtained by
  the \texttt{stat.table} function:
\begin{Schunk}
\begin{Sinput}
 stat.table( sex,
             list( D=sum( lex.Xst=="Dead" ),
                   Y=sum( lex.dur ),
                rate=ratio( lex.Xst=="Dead", lex.dur, 1000 ) ),
             margins=TRUE,
             data=LL )
\end{Sinput}
\begin{Soutput}
 --------------------------------- 
 sex           D        Y    rate  
 --------------------------------- 
 M       1343.00 27614.21   48.63  
 F       1156.00 26659.05   43.36  
                                   
 Total   2499.00 54273.27   46.04  
 --------------------------------- 
\end{Soutput}
\end{Schunk}
  So not surprising, we see that men have a higher mortality than
  women, here apparently only some 10\%.


  
\item When we want to assess how mortality depends on age, calendar
  time and duration or to comapre with population mortality, we would
  in principle split the follow-up along all three time scales, but in
  practice it is sufficient to split it along one of the time-scales
  and then just use the value of each of the time-scales at the left
  endpoint of the intervals for analysis and for matching with
  population mortality. Note however that this requires that
  time-scales are treated as \emph{quantitative} variables in the
  modeling.

  We note that the total follow-up time was some 54,000 person-years,
  so if we split the follow-up in 6-month intervals we should get a
  bit more than 110,000 records. Note that in the \texttt{popEpi}
  package there is a function with the same functionality, which is
  faster (particularly for large datasets) and has a somewhat smarter
  syntax --- it returns a \texttt{data.table}:
\begin{Schunk}
\begin{Sinput}
 system.time( SL <- splitLexis( LL, breaks=seq(0,125,1/2), time.scale="A" ) )
\end{Sinput}
\begin{Soutput}
   user  system elapsed 
  2.388   0.260   2.652 
\end{Soutput}
\begin{Sinput}
 summary( SL ) ; class( SL )
\end{Sinput}
\begin{Soutput}
Transitions:
     To
From     Alive Dead  Records:  Events: Risk time:  Persons:
  Alive 115974 2499    118473     2499   54273.27      9996
\end{Soutput}
\begin{Soutput}
[1] "Lexis"      "data.frame"
\end{Soutput}
\begin{Sinput}
 system.time( SL <- splitMulti( LL, A=seq(0,125,1/2) ) )
\end{Sinput}
\begin{Soutput}
   user  system elapsed 
  1.055   0.164   1.204 
\end{Soutput}
\begin{Sinput}
 summary( SL ) ; class( SL )
\end{Sinput}
\begin{Soutput}
Transitions:
     To
From     Alive Dead  Records:  Events: Risk time:  Persons:
  Alive 115974 2499    118473     2499   54273.27      9996
\end{Soutput}
\begin{Soutput}
[1] "Lexis"      "data.table" "data.frame"
\end{Soutput}
\begin{Sinput}
 summary( LL )
\end{Sinput}
\begin{Soutput}
Transitions:
     To
From    Alive Dead  Records:  Events: Risk time:  Persons:
  Alive  7497 2499      9996     2499   54273.27      9996
\end{Soutput}
\end{Schunk}
  We see that the number of records have increased, but the number of
  persons, events and person-years is still the same as in
  \texttt{LL}. Thus, the amount of follow-up information is still the
  same; it is just distributed over more records, and hence allowing
  more detailed analyses. 

\end{enumerate}

\subsection*{SMR}




The SMR is the \textbf{S}tandardized \textbf{M}ortality
\textbf{R}atio, which is the mortality rate-ratio between the diabetes
patients and the general population.  In real studies we would
subtract the deaths and the person-years among the diabetes patients
from those of the general population, but since we do not have access
to these, we make the comparison to the general population at large,
\textit{i.e.} also including the diabetes patients.

We now want to include the population mortality rates as a fixed
variable in the split dataset; for each record in the split dataset we
attach the value of the population mortality for the relevant sex, and
and calendar time.

This can be achieved in two ways: Either we just use the current split
of follow-up time and allocate the population mortality rates for some
suitably chosen (mid-)point of the follow-up in each, or we make a
second split by date, so that follow-up in the diabetes patients is in
the same classification of age and data as the population mortality
table.

\begin{enumerate}[resume]




\item Using the former approach we shall include as an extra variable
  the population mortality as available from the data set
  \texttt{M.dk}.

  First create the variables in the diabetes dataset that we need for
  matching with the age and period classification of the population
  mortality data, that is age, date (and sex) at the midpoint of each of
  the intervals (or rater at a point 3 months after the left endpoint
  of the interval --- recall we split the follow-up in 6 month
  intervals).

  We need to have variables of the same type when we merge, so we must
  transform the sex variable in \texttt{M.dk} to a factor, and must
  for each follow-up interval in the \texttt{SL} data have an age and
  a period variable that can be used in merging with the population data. 
\begin{Schunk}
\begin{Sinput}
 str( SL )
\end{Sinput}
\begin{Soutput}
Classes ‘Lexis’, ‘data.table’ and 'data.frame':	118473 obs. of  14 variables:
 $ lex.id : int  1 1 1 1 1 1 1 1 1 1 ...
 $ A      : num  58.7 59 59.5 60 60.5 ...
 $ P      : num  1999 1999 2000 2000 2001 ...
 $ dur    : num  0 0.339 0.839 1.339 1.839 ...
 $ lex.dur: num  0.339 0.5 0.5 0.5 0.5 ...
 $ lex.Cst: Factor w/ 2 levels "Alive","Dead": 1 1 1 1 1 1 1 1 1 1 ...
 $ lex.Xst: Factor w/ 2 levels "Alive","Dead": 1 1 1 1 1 1 1 1 1 1 ...
 $ sex    : Factor w/ 2 levels "M","F": 2 2 2 2 2 2 2 2 2 2 ...
 $ dobth  : num  1940 1940 1940 1940 1940 ...
 $ dodm   : num  1999 1999 1999 1999 1999 ...
 $ dodth  : num  NA NA NA NA NA NA NA NA NA NA ...
 $ dooad  : num  NA NA NA NA NA NA NA NA NA NA ...
 $ doins  : num  NA NA NA NA NA NA NA NA NA NA ...
 $ dox    : num  2010 2010 2010 2010 2010 ...
 - attr(*, ".internal.selfref")=<externalptr> 
 - attr(*, "time.scales")= chr  "A" "P" "dur"
 - attr(*, "time.since")= chr  "" "" ""
 - attr(*, "breaks")=List of 3
  ..$ A  : num  0 0.5 1 1.5 2 2.5 3 3.5 4 4.5 ...
  ..$ P  : NULL
  ..$ dur: NULL
\end{Soutput}
\begin{Sinput}
 SL$Am <- floor( SL$A+0.25 )
 SL$Pm <- floor( SL$P+0.25 )
 data( M.dk )
 str( M.dk )
\end{Sinput}
\begin{Soutput}
'data.frame':	7800 obs. of  6 variables:
 $ A   : num  0 0 0 0 0 0 0 0 0 0 ...
 $ sex : num  1 2 1 2 1 2 1 2 1 2 ...
 $ P   : num  1974 1974 1975 1975 1976 ...
 $ D   : num  459 303 435 311 405 258 332 205 312 233 ...
 $ Y   : num  35963 34383 36099 34652 34965 ...
 $ rate: num  12.76 8.81 12.05 8.97 11.58 ...
 - attr(*, "Contents")= chr "Number of deaths and risk time in Denmark"
\end{Soutput}
\begin{Sinput}
 M.dk <- transform( M.dk, Am = A,
                          Pm = P,
                         sex = factor( sex, labels=c("M","F") ) )
 str( M.dk )
\end{Sinput}
\begin{Soutput}
'data.frame':	7800 obs. of  8 variables:
 $ A   : num  0 0 0 0 0 0 0 0 0 0 ...
 $ sex : Factor w/ 2 levels "M","F": 1 2 1 2 1 2 1 2 1 2 ...
 $ P   : num  1974 1974 1975 1975 1976 ...
 $ D   : num  459 303 435 311 405 258 332 205 312 233 ...
 $ Y   : num  35963 34383 36099 34652 34965 ...
 $ rate: num  12.76 8.81 12.05 8.97 11.58 ...
 $ Am  : num  0 0 0 0 0 0 0 0 0 0 ...
 $ Pm  : num  1974 1974 1975 1975 1976 ...
\end{Soutput}
\end{Schunk}
We then match the rates from \texttt{M.dk} into \texttt{SL} ---
\texttt{sex}, \texttt{Am} and \texttt{Pm} are the common variables,
and therefore the match is on these variables:
\begin{Schunk}
\begin{Sinput}
 SLr <- merge( SL, M.dk[,c("sex","Am","Pm","rate")] )
 dim( SL )
\end{Sinput}
\begin{Soutput}
[1] 118473     16
\end{Soutput}
\begin{Sinput}
 dim( SLr )
\end{Sinput}
\begin{Soutput}
[1] 118454     17
\end{Soutput}
\end{Schunk}
This merge only takes rows that have information from both data sets,
hence the slightly fewer rows in \texttt{SLr} than in \texttt{SL} ---
there are a few record in \texttt{SL} with age and period values that
do not exist in the population mortality data.
    
    
\item We compute the expected number of deaths as the person-time
   multiplied by the corresponding population rate recalling that the
   rate is given in units of deaths per 1000 PY, whereas
   \texttt{lex.dur} is in units of 1 PY:
\begin{Schunk}
\begin{Sinput}
 SLr$E <- SLr$lex.dur * SLr$rate / 1000
 stat.table( sex, 
             list( D = sum(lex.Xst=="Dead"), 
                   Y = sum(lex.dur), 
                   E = sum(E), 
                 SMR = ratio(lex.Xst=="Dead",E) ), 
             data = SLr,
             margin = TRUE ) 
\end{Sinput}
\begin{Soutput}
 ----------------------------------------- 
 sex           D        Y       E     SMR  
 ----------------------------------------- 
 M       1342.00 27611.40  796.11    1.69  
 F       1153.00 26654.52  747.77    1.54  
                                           
 Total   2495.00 54265.91 1543.88    1.62  
 ----------------------------------------- 
\end{Soutput}
\begin{Sinput}
 stat.table( list( sex, Age = floor(pmax(A,39)/10)*10 ), 
             list( D = sum(lex.Xst=="Dead"), 
                   Y = sum(lex.dur), 
                   E = sum(E), 
                 SMR = ratio(lex.Xst=="Dead",E) ), 
              margin = TRUE,
                data = SLr )
\end{Sinput}
\begin{Soutput}
 ---------------------------------------------------------------------------- 
        ---------------------------------Age--------------------------------- 
 sex          30      40       50       60       70      80      90    Total  
 ---------------------------------------------------------------------------- 
 M          6.00   32.00   119.00   275.00   486.00  348.00   76.00  1342.00  
         2129.67 3034.15  6273.51  7940.22  5842.90 2165.80  225.14 27611.40  
            1.94    9.47    48.48   142.38   275.67  254.92   63.26   796.11  
            3.09    3.38     2.45     1.93     1.76    1.37    1.20     1.69  
                                                                              
 F          5.00   15.00    62.00   157.00   331.00  423.00  160.00  1153.00  
         2576.33 2742.03  4491.68  6112.30  6383.09 3786.78  562.31 26654.52  
            1.24    5.01    22.00    74.00   204.44  318.81  122.26   747.77  
            4.02    2.99     2.82     2.12     1.62    1.33    1.31     1.54  
                                                                              
                                                                              
 Total     11.00   47.00   181.00   432.00   817.00  771.00  236.00  2495.00  
         4706.00 5776.18 10765.19 14052.52 12225.99 5952.59  787.46 54265.91  
            3.18   14.48    70.47   216.39   480.11  573.73  185.51  1543.88  
            3.45    3.25     2.57     2.00     1.70    1.34    1.27     1.62  
 ---------------------------------------------------------------------------- 
\end{Soutput}
\end{Schunk}
  We see that the SMR is pretty much the same for women and men, but
  also that there is a steep decrease in SMR by age. 



\item We can fit a poisson model with sex as the explanatory variable and
  log-expected as offset to derive the SMR (and c.i.).

  Some of the population mortality rates are 0, so we must exclude
  those records from the analysis.
\begin{Schunk}
\begin{Sinput}
 msmr <- glm( (lex.Xst=="Dead") ~ sex - 1 + offset(log(E)),
              family = poisson,
              data = subset(SLr,E>0) )
 ci.exp( msmr )
\end{Sinput}
\begin{Soutput}
     exp(Est.)     2.5%    97.5%
sexM  1.685699 1.597881 1.778344
sexF  1.541922 1.455442 1.633540
\end{Soutput}
\end{Schunk}
  These are the same SMRs as just coomputed by \texttt{stat.table},
  but now with confidence intervals.
  
  We can assess the ratios of SMRs between men and women by using the
  \texttt{ctr.mat} argument:
\begin{Schunk}
\begin{Sinput}
 round( ci.exp( msmr, ctr.mat=rbind(M=c(1,0),W=c(0,1),'M/F'=c(1,-1)) ), 2 )
\end{Sinput}
\begin{Soutput}
    exp(Est.) 2.5% 97.5%
M        1.69 1.60  1.78
W        1.54 1.46  1.63
M/F      1.09 1.01  1.18
\end{Soutput}
\end{Schunk}

\end{enumerate}

\subsection*{Age-specific mortality}
 
\begin{enumerate}[resume]
  

\item We now use this dataset to estimate models with age-specific
  mortality curves for men and women separately, using splines (the
  function \texttt{gam}, using \texttt{s} from the \texttt{mgcv}
  package).
\begin{Schunk}
\begin{Sinput}
 r.m <- gam( (lex.Xst=="Dead") ~ s(A,k=20) + offset( log(lex.dur) ),
             family = poisson,
               data = subset( SL, sex=="M" ) )
 r.f <- update( r.m, data = subset( SL, sex=="F" ) )
 gam.check( r.m )
\end{Sinput}
\begin{Soutput}
Method: UBRE   Optimizer: outer newton
full convergence after 3 iterations.
Gradient range [7.812553e-10,7.812553e-10]
(score -0.8040744 & scale 1).
Hessian positive definite, eigenvalue range [1.064907e-05,1.064907e-05].
Model rank =  20 / 20 

Basis dimension (k) checking results. Low p-value (k-index<1) may
indicate that k is too low, especially if edf is close to k'.

        k'   edf k-index p-value
s(A) 19.00  4.66    0.94    0.87
\end{Soutput}
\begin{Sinput}
 gam.check( r.f )
\end{Sinput}
\begin{Soutput}
Method: UBRE   Optimizer: outer newton
full convergence after 1 iteration.
Gradient range [-5.035753e-08,-5.035753e-08]
(score -0.8232311 & scale 1).
Hessian positive definite, eigenvalue range [2.005875e-05,2.005875e-05].
Model rank =  20 / 20 

Basis dimension (k) checking results. Low p-value (k-index<1) may
indicate that k is too low, especially if edf is close to k'.

       k'  edf k-index p-value
s(A) 19.0  2.6    0.92   0.025
\end{Soutput}
\end{Schunk}
Here we are modeling the follow-up (events
(\texttt{(lex.Xst=="Dead")}) and person-years (\texttt{lex.dur}) ) as
a non-linear function of age --- represented by the penalized spline
function \texttt{s}.



\item From these objects we could get the estimated log-rates by using
  \texttt{predict}, by supplying a data frame of values for the
  variables used as predictors in the model. These will be values of
  age --- the ages where we want to see the predicted rates \emph{and}
  \texttt{lex.dur}.
  
  The default \texttt{predict.gam} function is a bit clunky as it
  gives the prediction and the standard errors of these in two different
  elements of a list, so in \texttt{Epi} there is a wrapper function
  \texttt{ci.pred} that uses this and computes predicted rates and
  confidence limits for these, which is usually what is needed. 
  
  Note that interms of prediction \texttt{lex.dur} is a covariate too;
  by setting this to 1000 throughout the data frame \texttt{nd} we get
  the rates in units of deaths per 1000 PY\footnote{Note however this is only
  the case if the offset is specified in the model formula. If the offset is
  specified as an argument (\texttt{offset = log(lex.dur)}), then the
  value of \texttt{lex.dur} in the dataframe will be ignored
  (equivaent of setting \texttt{lex.dur} to 1 in the \texttt{newdata}
  argument to \texttt{ci.pred}.)}
\begin{Schunk}
\begin{Sinput}
 nd <-  data.frame( A = seq(20,90,0.5),
              lex.dur = 1000)
 p.m <- ci.pred( r.m, newdata = nd )
 p.f <- ci.pred( r.f, newdata = nd )
 str( p.m )
\end{Sinput}
\begin{Soutput}
 num [1:141, 1:3] 1.32 1.37 1.42 1.47 1.52 ...
 - attr(*, "dimnames")=List of 2
  ..$ : chr [1:141] "1" "2" "3" "4" ...
  ..$ : chr [1:3] "Estimate" "2.5%" "97.5%"
\end{Soutput}
\end{Schunk}
  
  
\item We can then plot the predicted rates for men and women together
  using \texttt{matplot}:
\begin{Schunk}
\begin{Sinput}
 matplot( nd$A, cbind(p.m,p.f),
          type="l", col=rep(c("blue","red"),each=3), lwd=c(3,1,1), lty=1,
          log="y", xlab="Age", ylab="Mortality of DM ptt per 1000 PY")
\end{Sinput}
\end{Schunk}
An alternative is to use \texttt{matshade} that gives confidence
limist as shaded areas
\begin{Schunk}
\begin{Sinput}
 matshade( nd$A, cbind(p.m,p.f), plot=TRUE,
           col=c("blue","red"), lty=1,
           log="y", xlab="Age", ylab="Mortality of DM ptt per 1000 PY")
\end{Sinput}
\end{Schunk}
\insfig{A-rates}{0.7}{Age-specific mortality rates for Danish diabetes
  patients as estimated from a \textrm{\tt gam} model with only
  age. Blue: men, red: women.}

   Not surprisingly, the uncertainty on the rates is largest among the
   youngest where the no. of deths is smallest. 

\end{enumerate}

\subsection*{Period and duration effects}

\begin{enumerate}[resume]

  
\item We model the mortality rates among diabetes patients also
  including current date and duration of diabetes. Note that we for
  later prediction purposes put the offset in the model formula.
\begin{Schunk}
\begin{Sinput}
 Mcr <- gam( (lex.Xst=="Dead") ~ s(   A, bs="cr", k=10 ) +
                                 s(   P, bs="cr", k=10 ) +
                                 s( dur, bs="cr", k=10 ) + offset(log(lex.dur)),
             family = poisson,
               data = subset( SL, sex=="M" ) )
 summary( Mcr )
\end{Sinput}
\begin{Soutput}
Family: poisson 
Link function: log 

Formula:
(lex.Xst == "Dead") ~ s(A, bs = "cr", k = 10) + s(P, bs = "cr", 
    k = 10) + s(dur, bs = "cr", k = 10) + offset(log(lex.dur))

Parametric coefficients:
            Estimate Std. Error z value Pr(>|z|)
(Intercept) -3.54074    0.04938   -71.7   <2e-16

Approximate significance of smooth terms:
         edf Ref.df  Chi.sq  p-value
s(A)   3.646  4.518 1013.20  < 2e-16
s(P)   1.024  1.047   17.58 3.40e-05
s(dur) 7.586  8.384   74.46 1.87e-12

R-sq.(adj) =  -0.0255   Deviance explained = 9.87%
UBRE = -0.8054  Scale est. = 1         n = 60347
\end{Soutput}
\begin{Sinput}
 Fcr <- update( Mcr, data = subset( SL, sex=="F" ) )
 summary( Fcr )
\end{Sinput}
\begin{Soutput}
Family: poisson 
Link function: log 

Formula:
(lex.Xst == "Dead") ~ s(A, bs = "cr", k = 10) + s(P, bs = "cr", 
    k = 10) + s(dur, bs = "cr", k = 10) + offset(log(lex.dur))

Parametric coefficients:
            Estimate Std. Error z value Pr(>|z|)
(Intercept) -3.78479    0.05808  -65.17   <2e-16

Approximate significance of smooth terms:
         edf Ref.df Chi.sq  p-value
s(A)   2.668  3.368 988.51  < 2e-16
s(P)   1.887  2.370  20.04 0.000123
s(dur) 5.972  6.971  38.95 2.17e-06

R-sq.(adj) =  -0.0229   Deviance explained = 11.1%
UBRE = -0.82405  Scale est. = 1         n = 58126
\end{Soutput}
\begin{Sinput}
 gam.check( Mcr )
\end{Sinput}
\begin{Soutput}
Method: UBRE   Optimizer: outer newton
full convergence after 3 iterations.
Gradient range [-7.482163e-07,2.324787e-07]
(score -0.805401 & scale 1).
Hessian positive definite, eigenvalue range [7.229476e-07,1.904545e-05].
Model rank =  28 / 28 

Basis dimension (k) checking results. Low p-value (k-index<1) may
indicate that k is too low, especially if edf is close to k'.

         k'  edf k-index p-value
s(A)   9.00 3.65    0.90    0.45
s(P)   9.00 1.02    0.90    0.47
s(dur) 9.00 7.59    0.85  <2e-16
\end{Soutput}
\begin{Sinput}
 gam.check( Fcr )
\end{Sinput}
\begin{Soutput}
Method: UBRE   Optimizer: outer newton
full convergence after 3 iterations.
Gradient range [-1.401872e-08,1.022359e-08]
(score -0.8240482 & scale 1).
Hessian positive definite, eigenvalue range [1.061413e-05,2.307556e-05].
Model rank =  28 / 28 

Basis dimension (k) checking results. Low p-value (k-index<1) may
indicate that k is too low, especially if edf is close to k'.

         k'  edf k-index p-value
s(A)   9.00 2.67    0.90   0.005
s(P)   9.00 1.89    0.93   0.275
s(dur) 9.00 5.97    0.95   0.805
\end{Soutput}
\end{Schunk}
For the male rates the \texttt{edf} (effective degrees of freedom) is
quite close to the \texttt{k}, but as wee shall see no need to more
detailed modeling.


\item We can now plot the estimated effects for men and women:
\begin{Schunk}
\begin{Sinput}
 par( mfrow=c(2,3) )
 plot( Mcr, ylim=c(-3,3), col="blue" )
 plot( Fcr, ylim=c(-3,3), col="red" )
\end{Sinput}
\end{Schunk}
\insfig{plgam-default}{1.0}{Plot of the estimated smooth terms for men
(top) and women (bottom). Clearly it seems that the duration effects
are somewhat over-modeled.}
  
  
\item Not surprisingly, these models fit substantially better than the
  model with only age as we can see from this comparison:
\begin{Schunk}
\begin{Sinput}
 anova( Mcr, r.m, test="Chisq" )
\end{Sinput}
\begin{Soutput}
Analysis of Deviance Table

Model 1: (lex.Xst == "Dead") ~ s(A, bs = "cr", k = 10) + s(P, bs = "cr", 
    k = 10) + s(dur, bs = "cr", k = 10) + offset(log(lex.dur))
Model 2: (lex.Xst == "Dead") ~ s(A, k = 20) + offset(log(lex.dur))
  Resid. Df Resid. Dev      Df Deviance  Pr(>Chi)
1     60332      11717                           
2     60340      11812 -8.0248  -95.251 < 2.2e-16
\end{Soutput}
\begin{Sinput}
 anova( Fcr, r.f, test="Chisq" )
\end{Sinput}
\begin{Soutput}
Analysis of Deviance Table

Model 1: (lex.Xst == "Dead") ~ s(A, bs = "cr", k = 10) + s(P, bs = "cr", 
    k = 10) + s(dur, bs = "cr", k = 10) + offset(log(lex.dur))
Model 2: (lex.Xst == "Dead") ~ s(A, k = 20) + offset(log(lex.dur))
  Resid. Df Resid. Dev      Df Deviance  Pr(>Chi)
1     58112      10204                           
2     58122      10268 -9.3754   -63.35 4.457e-10
\end{Soutput}
\end{Schunk}





\item Since the fitted model has three time-scales: current age,
  current date and current duration of diabetes, so the effects that
  we see from the \texttt{plot.gam} are not really interpretable; they
  are (as in any kind of multiple regressions) to be interpreted as
  ``all else equal'' which they are not; the three time scales
  advance simultaneously at the same pace.

  The reporting would therefore more naturally be \emph{only} on one
  time scale, showing the mortality for persons diagnosed in
  different ages in a given year.

  This is most easily done using the \texttt{ci.pred} function with
  the \texttt{newdata=} argument. So a person diagnosed in age 50 in
  1995 will have a mortality measured in cases per 1000 PY as:
\begin{Schunk}
\begin{Sinput}
 pts <- seq(0,17,0.5)
 nd <- data.frame( A =   50+pts,
                   P = 1995+pts,
                 dur =      pts,
             lex.dur = 1000 )
 head( cbind( nd$A, ci.pred( Mcr, newdata=nd ) ) )
\end{Sinput}
\begin{Soutput}
       Estimate     2.5%    97.5%
1 50.0 29.75883 22.91502 38.64663
2 50.5 19.60347 15.38961 24.97114
3 51.0 15.71789 12.31884 20.05482
4 51.5 15.61155 12.22466 19.93677
5 52.0 16.18946 12.79954 20.47721
6 52.5 16.47424 12.94892 20.95933
\end{Soutput}
\end{Schunk}
Note, that we used \texttt{+ offset(log(lex.dur))} as the offset
specification in the model, the value of \texttt{lex.dur} in
\texttt{nd} will be honoured and prediction be made for the scale
chosen, in the model specification; so in this case as events per 1000
PY (since \texttt{lex.dur} is in units of single person-years).

Since there is no duration beyond 18 years in the dataset we only make
predictions for 12 years of duration, and do it for persons diagnosed
in 1995 and 2005 --- the latter is quite dubious too because we are
extrapolating calendar time trends way beyond data.
  
We form matrices of predictions with confidence intervals, that we
will plot in the same frame:
\begin{Schunk}
\begin{Sinput}
 pts <- seq(0,12,0.1)
 plot( NA, xlim = c(50,85), ylim = c(5,400), log="y",
           xlab="Age", ylab="Mortality rate for DM patients" )
 for( ip in c(1995,2005) )
 for( ia in c(50,60,70) )
    {
 nd <- data.frame( A=ia+pts,
                   P=ip+pts,
                 dur=   pts,
             lex.dur=1000 )
 matshade( nd$A, ci.pred( Mcr, nd ), col="blue" )
 matshade( nd$A, ci.pred( Fcr, nd ), col="red" )
    }
\end{Sinput}
\end{Schunk}
\insfig{rates}{1.0}{Mortality rates for diabetes patients diagnosed
  1995 and 2005 in ages 50, 60 and 70; as estimated by penalized
  splines. Men blue, women red.}


\item From figure \ref{fig:rates} it seems that the duration effect is
dramatically over-modeled, so we refit constraining the d.f. to 5 (note
that this choice is essentially an arbitray choice) and redo the whole
thing again:
\begin{Schunk}
\begin{Sinput}
 Mcr <- gam( (lex.Xst=="Dead") ~ s(   A, bs="cr", k=10 ) +
                                 s(   P, bs="cr", k=10 ) +
                                 s( dur, bs="cr", k=5 ) +
                                 offset( log(lex.dur) ),
             family = poisson,
               data = subset(SL,sex=="M") )
 Fcr <- update( Mcr, 
               data = subset(SL,sex=="F") )
 gam.check( Mcr )
\end{Sinput}
\begin{Soutput}
Method: UBRE   Optimizer: outer newton
full convergence after 4 iterations.
Gradient range [-2.87389e-07,5.520002e-08]
(score -0.805213 & scale 1).
Hessian positive definite, eigenvalue range [2.83738e-07,1.060921e-05].
Model rank =  23 / 23 

Basis dimension (k) checking results. Low p-value (k-index<1) may
indicate that k is too low, especially if edf is close to k'.

         k'  edf k-index p-value
s(A)   9.00 3.70    0.93    0.76
s(P)   9.00 1.01    0.92    0.27
s(dur) 4.00 3.93    0.90    0.06
\end{Soutput}
\begin{Sinput}
 gam.check( Fcr )
\end{Sinput}
\begin{Soutput}
Method: UBRE   Optimizer: outer newton
full convergence after 3 iterations.
Gradient range [1.704809e-09,4.79408e-08]
(score -0.8240014 & scale 1).
Hessian positive definite, eigenvalue range [4.823934e-06,1.936723e-05].
Model rank =  23 / 23 

Basis dimension (k) checking results. Low p-value (k-index<1) may
indicate that k is too low, especially if edf is close to k'.

         k'  edf k-index p-value
s(A)   9.00 2.65    0.94     0.2
s(P)   9.00 1.92    0.96     0.9
s(dur) 4.00 3.81    0.94     0.3
\end{Soutput}
\end{Schunk}
Thus it seems that \texttt{k=5} is a bit \texttt{under} modeling the
duration effect.

We also add the rate-ratio between men and women using the
\texttt{ci.ratio} function:
\begin{Schunk}
\begin{Sinput}
 plot( NA, xlim = c(50,80), ylim = c(0.9,100), log="y",
           xlab="Age", ylab="Mortality rate for DM patients" )
 abline( v = c(50,55,60,65,70), col=gray(0.8) )
 # for( ip in c(1995,2005) )
 ip <- 2005
 for( ia in c(50,55,60,65,70) )
    {
 nd <- data.frame( A=ia+pts,
                   P=ip+pts,
                 dur=   pts,
             lex.dur=1000 )
 matshade( nd$A, rm <- ci.pred( Mcr, nd ), col="blue", lwd=2 )
 matshade( nd$A, rf <- ci.pred( Fcr, nd ), col="red" , lwd=2 )
 matshade( nd$A, ci.ratio( rm, rf ), lwd=2 )
    } 
 abline( h=1, lty="55" )
\end{Sinput}
\end{Schunk}
\insfig{rates-5}{1.0}{Mortality rates for diabetes patients diagnosed
  1995 and 2005 in ages 50, 55, 60, 65 and 70; as estimated by penalized
  splines. Note that we did this for year of diagnosis 2005 alone, but
  for a longer duration period. Men blue, women red, M/W rate ratio gray.}
From figure \ref{fig:rates-dur} we see that there is an incraesed
mortality in the first few years afer diagnosis (this is a clinical
artifact) and little indication that age \emph{at} diagnosis has any effect.
(This can of course be tried explicitly).

Moreover it is pretty clear that the M/W mortality rate ratio is
constant across age and duration.

\end{enumerate}

\subsection{SMR modeling}

\begin{enumerate}[resume]

  
  

\item We can treat SMR exactly as mortality rates by including the log
  expected numbers instead of the log person-years as offset, again using
  separate models for men and women. 
 
  We exclude those records where no deaths in the population occur
  (that is where the rate is 0) --- you could say that this correspond
  to parts of the data where no follow-up on the population mortality
  scale is available. The rest is essentially just a repeat of the
  analyses for mortality rates:
\begin{Schunk}
\begin{Sinput}
 SLr <- subset( SLr, E>0 )
 Msmr <- gam( (lex.Xst=="Dead") ~ s(   A, bs="cr", k=10 ) +
                                  s(   P, bs="cr", k=10 ) +
                                  s( dur, bs="cr", k=5 ),
               offset = log( E ),
               family = poisson,
                 data = subset( SLr, sex=="M" ) )
 Fsmr <- update( Msmr, 
                 data = subset( SLr, sex=="F" ) )
 summary( Msmr )
\end{Sinput}
\begin{Soutput}
Family: poisson 
Link function: log 

Formula:
(lex.Xst == "Dead") ~ s(A, bs = "cr", k = 10) + s(P, bs = "cr", 
    k = 10) + s(dur, bs = "cr", k = 5)

Parametric coefficients:
            Estimate Std. Error z value Pr(>|z|)
(Intercept)  0.77299    0.04089   18.91   <2e-16

Approximate significance of smooth terms:
         edf Ref.df Chi.sq  p-value
s(A)   1.022  1.044 58.357 2.82e-14
s(P)   1.009  1.019  1.972    0.163
s(dur) 3.925  3.996 55.820 3.17e-11

R-sq.(adj) =  -0.0259   Deviance explained = 0.992%
UBRE = -0.80531  Scale est. = 1         n = 60333
\end{Soutput}
\begin{Sinput}
 summary( Fsmr )
\end{Sinput}
\begin{Soutput}
Family: poisson 
Link function: log 

Formula:
(lex.Xst == "Dead") ~ s(A, bs = "cr", k = 10) + s(P, bs = "cr", 
    k = 10) + s(dur, bs = "cr", k = 5)

Parametric coefficients:
            Estimate Std. Error z value Pr(>|z|)
(Intercept)  0.75404    0.05384   14.01   <2e-16

Approximate significance of smooth terms:
         edf Ref.df Chi.sq  p-value
s(A)   1.699  2.149 56.274 3.95e-12
s(P)   1.837  2.306  7.508   0.0375
s(dur) 3.818  3.977 35.807 1.57e-06

R-sq.(adj) =  -0.0228   Deviance explained = 0.947%
UBRE = -0.82414  Scale est. = 1         n = 58099
\end{Soutput}
\begin{Sinput}
 par( mfrow=c(2,3) )
 plot( Msmr, ylim=c(-1,2), col="blue" )
 plot( Fsmr, ylim=c(-1,2), col="red" )
\end{Sinput}
\end{Schunk}
\insfig{SMReff}{1.0}{Estimated effects of age, calendar time and
  duration on SMR --- top men, bottom women}


\item We then compute the predicted SMRs from the models for men and
  women diagnosed in ages 50, 60 and 70 in 1995 and 2005,
  respectively, and show them in plots side by side. We are going to
  make this type of plot for other models (well, pairs, for men and
  women) so we wrap it in a function:
\begin{Schunk}
\begin{Sinput}
 plot( NA, xlim = c(50,80), ylim = c(0.8,5), log="y",
           xlab="Age", ylab="SMR relative to total population" )
 abline( v = c(50,55,60,65,70), col=gray(0.8) )
 # for( ip in c(1995,2005) )
 ip <- 2005
 for( ia in c(50,55,60,65,70) )
    {
 nd <- data.frame( A=ia+pts,
                   P=ip+pts,
                 dur=   pts,
                   E=1 )
 matshade( nd$A, rm <- ci.pred( Msmr, nd ), col="blue", lwd=2 )
 matshade( nd$A, rf <- ci.pred( Fsmr, nd ), col="red" , lwd=2 )
 matshade( nd$A, ci.ratio( rm, rf ), lwd=2, col=gray(0.5) )
    } 
 abline( h=1, lty="55" )
\end{Sinput}
\end{Schunk}
\insfig{SMRsm}{1.0}{Mortality rates for diabetes patients diagnosed
  1995 and 2005 in ages 50, 60 and 70; as estimated by penalized
  splines. Men blue, women red.}

From figure \ref{fig:SMRsm} we see that like for mortality there is a
clear peak at diagnosis and flattening after approximately 2
years. But also that the duration is possibly over-modeled. Finally
there is no indication that the SMR is different for men and women.
  
  
\item It would be natural to simplify the model to one with a non-linear
  effect of duration and linear effects of age at diagnosis and
  calendar time, and moreover to squeeze the number of d.f. for the
  non-linear smooth term for duration:
\begin{Schunk}
\begin{Sinput}
 Asmr <- gam( (lex.Xst=="Dead") ~ sex +
                                  sex:I(A-60) + 
                                  sex:I(P-2000) +
                                  s( dur, k=5 ) +
                                  offset( log(E) ),
              family = poisson,
                data = SLr )
 summary( Asmr )
\end{Sinput}
\begin{Soutput}
Family: poisson 
Link function: log 

Formula:
(lex.Xst == "Dead") ~ sex + sex:I(A - 60) + sex:I(P - 2000) + 
    s(dur, k = 5) + offset(log(E))

Parametric coefficients:
                  Estimate Std. Error z value Pr(>|z|)
(Intercept)       0.868923   0.055576  15.635  < 2e-16
sexF              0.035465   0.085183   0.416   0.6772
sexM:I(A - 60)   -0.018857   0.002421  -7.789 6.73e-15
sexF:I(A - 60)   -0.019701   0.002696  -7.306 2.74e-13
sexM:I(P - 2000) -0.013296   0.007936  -1.675   0.0939
sexF:I(P - 2000) -0.018563   0.008499  -2.184   0.0289

Approximate significance of smooth terms:
         edf Ref.df Chi.sq  p-value
s(dur) 3.856  3.987  60.13 1.57e-12

R-sq.(adj) =  -0.0242   Deviance explained = 0.865%
UBRE = -0.8144  Scale est. = 1         n = 118432
\end{Soutput}
\begin{Sinput}
 gam.check( Asmr )
\end{Sinput}
\begin{Soutput}
Method: UBRE   Optimizer: outer newton
full convergence after 4 iterations.
Gradient range [6.36467e-07,6.36467e-07]
(score -0.8143976 & scale 1).
Hessian positive definite, eigenvalue range [3.017067e-06,3.017067e-06].
Model rank =  10 / 10 

Basis dimension (k) checking results. Low p-value (k-index<1) may
indicate that k is too low, especially if edf is close to k'.

         k'  edf k-index p-value
s(dur) 4.00 3.86    0.92    0.64
\end{Soutput}
\begin{Sinput}
 round( ( ci.exp(Asmr,subset="sex")-1 )*100, 1 )
\end{Sinput}
\begin{Soutput}
                 exp(Est.)  2.5% 97.5%
sexF                   3.6 -12.3  22.4
sexM:I(A - 60)        -1.9  -2.3  -1.4
sexF:I(A - 60)        -2.0  -2.5  -1.4
sexM:I(P - 2000)      -1.3  -2.8   0.2
sexF:I(P - 2000)      -1.8  -3.5  -0.2
\end{Soutput}
\end{Schunk}
Thus the decrease in SMR per year of age is about 2\% / year for both
sexes and 1.3\% per calendar year for men, but 1.8\% per year for women.

The estimate of sex indicates that the SMR for 60 year old women in
2000 at duration 0 of DM is about 3.6\% larger than that of men, but nowhere near
significantly so.


\item We can use the previous code to show the predicted mortality
  under this model. 
\begin{Schunk}
\begin{Sinput}
 plot( NA, xlim = c(50,80), ylim = c(0.8,5), log="y",
           xlab="Age", ylab="SMR relative to total population" )
 abline( v = c(50,55,60,65,70), col=gray(0.8) )
 # for( ip in c(1995,2005) )
 ip <- 2005
 for( ia in c(50,55,60,65,70) )
    {
 nd <- data.frame( A=ia+pts,
                   P=ip+pts,
                 dur=   pts,
                   E=1 )
 matshade( nd$A, rm <- ci.pred( Asmr, cbind(nd,sex="M") ), col="blue", lwd=2 )
 matshade( nd$A, rf <- ci.pred( Asmr, cbind(nd,sex="F") ), col="red" , lwd=2 )
 matshade( nd$A, ci.ratio( rm, rf ), lwd=2, col=gray(0.5) )
    } 
 abline( h=1, lty="55" )
\end{Sinput}
\end{Schunk}
\insfig{SMRsl}{1.0}{Mortality rates for diabetes patients diagnosed
  1995 and 2005 in ages 50, 60 and 70; as estimated by penalized
  splines. Men blue, women red.}
We see that there is absolutely no indication of difference between
men and women, but also that the estimated effect of duraion is not
exactly credible.


\item If we deem the curves non-credible, we may resort to a brutal
  parametric assumption without any penalization of curvature
  involved. If we choose a natural spline for the duration with knots
  at 0,1,3,6 years we get a model with 3 parameters, try:
\begin{Schunk}
\begin{Sinput}
 dim( Ns(SLr$dur, knots=c(0,1,4,8) ) )
\end{Sinput}
\begin{Soutput}
[1] 118432      3
\end{Soutput}
\end{Schunk}
  Now fit the same model as above using this:
\begin{Schunk}
\begin{Sinput}
 SMRglm <- glm( (lex.Xst=="Dead") ~ I(A-60) + 
                                    I(P-2000) + 
                                    Ns( dur, knots=c(0,1,4,8) ) +
                                    offset( log(E) ),
                family = poisson,
                  data = SLr )
\end{Sinput}
\end{Schunk}
Then we can plot again:
\begin{Schunk}
\begin{Sinput}
 plot( NA, xlim = c(50,80), ylim = c(0.8,5), log="y",
           xlab="Age", ylab="SMR relative to total population" )
 abline( v = c(50,55,60,65,70), col=gray(0.8) )
 # for( ip in c(1995,2005) )
 ip <- 2005
 for( ia in c(50,55,60,65,70) )
    {
 nd <- data.frame( A=ia+pts,
                   P=ip+pts,
                 dur=   pts,
                   E=1 )
 matshade( nd$A, ci.pred( SMRglm, nd ), lwd=2 )
    } 
 abline( h=1, lty="55" )
\end{Sinput}
\end{Schunk}
\insfig{SMRsp}{1.0}{Mortality rates for diabetes patients diagnosed
  1995 and 2005 in ages 50, 60 and 70; as estimated by penalized
  splines, no difference between men and women assumed} Form figure
\ref{fig:SMRsp} it appears that the SMR drops by a factor 2 during the
first 2 years, increases a bit and after that decreases by age. But
the general pattern is that after som 5 years the SMR no longer depend
on the age at diagnosis --- but the evidence for this is pretty thin.

\end{enumerate}
