%% \SweaveOpts{results=verbatim,keep.source=TRUE,include=FALSE,eval=FALSE,prefix.string=./graph/renal}
%% <<echo=FALSE,eval=TRUE>>=
%% options( width=90,
%%          prompt=" ", continue=" ",
%%          SweaveHooks=list( fig=function()
%%          par(mar=c(3,3,1,1),mgp=c(3,1,0)/1.6,las=1,bty="n") ) )
%% @ %
%% \renewcommand{\rwpre}{./graph/renal}
%% \section{Time-dependent variables and multiple states}


\renewcommand{\rwpre}{./graph/renal}
\section{Time-dependent variables and multiple states}

%% The following practical exercise is based on the data from paper:
%% \begin{description}
%% \item
%% P Hovind, L Tarnow, P Rossing, B Carstensen, and HH Parving:
%% Improved survival in patients obtaining remission of nephrotic range
%%   albuminuria in diabetic nephropathy.
%% \textit{Kidney Int}, \textbf{66}(3):1180--1186, Sept 2004.
%% \end{description}
%% You can find a \texttt{.pdf}-version of the paper here:
%% \url{http://BendixCarstensen.com/~bxc/AdvCoh/papers/Hovind.2004.pdf}

%% \subsection{The renal failure dataset}

%% The dataset \texttt{renal.dta} contains data on follow up of 125
%% patients from Steno Diabetes Center. They enter the study when they
%% are diagnosed with nephrotic range albuminuria (NRA). This is a
%% condition where the levels of albumin in the urine is exceeds a
%% certain level as a sign of kidney disease. The levels may however drop
%% as a consequence of treatment, this is called remission. Patients exit
%% the study at death or kidney failure (dialysis or transplant).
%% \begin{table}[htbp]
%%   \centering
%%   \caption{\it Variables in {\rm \texttt{renal.dta}}.}
%% % \small
%% % \renewcommand{\arraystretch}{0.95}
%% \begin{tabular}{@{\extracolsep{1ex}}rl}\\
%% \toprule
%% \verb+id+    & Patient id \\
%% \verb+sex+   & 1=male, 2=female \\
%% \verb+dob+   & Date of birth \\
%% \verb+doe+   & Date of entry into the study (2.5 years after NRA) \\
%% \verb+dor+   & Date of remission. Missing if no remission has occurred \\
%% \verb+dox+    & Date of exit from study \\
%% \verb+event+  & Exit status: 1,2,3=event (death, ESRD), 0=censored \\
%% \bottomrule
%%   \end{tabular}
%%   \label{tab:fincol}
%% \renewcommand{\arraystretch}{1.0}
%% \end{table}

\subsection{The renal failure dataset}

%% \begin{enumerate}
\begin{enumerate}

%% \item The dataset is in Stata-format, so you must read the dataset
%%   using \texttt{read.dta} from the \texttt{foreign} package (which is
%%   part of the standard \R-distribution). At the same time, convert
%%   \texttt{sex} to a proper factor. Choose where to read the dataset.
%% <<>>=
%% library( Epi ) ; clear()
%% library( foreign )
%% renal <- read.dta(
%%          "https://raw.githubusercontent.com/SPE-R/SPE/master/pracs/data/renal.dta")
%% # renal <- read.dta( "http://BendixCarstensen.com/SPE/data/renal.dta" )
%% # renal <- read.dta( "./data/renal.dta" )
%% renal$sex <- factor( renal$sex, labels=c("M","F") )
%% head( renal )
%% @ %

\item The dataset is in Stata-format, so we read the dataset
  using \texttt{read.dta} from the \texttt{foreign} package (which is
  part of the standard \R-distribution):
\begin{Schunk}
\begin{Sinput}
 library( Epi )
 library( foreign )
 clear()
 renal <- read.dta(
          "https://raw.githubusercontent.com/SPE-R/SPE/master/pracs/data/renal.dta")
 # renal <- read.dta( "http://BendixCarstensen.com/SPE/data/renal.dta" )
 # renal <- read.dta( "./data/renal.dta" )
 renal$sex <- factor( renal$sex, labels=c("M","F") )
 head(renal)
\end{Sinput}
\begin{Soutput}
  id sex      dob      doe      dor      dox event
1 17   M 1967.944 1996.013       NA 1997.094     2
2 26   F 1959.306 1989.535 1989.814 1996.136     1
3 27   F 1962.014 1987.846       NA 1993.239     3
4 33   M 1950.747 1995.243 1995.717 2003.993     0
5 42   F 1961.296 1987.884 1996.650 2003.955     0
6 46   F 1952.374 1983.419       NA 1991.484     2
\end{Soutput}
\end{Schunk}

%% \item Use the \texttt{Lexis} function to declare the data as
%%   survival data with age, calendar time and time since entry into
%%   the study as timescales. Label any event $>0$ as ``ESRD'',
%%   i.e. renal death (death of kidney (transplant or dialysis), or
%%   person).
%%   Note that you must make sure that the ``alive'' state (here
%%   \texttt{NRA}) is the first, as \texttt{Lexis} assumes that
%%   everyone starts in this state (unless of course
%%   \texttt{entry.status} is specified):
%% <<>>=
%% Lr <- Lexis( entry = list( per=doe,
%%                            age=doe-dob,
%%                            tfi=0 ),
%%               exit = list( per=dox ),
%%        exit.status = factor( event>0, labels=c("NRA","ESRD") ),
%%               data = renal )
%% str( Lr )
%% summary( Lr )
%% @ %
%% Make sure you know what the variables in \texttt{Lr} stand for.

\item We use the \texttt{Lexis} function to declare the data as
  survival data with age, calendar time and time since entry into the
  study as timescales. Note that any coding of event $>0$ will be
  labeled ``ESRD'', i.e. renal death (death of kidney (transplant or
  dialysis), or person).

  Note that you must make sure that the ``alive'' state (here
  \texttt{NRA}) is the first, as \texttt{Lexis} assumes that everyone
  starts in this state (unless of course \texttt{entry.status} is
  specified):
\begin{Schunk}
\begin{Sinput}
 Lr <- Lexis( entry = list( per = doe,
                            age = doe-dob,
                            tfi = 0 ),
               exit = list( per = dox ),
        exit.status = factor( event>0, labels=c("NRA","ESRD") ),
               data = renal )
\end{Sinput}
\begin{Soutput}
NOTE: entry.status has been set to "NRA" for all.
\end{Soutput}
\begin{Sinput}
 str( Lr )
\end{Sinput}
\begin{Soutput}
Classes 'Lexis' and 'data.frame':	125 obs. of  14 variables:
 $ per    : num  1996 1990 1988 1995 1988 ...
 $ age    : num  28.1 30.2 25.8 44.5 26.6 ...
 $ tfi    : num  0 0 0 0 0 0 0 0 0 0 ...
 $ lex.dur: num  1.08 6.6 5.39 8.75 16.07 ...
 $ lex.Cst: Factor w/ 2 levels "NRA","ESRD": 1 1 1 1 1 1 1 1 1 1 ...
 $ lex.Xst: Factor w/ 2 levels "NRA","ESRD": 2 2 2 1 1 2 2 1 2 1 ...
 $ lex.id : int  1 2 3 4 5 6 7 8 9 10 ...
 $ id     : num  17 26 27 33 42 46 47 55 62 64 ...
 $ sex    : Factor w/ 2 levels "M","F": 1 2 2 1 2 2 1 1 2 1 ...
 $ dob    : num  1968 1959 1962 1951 1961 ...
 $ doe    : num  1996 1990 1988 1995 1988 ...
 $ dor    : num  NA 1990 NA 1996 1997 ...
 $ dox    : num  1997 1996 1993 2004 2004 ...
 $ event  : num  2 1 3 0 0 2 1 0 2 0 ...
 - attr(*, "time.scales")= chr [1:3] "per" "age" "tfi"
 - attr(*, "time.since")= chr [1:3] "" "" ""
 - attr(*, "breaks")=List of 3
  ..$ per: NULL
  ..$ age: NULL
  ..$ tfi: NULL
\end{Soutput}
\begin{Sinput}
 summary( Lr )
\end{Sinput}
\begin{Soutput}
Transitions:
     To
From  NRA ESRD  Records:  Events: Risk time:  Persons:
  NRA  48   77       125       77    1084.67       125
\end{Soutput}
\end{Schunk}

%% \item Visualize the follow-up in a Lexis-diagram, by using the
%%   \texttt{plot} method for \texttt{Lexis} objects.
%% <<Lexis-ups, fig=TRUE>>=
%% plot( Lr, col="black", lwd=3 )
%% subset( Lr, age<0 )
%% @ %
%% What is wrong here? List the data for the person with negative entry age.

\item We can visualize the follow-up in a Lexis-diagram, using the
  \texttt{plot} method for \texttt{Lexis} objects.
\begin{Schunk}
\begin{Sinput}
 plot( Lr, col="black", lwd=3 )
 subset( Lr, age<0 )
\end{Sinput}
\begin{Soutput}
 lex.id     per    age tfi lex.dur lex.Cst lex.Xst  id sex      dob      doe dor      dox
     88 1989.34 -38.81   0     3.5     NRA    ESRD 586   M 2028.155 1989.343  NA 1992.839
 event
     1
\end{Soutput}
\end{Schunk}
The result is the left hand plot in figure \ref{fig:Lexis-ups},
and we see a person entering at a negative age, clearly because he is
born way out in the future.

%% \item Correct the data and make a new plot, for example by:
%% <<Lexis-def,fig=TRUE>>=
%% Lr <- transform( Lr, dob = ifelse( dob>2000, dob-100, dob ),
%%                      age = ifelse( dob>2000, age+100, age ) )
%% subset( Lr, id==586 )
%% plot( Lr, col="black", lwd=3 )
%% @ %

\item So we correct the data and make the correct plot, as seen in the right
   hand plot in figure \ref{fig:Lexis-ups}:
\begin{Schunk}
\begin{Sinput}
 Lr <- transform( Lr, dob = ifelse( dob>2000, dob-100, dob ),
                      age = ifelse( dob>2000, age+100, age ) )
 subset( Lr, id==586 )
\end{Sinput}
\begin{Soutput}
 lex.id     per   age tfi lex.dur lex.Cst lex.Xst  id sex      dob      doe dor      dox
     88 1989.34 61.19   0     3.5     NRA    ESRD 586   M 1928.155 1989.343  NA 1992.839
 event
     1
\end{Soutput}
\begin{Sinput}
 plot( Lr, col="black", lwd=3 )
\end{Sinput}
\end{Schunk}

%% \item (\emph{Optional, esoteric}) We can produce a slightly more
%%   fancy Lexis diagram. Note that we have a $x$-axis of 40 years, and
%%   a $y$-axis of 80 years, so when specifying the output file adjust
%%   the \emph{total} width of the plot so that the use of \texttt{mai}
%%   (look up the help page for \texttt{par}) to specify the margins of
%%   the plot so that it leaves a plotting area twice as high as wide. The
%%   \texttt{mai} argument to \texttt{par} gives the margins in inches,
%%   so the total size of the horizontal and vertical margins is 1
%%   inch each, to which we add 80/5 in the height, and 40/5 in the
%%   horizontal direction, each giving exactly 5 years per inch in
%%   physical size.

\item We can produce a slightly more fancy Lexis diagram. Note that we
  have a $x$-axis of 40 years, and a $y$-axis of 80 years, so when
  specifying the output file adjust the \emph{total} width of the plot
  so that the use mai \texttt{mai} to specify the margins of the plot
  leaves a plotting area twice as high as wide. The \texttt{mai}
  argument to \texttt{par} gives the margins in inches, so the total
  size of the horizontal and vertical margins is 1 inch, to which we
  add 80/5 in the height, and 40/5 in the horizontal direction, each
  giving exactly 5 years per inch in physical size.
\begin{Schunk}
\begin{Sinput}
 pdf( "./graph/renal-Lexis-fancy.pdf", height=80/5+1, width=40/5+1 )
 par( mai=c(3,3,1,1)/4, mgp=c(3,1,0)/1.6 )
 plot( Lr, 1:2, col=c("blue","red")[Lr$sex], lwd=3, grid=0:20*5,
       xlab="Calendar time", ylab="Age",
       xlim=c(1970,2010), ylim=c(0,80), xaxs="i", yaxs="i", las=1 )
 dev.off()
\end{Sinput}
\begin{Soutput}
null device 
          1 
\end{Soutput}
\end{Schunk}
\begin{figure}[tb]
  \centering
  \includegraphics[width=0.45\textwidth]{./graph/renal-Lexis-ups}
  \includegraphics[width=0.45\textwidth]{./graph/renal-Lexis-def}
  \caption{\it Default Lexis diagram before and after correction of
    the obvious data outlier.}
  \label{fig:Lexis-ups}
\end{figure}

\begin{figure}[tb]
  \centering
  \includegraphics[height=0.6\textheight,keepaspectratio]{./graph/renal-Lexis-fancy}
  \caption{\it The more fancy version of the Lexis diagram for the
    renal data.}
  \label{fig:Lexis-fancy}
\end{figure}

%% \item Now make a Cox-regression analysis of the enpoint ESRD with
%%   the variables sex and age at entry into the study, using time
%%   since entry to the study as time scale.
%% <<>>=
%% library( survival )
%% mc <- coxph( Surv( lex.dur, lex.Xst=="ESRD" ) ~
%%              I(age/10) + sex, data=Lr )
%% summary( mc )
%% @ %
%% What is the The hazard ratio between males and females?

%% Between two persons who differ 10 years in age at entry?

\item We now do a Cox-regression analysis with the variables sex and age at
  entry into the study, using time since entry to the study as time scale.
\begin{Schunk}
\begin{Sinput}
 library( survival )
 mc <- coxph( Surv( lex.dur, lex.Xst=="ESRD" ) ~
              I(age/10) + sex, data=Lr )
 summary( mc )
\end{Sinput}
\begin{Soutput}
Call:
coxph(formula = Surv(lex.dur, lex.Xst == "ESRD") ~ I(age/10) + 
    sex, data = Lr)

  n= 125, number of events= 77 

             coef exp(coef) se(coef)      z Pr(>|z|)    
I(age/10)  0.5514    1.7357   0.1402  3.932 8.43e-05 ***
sexF      -0.1817    0.8338   0.2727 -0.666    0.505    
---
Signif. codes:  0 '***' 0.001 '**' 0.01 '*' 0.05 '.' 0.1 ' ' 1

          exp(coef) exp(-coef) lower .95 upper .95
I(age/10)    1.7357     0.5761    1.3186     2.285
sexF         0.8338     1.1993    0.4886     1.423

Concordance= 0.612  (se = 0.036 )
Likelihood ratio test= 16.07  on 2 df,   p=3e-04
Wald test            = 16.38  on 2 df,   p=3e-04
Score (logrank) test = 16.77  on 2 df,   p=2e-04
\end{Soutput}
\end{Schunk}
The hazard ratio between males and females is 1.19 (0.70--2.04) (the
inverse of the c.i. for female vs male) and between two persons who
differ 10 years in age at entry it is 1.74 (1.32--2.29).

%% \item The main focus of the paper was to assess whether the occurrence of
%%   remission (return to a lower level of albumin excretion, an
%%   indication of kidney recovery) influences mortality.
%%   ``Remission'' is a time-dependent variable which is initially 0, but
%%   takes the value 1 when remission occurs. In order to handle this, each
%%   person who sees a remission must have two records:
%%   \begin{itemize}
%%   \item One record for the time before remission, where entry is
%%     \texttt{doe}, exit is \texttt{dor}, remission is 0, and event is
%%     0.
%%   \item One record for the time after remission, where entry is
%%     \texttt{dor}, exit is \texttt{dox}, remission is 1, and event is 0
%%     or 1 according to whether the person had an event at \texttt{dox}.
%%   \end{itemize}
%%   This is accomplished using the \texttt{cutLexis} function on the
%%   \texttt{Lexis} object, where we introduce a remission state ``Rem''.
%%   You must declare the ``NRA'' state as a precursor state, i.e. a
%%   state that is \emph{less} severe than ``Rem'' in the sense that a
%%   person who see a remission will stay in the ``Rem'' state unless he
%%   goes to the ``ESRD'' state. Also use \texttt{split.state=TRUE} to
%%   have different ESRD states according to whether a person had had
%%   remission or not prioer to ESRD. The statement to do this is:
%% <<>>=
%% Lc <- cutLexis( Lr, cut = Lr$dor, # where to cut follow up
%%               timescale = "per",  # what timescale are we referring to
%%               new.state = "Rem",  # name of the new state
%%             split.state = TRUE,   # different states sepending on previous
%%        precursor.states = "NRA" ) # which states are less severe
%% summary( Lc )
%% @ %
%% List the records from a few select persons (choose values for
%% \texttt{lex.id}, using for example \texttt{subset( Lc, lex.id \%in\%
%% c(5,7,9) )}, or other numbers).

\item The main focus of the paper was to assess whether the occurrence
  of remission (return to a lower level of albumin excretion, an
  indication of kidney recovery) influences mortality.

  ``Remission'' is a time-dependent variable which is initially 0, but
  takes the value 1 when remission occurs. This is accomplished using
  the \texttt{cutLexis} function on the \texttt{Lexis} object, where
  we introduce a remission state ``Rem''. We declare the ``NRA'' state
  as a precursor state, i.e. a state that is \emph{less} severe than
  ``Rem'' in the sense that a person who see a remission will stay in
  the ``Rem'' state unless he goes to the ``ESRD'' state. The
  statement to do this is:
\begin{Schunk}
\begin{Sinput}
 Lc <- cutLexis( Lr, cut = Lr$dor, # where to cut follow up
               timescale = "per",  # what timescale are we referring to
               new.state = "Rem",  # name of the new state
             split.state = TRUE,   # different states sepending on previous
        precursor.states = "NRA" ) # which states are less severe
 summary( Lc )
\end{Sinput}
\begin{Soutput}
Transitions:
     To
From  NRA Rem ESRD ESRD(Rem)  Records:  Events: Risk time:  Persons:
  NRA  24  29   69         0       122       98     824.77       122
  Rem   0  24    0         8        32        8     259.90        32
  Sum  24  53   69         8       154      106    1084.67       125
\end{Soutput}
\end{Schunk}
Note that we have two different ESRD states depending on whether the
person was in remission or not at the time of ESRD.

To illustrate how the cutting of follow-up has worked we can list the
records for select persons before and after the split:
\begin{Schunk}
\begin{Sinput}
 subset( Lr, lex.id %in% c(2:4,21) )[,c(1:9,12)]
\end{Sinput}
\begin{Soutput}
 lex.id     per   age tfi lex.dur lex.Cst lex.Xst  id sex      dor
      2 1989.53 30.23   0    6.60     NRA    ESRD  26   F 1989.814
      3 1987.85 25.83   0    5.39     NRA    ESRD  27   F       NA
      4 1995.24 44.50   0    8.75     NRA     NRA  33   M 1995.717
     21 1992.95 32.36   0    0.08     NRA     NRA 152   F       NA
\end{Soutput}
\begin{Sinput}
 subset( Lc, lex.id %in% c(2:4,21) )[,c(1:9,12)]
\end{Sinput}
\begin{Soutput}
 lex.id     per   age  tfi lex.dur lex.Cst   lex.Xst  id sex      dor
      2 1989.53 30.23 0.00    0.28     NRA       Rem  26   F 1989.814
      2 1989.81 30.51 0.28    6.32     Rem ESRD(Rem)  26   F 1989.814
      3 1987.85 25.83 0.00    5.39     NRA      ESRD  27   F       NA
      4 1995.24 44.50 0.00    0.47     NRA       Rem  33   M 1995.717
      4 1995.72 44.97 0.47    8.28     Rem       Rem  33   M 1995.717
     21 1992.95 32.36 0.00    0.08     NRA       NRA 152   F       NA
\end{Soutput}
\end{Schunk}

%% \item Now show how the states are connected and the number of transitions
%%   between them by using \texttt{boxes}. This is an interactive command
%%   that requires you to click in the graph window:
%% <<eval=FALSE>>=
%% boxes( Lc )
%% @ % $
%% It has a couple of fancy arguments, try:
%% <<Lc-boxes,fig=TRUE>>=
%% boxes( Lc, boxpos=TRUE, scale.R=100, show.BE=TRUE, hm=1.5, wm=1.5 )
%% @ % $
%% You may even be tempted to read the help page for
%% \texttt{boxes.Lexis} \ldots

\item We can show how the states are connected and the number of transitions
  between them by using \texttt{boxes}. This is an interactive command
  that requires you to click in the graph window

  Alternatively you can let R try to place the boxes for you, and even
  compute rates (in this case in units of events per 100 PY):
\begin{Schunk}
\begin{Sinput}
 # boxes( Lc, boxpos=TRUE, scale.R=100, show.BE=TRUE, hm=1.5, wm=1.5 )
 boxes( Relevel(Lc,c(1,2,4,3)),
        boxpos=TRUE, scale.R=100, show.BE=TRUE, hm=1.5, wm=1.5 )
\end{Sinput}
\end{Schunk}
\insfig{Lc-boxes}{0.7}{States and transitions between them.\\
  The numbers in each box are the person-years and the number of
  persons starting (left) and ending (right) their follow-up in each
  state; the numbers on the arrows are the number of transitions and
  the overall transition rates (in per 100 PY, by the \textrm{\tt
    scale.R=100}).}

%% \item Plot a Lexis diagram where different coloring is
%%   used for different segments of the follow-up. The
%%   \texttt{plot.Lexis} function draws a line for each record in the
%%   dataset, so you can index the coloring by \texttt{lex.Cst} and
%%   \texttt{lex.Xst} as appropriate --- indexing by a factor corresponds
%%   to indexing by the \emph{index number} of the factor levels, so you
%%   must be know which order the factor levels are in:
%% <<Lexis-rem,fig=TRUE>>=
%% par( mai=c(3,3,1,1)/4, mgp=c(3,1,0)/1.6 )
%% plot( Lc, col=c("red","limegreen")[Lc$lex.Cst],
%%       xlab="Calendar time", ylab="Age",
%%       lwd=3, grid=0:20*5, xlim=c(1970,2010), ylim=c(0,80), xaxs="i", yaxs="i", las=1 )
%% points( Lc, pch=c(NA,NA,16)[Lc$lex.Xst],
%%             col=c("red","limegreen","transparent")[Lc$lex.Cst])
%% points( Lc, pch=c(NA,NA,1)[Lc$lex.Xst],
%%             col="black", lwd=2 )
%% @ %

\item We can make a Lexis diagram where different coloring is
  used for different segments of the follow-up. The
  \texttt{plot.Lexis} function draws a line for each record in the
  dataset, so we can just index the coloring by \texttt{lex.Cst} and
  \texttt{lex.Xst} as appropriate --- indexing by a factor corresponds
  to indexing by the \emph{index number} of the factor levels, so you
  must be know which order the factor levels are in.
\begin{Schunk}
\begin{Sinput}
 par( mai=c(3,3,1,1)/4, mgp=c(3,1,0)/1.6 )
 plot( Lc, col=c("red","limegreen")[Lc$lex.Cst],
       xlab="Calendar time", ylab="Age",
       lwd=3, grid=0:20*5, xlim=c(1970,2010), ylim=c(0,80), xaxs="i", yaxs="i", las=1 )
 points( Lc, pch=c(NA,NA,16,16)[Lc$lex.Xst],
             col=c("red","limegreen","transparent")[Lc$lex.Cst])
 points( Lc, pch=c(NA,NA,1,1)[Lc$lex.Xst],
             col="black", lwd=2 )
\end{Sinput}
\end{Schunk}
\insfig{Lexis-rem}{0.9}{Lexis diagram for the split data, where time after
    remission is shown in green.}

%% \item Make Cox-regression of mortality (i.e. endpoint ``ESRD'' or
%%   ``ESRD(Rem)'') with sex, age at entry and remission as
%%   explanatory variables, using time since entry as timescale, and
%%   include \texttt{lex.Cst} as time-dependent variable, and
%%   indicate that each record represents follow-up from
%%   \texttt{tfi} to \texttt{tfi+lex.dur}. Make sure that you know
%%   why what goes where here in the call to \texttt{coxph}.
%% <<>>=
%% ( EP <- levels(Lc)[3:4] )
%% m1 <- coxph( Surv( tfi,                  # from
%%                    tfi+lex.dur,          # to
%%                    lex.Xst %in% EP ) ~   # event
%%              sex + I((doe-dob-50)/10) +  # fixed covariates
%%              (lex.Cst=="Rem"),           # time-dependent variable
%%              data = Lc )
%% summary( m1 )
%% @ %
%% What is the effect of of remission on the rate of ESRD?

\item We now make Cox-regression of mortality (i.e. endpoint
       ``ESRD'') with sex, age at entry and remission as explanatory
       variables, using time since entry as timescale.

  We include \texttt{lex.Cst} as time-dependent variable, and
  indicate that each record represents follow-up from \texttt{tfi}
  to \texttt{tfi+lex.dur}.
\begin{Schunk}
\begin{Sinput}
 ( EP <- levels(Lc)[3:4] )
\end{Sinput}
\begin{Soutput}
[1] "ESRD"      "ESRD(Rem)"
\end{Soutput}
\begin{Sinput}
 m1 <- coxph( Surv( tfi,                  # from
                    tfi+lex.dur,          # to
                    lex.Xst %in% EP ) ~   # event
              sex + I((doe-dob-50)/10) +  # fixed covariates
              (lex.Cst=="Rem"),           # time-dependent variable
              data = Lc )
 summary( m1 )
\end{Sinput}
\begin{Soutput}
Call:
coxph(formula = Surv(tfi, tfi + lex.dur, lex.Xst %in% EP) ~ sex + 
    I((doe - dob - 50)/10) + (lex.Cst == "Rem"), data = Lc)

  n= 154, number of events= 77 

                           coef exp(coef) se(coef)      z Pr(>|z|)    
sexF                   -0.05534   0.94616  0.27500 -0.201 0.840517    
I((doe - dob - 50)/10)  0.52190   1.68522  0.13655  3.822 0.000132 ***
lex.Cst == "Rem"TRUE   -1.26241   0.28297  0.38483 -3.280 0.001036 ** 
---
Signif. codes:  0 '***' 0.001 '**' 0.01 '*' 0.05 '.' 0.1 ' ' 1

                       exp(coef) exp(-coef) lower .95 upper .95
sexF                      0.9462     1.0569    0.5519    1.6220
I((doe - dob - 50)/10)    1.6852     0.5934    1.2895    2.2024
lex.Cst == "Rem"TRUE      0.2830     3.5339    0.1331    0.6016

Concordance= 0.664  (se = 0.033 )
Likelihood ratio test= 30.31  on 3 df,   p=1e-06
Wald test            = 27.07  on 3 df,   p=6e-06
Score (logrank) test = 29.41  on 3 df,   p=2e-06
\end{Soutput}
\end{Schunk}
  We see that the rate of ESRD is less than a third among those
  who obtain remission --- 0.28 (0.13--0.60), showing that we can be
  pretty sure that the rate is at least halved.

%% \end{enumerate}
\end{enumerate}

%% \subsection{Splitting the follow-up time}
\subsection{Splitting the follow-up time}

%% In order to explore the effect of remission on the rate of ESRD, we
%% shall split the data further into small pieces of follow-up. To this
%% end we use the function \texttt{splitLexis}. The rates can then be
%% modeled using a Poisson-model, and the shape of the underlying
%% \emph{rates} be explored. Furthermore, we can allow effects of both
%% time since NRA and current age. To this end we will use splines, so we
%% need the \texttt{splines} and also the \texttt{mgcv} packages.

%% \begin{enumerate}[resume]
\begin{enumerate}[resume]

%% \item Now split the follow-up time every month after entry, and verify
%%   that the number of events and risk time is the same as before and
%%   after the split:
%% <<>>=
%% sLc <- splitLexis( Lc, "tfi", breaks=seq(0,30,1/12) )
%% summary( Lc, scale=100 )
%% summary(sLc, scale=100 )
%% @ %

\item We split the follow-up time every month after entry, and verify
  that the number of events and risk time is the same as before and
  after the split:
\begin{Schunk}
\begin{Sinput}
 sLc <- splitLexis( Lc, "tfi", breaks=seq(0,30,1/12) )
 summary( Lc, scale=100 )
\end{Sinput}
\begin{Soutput}
Transitions:
     To
From  NRA Rem ESRD ESRD(Rem)  Records:  Events: Risk time:  Persons:
  NRA  24  29   69         0       122       98       8.25       122
  Rem   0  24    0         8        32        8       2.60        32
  Sum  24  53   69         8       154      106      10.85       125
\end{Soutput}
\begin{Sinput}
 summary(sLc, scale=100 )
\end{Sinput}
\begin{Soutput}
Transitions:
     To
From   NRA  Rem ESRD ESRD(Rem)  Records:  Events: Risk time:  Persons:
  NRA 9854   29   69         0      9952       98       8.25       122
  Rem    0 3139    0         8      3147        8       2.60        32
  Sum 9854 3168   69         8     13099      106      10.85       125
\end{Soutput}
\end{Schunk}
  Thus both the cutting and splitting preserves the number of ESRD
  events and the person-years. The cut added the ``Rem'' events, but
  these were preserved by the splitting.

%% \item Try to fit the Poisson-model corresponding to the Cox-model
%%   we fitted previously. The function \texttt{ns()} produces a model
%%   matrix corresponding to a piece-wise cubic function, modeling the
%%   baseline hazard explicitly (think of the \texttt{ns} terms as the
%%   baseline hazard that is not visible in the Cox-model)
%% <<>>=
%% library( splines )
%% mp <- glm( lex.Xst %in% EP ~ ns( tfi, df=4 ) +
%%            sex + I((doe-dob-40)/10) + I(lex.Cst=="Rem"),
%%            offset = log(lex.dur),
%%            family = poisson,
%%              data = sLc )
%% ci.exp( mp )
%% @ %
%% How does the effects of sex change from the Cox-model?

\item Now we fit the Poisson-model corresponding to the Cox-model
  we fitted previously. The function \texttt{Ns()} produces a model
  matrix corresponding to a piece-wise cubic function, modeling the
  baseline hazard explicitly (think of the \texttt{Ns} terms as the
  baseline hazard that is not visible in the Cox-model)
\begin{Schunk}
\begin{Sinput}
 library( splines )
 mp <- glm( lex.Xst %in% EP ~ Ns( tfi, knots=c(0,2,5,10) ) +
            sex + I((doe-dob-40)/10) + I(lex.Cst=="Rem"),
            offset = log(lex.dur),
            family = poisson,
              data = sLc )
 ci.exp( mp )
\end{Sinput}
\begin{Soutput}
                                   exp(Est.)        2.5%        97.5%
(Intercept)                       0.01664432 0.003956765   0.07001509
Ns(tfi, knots = c(0, 2, 5, 10))1  5.18917654 1.949220473  13.81452410
Ns(tfi, knots = c(0, 2, 5, 10))2 34.20004192 1.764901998 662.72397483
Ns(tfi, knots = c(0, 2, 5, 10))3  4.43318269 2.179992749   9.01521750
sexF                              0.91751162 0.536258807   1.56981584
I((doe - dob - 40)/10)            1.70082390 1.300814311   2.22383927
I(lex.Cst == "Rem")TRUE           0.27927558 0.131397003   0.59358165
\end{Soutput}
\end{Schunk}
We see that the effects are pretty much the same as from the
Cox-model.

%% \item Try instead using the \texttt{gam} function from the
%%   \texttt{mgcv} package. There is convenience wrapper for this for
%%   \texttt{Lexis} objects as well:
%% <<>>=
%% library( mgcv )
%% mx <- gam.Lexis(sLc, 
%%                 ~ s(tfi, k=10) + sex + I((doe-dob-40)/10) + I(lex.Cst=="Rem"))
%% ci.exp( mp, subset=c("Cst","doe","sex") )
%% ci.exp( mx, subset=c("Cst","doe","sex") )
%% @ %
%% We see that there is virtually no difference between the two
%% approaches in terms of the regression parameters.

\item We may instead use the \texttt{gam} function from the
  \texttt{mgcv} package. There is convenience wrapper for this as well:
\begin{Schunk}
\begin{Sinput}
 library( mgcv )
 mx <- gam.Lexis(sLc, 
                 ~ s(tfi, k=10) + sex + I((doe-dob-40)/10) + I(lex.Cst=="Rem"))
\end{Sinput}
\begin{Soutput}
mgcv::gam Poisson analysis of Lexis object sLc with log link:
Rates for transitions:
NRA->ESRD
Rem->ESRD(Rem)
\end{Soutput}
\begin{Sinput}
 ci.exp( mp, subset=c("Cst","doe","sex") )
\end{Sinput}
\begin{Soutput}
                        exp(Est.)      2.5%     97.5%
I(lex.Cst == "Rem")TRUE 0.2792756 0.1313970 0.5935816
I((doe - dob - 40)/10)  1.7008239 1.3008143 2.2238393
sexF                    0.9175116 0.5362588 1.5698158
\end{Soutput}
\begin{Sinput}
 ci.exp( mx, subset=c("Cst","doe","sex") )
\end{Sinput}
\begin{Soutput}
                        exp(Est.)      2.5%     97.5%
I(lex.Cst == "Rem")TRUE 0.2784659 0.1309446 0.5921838
I((doe - dob - 40)/10)  1.6992069 1.2995225 2.2218192
sexF                    0.9309945 0.5435486 1.5946150
\end{Soutput}
\end{Schunk}
We see that there is virtually no difference between the two
approaches in terms of the regression parameters.

%% \item Extract the regression parameters from the models using
%%   \texttt{ci.exp} and compare with the estimates from the Cox-model:
%% <<>>=
%% ci.exp( mx, subset=c("sex","dob","Cst"), pval=TRUE )
%% ci.exp( m1 )
%% round( ci.exp( mp, subset=c("sex","dob","Cst") ) / ci.exp( m1 ), 2 )
%% @ %
%% How lare is the difference in estimated regression parameters?

\item We extract the regression parameters from the models using
  \texttt{ci.exp} and compare with the estimates from the Cox-model:
\begin{Schunk}
\begin{Sinput}
 ci.exp( mx, subset=c("sex","dob","Cst"), pval=TRUE )
\end{Sinput}
\begin{Soutput}
                        exp(Est.)      2.5%     97.5%            P
sexF                    0.9309945 0.5435486 1.5946150 0.7945394004
I((doe - dob - 40)/10)  1.6992069 1.2995225 2.2218192 0.0001066910
I(lex.Cst == "Rem")TRUE 0.2784659 0.1309446 0.5921838 0.0008970863
\end{Soutput}
\begin{Sinput}
 ci.exp( m1 )
\end{Sinput}
\begin{Soutput}
                       exp(Est.)      2.5%    97.5%
sexF                   0.9461646 0.5519334 1.621985
I((doe - dob - 50)/10) 1.6852196 1.2895097 2.202360
lex.Cst == "Rem"TRUE   0.2829710 0.1330996 0.601599
\end{Soutput}
\begin{Sinput}
 round( ci.exp( mp, subset=c("sex","dob","Cst") ) / ci.exp( m1 ), 3 )
\end{Sinput}
\begin{Soutput}
                        exp(Est.)  2.5% 97.5%
sexF                        0.970 0.972 0.968
I((doe - dob - 40)/10)      1.009 1.009 1.010
I(lex.Cst == "Rem")TRUE     0.987 0.987 0.987
\end{Soutput}
\end{Schunk}
Thus we see that it has an absolute minimal influence on the
regression parameters to impose the assumption of smoothly varying
rates or not.

%% \item The model has the same assumptions as the Cox-model about
%%   proportionality of rates, but there is an additional assumption that
%%   the hazard is a smooth function of time since entry. It seems to be
%%   a sensible assumption (well, restriction) to put on the rates that
%%   they vary smoothly by time. No such restriction is made in the Cox
%%   model. The \texttt{gam} model optimizes the shape of the smoother by
%%   general cross-validation. Try to look at the shape of the
%%   estimated effect of \texttt{tfi}:
%% <<>>=
%% plot( mx )
%% @ %
%% Is this a useful plot?

\item The model has the same assumptions as the Cox-model about
  proportionality of rates, but there is an additional assumption that
  the hazard is a smooth function of time since entry. It seems to be
  a sensible assumption (well, restriction) to put on the rates that
  they vary smoothly by time. No such restriction is made in the Cox
  model. The \texttt{gam} model optimizes the shape of the smoother by
  general cross-validation:
\begin{Schunk}
\begin{Sinput}
 plot( mx )
\end{Sinput}
\end{Schunk}
\insfig{tfi-gam}{0.7}{Estimated non-linear effect of \texttt{tfi} as
  estimated by \texttt{gam}.}

%% \item However, \texttt{plot} does not give you the \emph{absolute}
%%   level of the underlying rates because it bypasses the intercept. So
%%   try to predict the rates as a function of \texttt{tfi} and the
%%   covariates, by setting up a prediction data frame. Note that age
%%   in the model specification is entered as \texttt{doe-dob}, hence
%%   the prediction data frame must have these two variables and not
%%   the age, but it is onlythe difference that matters for the prediction:
%% <<pred,fig=TRUE>>=
%% nd <- data.frame( tfi = seq(0,20,0.1),
%%                   sex = "M",
%%                   doe = 1990,
%%                   dob = 1940,
%%               lex.Cst = "NRA",
%%               lex.dur = 1 )
%% str( nd )
%% matplot( nd$tfi, ci.pred( mx, newdata=nd )*100,
%%          type="l", lty=1, lwd=c(3,1,1), col="black",
%%          log="y", xlab="Time since entry (years)",
%%                   ylab="ESRD rate (per 100 PY) for 50 year man" )
%% @ %
%% Try to overlay with the corresponding prediction from the
%% \texttt{glm} model using \texttt{Ns}.

\item However, \texttt{plot} does not give you the \emph{absolute}
  level of the underlying rates because it bypasses the intercept. If
  we want this we can predict the rates as a function of the covariates:
\begin{Schunk}
\begin{Sinput}
 nd <- data.frame( tfi = seq(0,20,.1),
                   sex = "M",
                   doe = 1990,
                   dob = 1940,
               lex.Cst = "NRA",
               lex.dur = 100 )
 str( nd )
\end{Sinput}
\begin{Soutput}
'data.frame':	201 obs. of  6 variables:
 $ tfi    : num  0 0.1 0.2 0.3 0.4 0.5 0.6 0.7 0.8 0.9 ...
 $ sex    : chr  "M" "M" "M" "M" ...
 $ doe    : num  1990 1990 1990 1990 1990 1990 1990 1990 1990 1990 ...
 $ dob    : num  1940 1940 1940 1940 1940 1940 1940 1940 1940 1940 ...
 $ lex.Cst: chr  "NRA" "NRA" "NRA" "NRA" ...
 $ lex.dur: num  100 100 100 100 100 100 100 100 100 100 ...
\end{Soutput}
\begin{Sinput}
 matshade( nd$tfi, cbind( ci.pred( mp, newdata=nd ),
                          ci.pred( mx, newdata=nd ) ), plot=TRUE,
           type="l", lwd=3:4, col=c("black","forestgreen"),
           log="y", xlab="Time since entry (years)",
                    ylab="ESRD rate (per 100 PY) for 50 year man" )
\end{Sinput}
\end{Schunk}
\insfig{pred}{0.7}{Rates of ESRD by time since NRA for a man aged 50 at
  start of NRA. The green line is the curve fitted by \texttt{gam}, the
  black the one fitted by an ordinary \texttt{glm} using \texttt{Ns}
  with knots at 0, 2, 5 and 10 years.}

%% \end{enumerate}
\end{enumerate}

%% \subsection{Prediction from the multistate model}
\subsection{Prediction from the multistate model}

%% If we want to make proper statements about the survival and disease
%% probabilities we must know not only how the occurrence of remission
%% influences the rate of death/ESRD, but we must also model the
%% occurrence rate of remission itself.

If we want to make proper statements about the survival and disease
probabilities we must know not only how the occurrence of remission
influences the rate of death/ESRD, but we must also model the
occurrence rate of remission itself.

%% \begin{enumerate}[resume]
\begin{enumerate}[resume]

%% \item The rates of ESRD were modelled by a Poisson model with
%%   effects of age and time since NRA --- in the models \texttt{mp}
%%   and \texttt{mx}.  But if we want to model whole process we must
%%   also model the remission rates transition from ``NRA'' to
%%   ``Rem'', but the number of events is rather small so we restrict
%%   covariates in this model to only time since NRA and sex. Note
%%   that only the records that represent follow-up in the ``NRA''
%%   state should be used; this is most easily done using the
%%   \texttt{gam.Lexis} function
%% <<rem-inc-mgcv>>=
%% mr <- gam.Lexis(sLc, ~ s( tfi, k=10 ) + sex,
%%                      from = "NRA", 
%%                        to = "Rem")
%% ci.exp(mr, pval = TRUE)
%% @
%% What is the remission rate-ration between men and women?

\item The rates of ESRD were modelled by a Poisson model with effects
  of age and time since NRA --- in the models \texttt{mp} and
  \texttt{mx}.  But if we want to model whole process we must also
  model the remission rates transition from ``NRA'' to ``Rem'', but
  the number of events is rather small (see figure
  \ref{fig:Lc-boxes}), so we restrict covariates in this model to only time since
  NRA and sex. Note that only the records that relate to the ``NRA'' state
  can be used; this is done with the \texttt{from} argument to \texttt{gam.Lexis}:
\begin{Schunk}
\begin{Sinput}
 mr <- gam.Lexis(sLc, ~ s( tfi, k=10 ) + sex,
                      from = "NRA", 
                        to = "Rem")
\end{Sinput}
\begin{Soutput}
mgcv::gam Poisson analysis of Lexis object sLc with log link:
Rates for the transition:
NRA->Rem
\end{Soutput}
\begin{Sinput}
 ci.exp(mr, pval = TRUE)
\end{Sinput}
\begin{Soutput}
             exp(Est.)       2.5%      97.5%            P
(Intercept) 0.02466177 0.01486719 0.04090905 1.254086e-46
sexF        2.60619252 1.25503371 5.41199761 1.019157e-02
s(tfi).1    1.00756081 0.83823640 1.21108889 9.360463e-01
s(tfi).2    0.99578104 0.81097748 1.22269718 9.678018e-01
s(tfi).3    0.99740759 0.89841258 1.10731073 9.611811e-01
s(tfi).4    1.00231534 0.89552909 1.12183519 9.679048e-01
s(tfi).5    0.99773836 0.90957003 1.09445321 9.617435e-01
s(tfi).6    0.99775106 0.90656107 1.09811376 9.632777e-01
s(tfi).7    1.00244317 0.91723283 1.09556950 9.570639e-01
s(tfi).8    0.99404123 0.69737098 1.41691870 9.736371e-01
s(tfi).9    0.94790708 0.63364240 1.41803616 7.946051e-01
\end{Soutput}
\end{Schunk}
We see that there is a clear effect of sex; women have a remission
rate 2.6 times higher than for men.

%% \item If we want to predict the probability of being in each of the
%%   three states using these estimated rates, we may resort to
%%   analytical calculations of the probabilities from the estimated
%%   rates, which is actually doable in this case, but which will be largely
%%   intractable for more complicated models.
%%
%%   Alternatively we can \emph{simulate} the life course for a large
%%   group of (identical) individuals through a model using the estimated
%%   rates. That will give a simulated cohort (in the form of a
%%   \texttt{Lexis} object), and we can then just count the number of
%%   persons in each state at each of a set of time points.
%%   This is accomplished using the function \texttt{simLexis}. The input
%%   to this is the initial status of the persons whose life-course we
%%   shall simulate, and the transition rates in suitable form:
%% \begin{itemize}
%% \item Suppose we want predictions for men aged 50 at
%%   NRA. The input is in the form of a \texttt{Lexis} object (where
%%   \texttt{lex.dur} and \texttt{lex.Xst} will be ignored). Note that in
%%   order to carry over the \texttt{time.scales} and the
%%   \texttt{time.since} attributes, we construct the input object using
%%   \texttt{subset} to select columns, and \texttt{NULL} to select rows
%%   (see the example in the help file for \texttt{simLexis}):
%% <<>>=
%% inL <- subset( sLc, select=1:11 )[NULL,]
%% str( inL )
%% timeScales(inL)
%% inL[1,"lex.id"] <- 1
%% inL[1,"per"] <- 2000
%% inL[1,"age"] <- 50
%% inL[1,"tfi"] <- 0
%% inL[1,"lex.Cst"] <- "NRA"
%% inL[1,"lex.Xst"] <- NA
%% inL[1,"lex.dur"] <- NA
%% inL[1,"sex"] <- "M"
%% inL[1,"doe"] <- 2000
%% inL[1,"dob"] <- 1950
%% inL <- rbind( inL, inL )
%% inL[2,"sex"] <- "F"
%% inL
%% str( inL )
%% @ %
%% \item The other input for the simulation is the transitions, which is
%%   a list with an element for each transient state (that is ``NRA'' and
%%   ``Rem''), each of which is again a list with names equal to the
%%   states that can be reached from the transient state. The content of
%%   the list will be \texttt{glm} objects, in this case the models we
%%   just fitted, describing the transition rates:
%% <<>>=
%% Tr <- list( "NRA" = list( "Rem"  = mr,
%%                           "ESRD" = mx ),
%%             "Rem" = list( "ESRD(Rem)" = mx ) )
%% @ %
%% \end{itemize}
%% With this as input we can now generate a cohort, using \texttt{N=5}
%% to simulate life course of 10 persons (5 for each set of starting values
%% in \texttt{inL}):
%% <<first-sim>>=
%% ( iL <- simLexis( Tr, inL, N=10 ) )
%% summary( iL, by="sex" )
%% @ %
%% What type of object have you got as \texttt{iL}.
%% Simulate a couple of thousand persons.

\item In order to use the function \texttt{simLexis} we must have as input
  to this the initial status of the persons whose life-course we
  shall simulate, and the transition rates in suitable form:

\begin{itemize}
\item Suppose we want predictions for men aged 50 at
  NRA. The input is in the form of a \texttt{Lexis} object (where
  \texttt{lex.dur} and \texttt{lex.Xst} will be ignored). Note that in
  order to carry over the \texttt{time.scales} and the
  \texttt{time.since} attributes, we construct the input object using
  \texttt{subset} to select columns, and \texttt{NULL} to select rows
  (see the example in the help file for \texttt{simLexis}):
\begin{Schunk}
\begin{Sinput}
 inL <- subset( sLc, select=1:11 )[NULL,]
 str( inL )
\end{Sinput}
\begin{Soutput}
Classes 'Lexis' and 'data.frame':	0 obs. of  11 variables:
 $ lex.id : int 
 $ per    : num 
 $ age    : num 
 $ tfi    : num 
 $ lex.dur: num 
 $ lex.Cst: Factor w/ 4 levels "NRA","Rem","ESRD",..: 
 $ lex.Xst: Factor w/ 4 levels "NRA","Rem","ESRD",..: 
 $ id     : num 
 $ sex    : Factor w/ 2 levels "M","F": 
 $ dob    : num 
 $ doe    : num 
 - attr(*, "time.scales")= chr [1:3] "per" "age" "tfi"
 - attr(*, "time.since")= chr [1:3] "" "" ""
 - attr(*, "breaks")=List of 3
  ..$ per: NULL
  ..$ age: NULL
  ..$ tfi: num [1:361] 0 0.0833 0.1667 0.25 0.3333 ...
\end{Soutput}
\begin{Sinput}
 timeScales(inL)
\end{Sinput}
\begin{Soutput}
[1] "per" "age" "tfi"
\end{Soutput}
\begin{Sinput}
 inL[1,"lex.id"] <- 1
 inL[1,"per"] <- 2000
 inL[1,"age"] <- 50
 inL[1,"tfi"] <- 0
 inL[1,"lex.Cst"] <- "NRA"
 inL[1,"lex.Xst"] <- NA
 inL[1,"lex.dur"] <- NA
 inL[1,"sex"] <- "M"
 inL[1,"doe"] <- 2000
 inL[1,"dob"] <- 1950
 inL <- rbind( inL, inL )
 inL[2,"sex"] <- "F"
 inL
\end{Sinput}
\begin{Soutput}
 lex.id  per age tfi lex.dur lex.Cst lex.Xst id sex  dob  doe
      1 2000  50   0      NA     NRA    <NA> NA   M 1950 2000
      1 2000  50   0      NA     NRA    <NA> NA   F 1950 2000
\end{Soutput}
\begin{Sinput}
 str( inL )
\end{Sinput}
\begin{Soutput}
Classes 'Lexis' and 'data.frame':	2 obs. of  11 variables:
 $ lex.id : num  1 1
 $ per    : num  2000 2000
 $ age    : num  50 50
 $ tfi    : num  0 0
 $ lex.dur: num  NA NA
 $ lex.Cst: Factor w/ 4 levels "NRA","Rem","ESRD",..: 1 1
 $ lex.Xst: Factor w/ 4 levels "NRA","Rem","ESRD",..: NA NA
 $ id     : num  NA NA
 $ sex    : Factor w/ 2 levels "M","F": 1 2
 $ dob    : num  1950 1950
 $ doe    : num  2000 2000
 - attr(*, "breaks")=List of 3
  ..$ per: NULL
  ..$ age: NULL
  ..$ tfi: num [1:361] 0 0.0833 0.1667 0.25 0.3333 ...
 - attr(*, "time.scales")= chr [1:3] "per" "age" "tfi"
 - attr(*, "time.since")= chr [1:3] "" "" ""
\end{Soutput}
\end{Schunk}

\item The other input for the simulation is the transitions, which is
  a list with an element for each transient state (that is ``NRA'' and
  ``Rem''), each of which is again a list with names equal to the
  states that can be reached from the transient state. The content of
  the list will be \texttt{glm} objects, in this case the models we
  just fitted, describing the transition rates:
\begin{Schunk}
\begin{Sinput}
 Tr <- list("NRA" = list("Rem"       = mr,
                         "ESRD"      = mx ),
            "Rem" = list("ESRD(Rem)" = mx))
\end{Sinput}
\end{Schunk}
\end{itemize}

%% \item Now generate the life course of 5,000 persons, and look at the summary.
%%   The \texttt{system.time} command is just to tell you how long it
%%   took, you may want to start with 1000 just to see how long that takes.
%% <<5000-sim>>=
%% system.time(
%% sM <- simLexis( Tr, inL, N = 5000, t.range = 12 ) )
%% summary( sM, by="sex" )
%% @ %
%% Why are there so many ESRD-events in the resulting data set?

\item Now generate the life course of 5,000 persons (of each sex), and
  look at the summary.  The \texttt{system.time} command is just to
  tell you how long it took, you may want to start with 100 just to
  see how long that takes.
\begin{Schunk}
\begin{Sinput}
 system.time( sM <- simLexis( Tr, inL, N=5000 ) )
\end{Sinput}
\begin{Soutput}
   user  system elapsed 
  24.37    1.86   26.24 
\end{Soutput}
\begin{Sinput}
 # save( sM, file="sM.Rda" )
 # load(     file="sM.Rda" )
 summary( sM, by="sex" )
\end{Sinput}
\begin{Soutput}
$M
     
Transitions:
     To
From  NRA Rem ESRD ESRD(Rem)  Records:  Events: Risk time:  Persons:
  NRA  16 744 4240         0      5000     4984   27966.09      5000
  Rem   0 212    0       532       744      532    7480.86       744
  Sum  16 956 4240       532      5744     5516   35446.94      5000

$F
     
Transitions:
     To
From  NRA  Rem ESRD ESRD(Rem)  Records:  Events: Risk time:  Persons:
  NRA  17 1677 3306         0      5000     4983   25396.38      5000
  Rem   0  490    0      1187      1677     1187   17217.14      1677
  Sum  17 2167 3306      1187      6677     6170   42613.52      5000
\end{Soutput}
\end{Schunk}
The many ESRD-events in the resulting data set is attributable to the
fact that we simulate for a very long follow-up time.

%% \item Now count how many persons are present in each state
%%   at each time for the first 10 years after entry (which is at age 50). This
%%   can be done by using \texttt{nState}. Try:
%% <<nState>>=
%% nStm <- nState( subset(sM,sex=="M"), at=seq(0,10,0.1), from=50, time.scale="age" )
%% nStf <- nState( subset(sM,sex=="F"), at=seq(0,10,0.1), from=50, time.scale="age" )
%% head( nStf )
%% @ %
%% What is tn the object \texttt{nStf}?

\item Now we want to count how many persons are present in each state
  at each time for the first 10 years after entry (which is at age 50). This
  can be done by using \texttt{nState}:
\begin{Schunk}
\begin{Sinput}
 nStm <- nState( subset(sM,sex=="M"), at=seq(0,10,0.1), from=50, time.scale="age" )
 nStf <- nState( subset(sM,sex=="F"), at=seq(0,10,0.1), from=50, time.scale="age" )
 head( nStf )
\end{Sinput}
\begin{Soutput}
      State
when    NRA  Rem ESRD ESRD(Rem)
  50   5000    0    0         0
  50.1 4938   43   19         0
  50.2 4887   76   37         0
  50.3 4840  100   60         0
  50.4 4783  134   83         0
  50.5 4742  162   96         0
\end{Soutput}
\end{Schunk}
  We see that we get a count of persons in each state at time points
  0,0.1,0.2,\ldots years after 50 on the age time scale.

%% \item With the counts of persons in each state at the
%%   designated time points (in \texttt{nStm}), compute the cumulative fraction over the
%%   states, arranged in order given by \texttt{perm}:
%% <<pState>>=
%% ppm <- pState( nStm, perm=c(2,1,3,4) )
%% ppf <- pState( nStf, perm=c(2,1,3,4) )
%% head( ppf )
%% tail( ppf )
%% @ %
%% What do the entries in \texttt{ppf} represent?

\item Once we have the counts of persons in each state at the
  designated time points, we compute the cumulative fraction over the
  states, arranged in order given by \texttt{perm}:
\begin{Schunk}
\begin{Sinput}
 ppm <- pState( nStm, perm=c(2,1,3,4) )
 ppf <- pState( nStf, perm=c(2,1,3,4) )
 head( ppf )
\end{Sinput}
\begin{Soutput}
      State
when      Rem    NRA ESRD ESRD(Rem)
  50   0.0000 1.0000    1         1
  50.1 0.0086 0.9962    1         1
  50.2 0.0152 0.9926    1         1
  50.3 0.0200 0.9880    1         1
  50.4 0.0268 0.9834    1         1
  50.5 0.0324 0.9808    1         1
\end{Soutput}
\begin{Sinput}
 tail( ppf )
\end{Sinput}
\begin{Soutput}
      State
when      Rem    NRA   ESRD ESRD(Rem)
  59.5 0.2324 0.3546 0.9180         1
  59.6 0.2324 0.3512 0.9172         1
  59.7 0.2314 0.3484 0.9158         1
  59.8 0.2308 0.3446 0.9146         1
  59.9 0.2308 0.3414 0.9138         1
  60   0.2304 0.3376 0.9126         1
\end{Soutput}
\end{Schunk}

%% \item Try to plot the cumulative probabilities using the \texttt{plot}
%%   method for \texttt{pState} objects:
%% <<plot-pp,fig=TRUE>>=
%% plot( ppf )
%% @ %
%% Is this useful?

\item Then we plot the cumulative probabilities using the \texttt{plot}
  method for \texttt{pState} objects:
\begin{Schunk}
\begin{Sinput}
 plot( ppf )
\end{Sinput}
\end{Schunk}
\insfig{plot-pp}{0.7}{The default plot for a \texttt{pState} object,
  bottom to top: Alive remission, alive no remission, dead no remission,
  dead remission.}

%% \item Now try to improve the plot so that it is easier to read, and
%%   easier to comapre men and women:
%% <<new-nState,fig=TRUE>>=
%% par( mfrow=c(1,2) )
%% plot( ppm, col=c("limegreen","red","#991111","forestgreen") )
%% lines( as.numeric(rownames(ppm)), ppm[,"Rem"], lwd=4 )
%% text( 59.5, 0.95, "Men", adj=1, col="white", font=2, cex=1.2 )
%% axis( side=4, at=0:10/10 )
%% axis( side=4, at=1:99/100, labels=NA, tck=-0.01 )
%% plot( ppf, col=c("limegreen","red","#991111","forestgreen"), xlim=c(60,50) )
%% lines( as.numeric(rownames(ppf)), ppf[,"Rem"], lwd=4 )
%% text( 59.5, 0.95, "Women", adj=0, col="white", font=2, cex=1.2 )
%% axis( side=2, at=0:10/10 )
%% axis( side=2, at=1:99/100, labels=NA, tck=-0.01 )
%% @ %
%% What is the 10-year risk of remission for men and women respectively?

\item Now try to improve the plot so that it is easier to read, and
  easier to compare men and women:
\begin{Schunk}
\begin{Sinput}
 par( mfrow=c(1,2), mar=c(3,1.5,1,1), oma=c(0,2,0,0), las=1 )
 #
 plot( ppm, col=c("limegreen","red","#991111","forestgreen") )
 mtext( "Probability", side=2, las=0, outer=T, line=0.5 )
 lines( as.numeric(rownames(ppm)), ppm[,"NRA"], lwd=4 )
 text( 59.5, 0.95, "Men", adj=1, col="white", font=2, cex=1.2 )
 axis( side=4, at=0:10/10 )
 axis( side=4, at=0:20/20 , labels=NA, tck=-0.02 )
 axis( side=4, at=1:99/100, labels=NA, tck=-0.01 )
 #
 plot( ppf, col=c("limegreen","red","#991111","forestgreen"),
            xlim=c(60,50), yaxt="n", ylab="" )
 lines( as.numeric(rownames(ppf)), ppf[,"NRA"], lwd=4 )
 text( 59.5, 0.95, "Women", adj=0, col="white", font=2, cex=1.2 )
 axis( side=2, at=0:10/10 , labels=NA )
 axis( side=2, at=0:20/20 , labels=NA, tck=-0.02 )
 axis( side=2, at=1:99/100, labels=NA, tck=-0.01 )
\end{Sinput}
\end{Schunk}
\insfig{new-pState}{1.05}{Predicted state occupancy for men and women
  entering at age 50. The green areas are remission, the red without
  remission; the black line is the survival curve.}

We see that the probability that a 50-year old man with NRA sees a
remission from NRA during the next 10 years is about 25\% whereas the
same for a woman is about 50\%. Also it is apparent that no new
remissions occur after about 5 years since NRA --- mainly because only
persons with remission are alive after 5 years.

%% \end{enumerate}
\end{enumerate}
