
\renewcommand{\rwpre}{./graph/DMDK}
\section{Time-splitting, time-scales and SMR}
This exercise is about mortaity among Danish Diabetes patients. It is
based on the dataset \texttt{DMlate}, a random sample of 10,000
patients from the Danish Diabetes Register (scrambeled dates), all
with date of diagnosis after 1994.
\begin{enumerate}
\item First load the data and take a look at the data:
\begin{Schunk}
\begin{Sinput}
 library( Epi )
 library( mgcv )
 library( splines )
 sessionInfo()
 data( DMlate )
 str( DMlate )
\end{Sinput}
\end{Schunk}
You can get a more detailed explanation of the data by referring to
the help page:
\begin{Schunk}
\begin{Sinput}
 ?DMlate
\end{Sinput}
\end{Schunk}
\item Set up the dataset as a \texttt{Lexis} object with age, calendar
  time and duration of diabetes as timescales, and date of death as
  event. Make sure that you know what each of the arguments to
  \texttt{Lexis} mean:
\begin{Schunk}
\begin{Sinput}
 LL <- Lexis( entry = list( A = dodm-dobth,
                            P = dodm,
                          dur = 0 ),
               exit = list( P = dox ),
        exit.status = factor( !is.na(dodth),
                              labels=c("Alive","Dead") ),
               data = DMlate )
\end{Sinput}
\end{Schunk}
Take a look at the first few lines of the resulting dataset using \texttt{head()}.
\item Get an overall overview of the mortality by using
  \texttt{stat.table} to tabulate no. deaths, person-years and the
  crude mortality rate by sex.
\item If we want to assess how mortality depends on age, calendar time
  and duration or how it relates to population mortality, we should
  in principle split the follow-up along all 
  three time scales. In practice it is sufficient to split it along
  one of the time-scales and then use the value of each of the
  time-scales at the left endpoint of the intervals.
  Use \texttt{splitLexis} (or \texttt{splitMulti} from the
  \texttt{popEpi} package) to split the follow-up along the
  age-axis in sutiable intervals (here set to 1/2 year, but really
  immaterial as long as it is small):
\begin{Schunk}
\begin{Sinput}
 SL <- splitLexis( LL, breaks=seq(0,125,1/2), time.scale="A" )
 summary( SL )
\end{Sinput}
\end{Schunk}
  How many records are now in the dataset? How many person-years?
  Compare to the original \texttt{Lexis}-dataset.
\end{enumerate}
\subsection*{SMR}
The SMR is the \textbf{S}tandardized \textbf{M}ortality
\textbf{R}atio, which is the mortality rate-ratio between the diabetes
patients and the general population.  In real studies we would
subtract the deaths and the person-years among the diabetes patients
from those of the general population, but since we do not have access
to these, we make the comparison to the general population at large,
\textit{i.e.} also including the diabetes patients.
We now want to include the population mortality rates as a fixed
variable in the split dataset; for each record in the split dataset we
attach the value of the population mortality for the relevant sex, and
and calendar time.
This can be achieved in two ways: Either we just use the current split
of follow-up time and allocate the population mortality rates for some
suitably chosen (mid-)point of the follow-up in each, or we make a
second split by date, so that follow-up in the diabetes patients is in
the same classification of age and data as the population mortality
table.
\begin{enumerate}[resume]
\item We will use the former approach, using the dataset split in
  6 month intervals, and then include as an extra variable the
  population mortality as available from the data set
  \texttt{M.dk}.
  First create the variables in the diabetes dataset that we need
  for matching with the population mortality data, that is age, date
  and sex at the midpoint of each of the intervals (or rater at a
  point 3 months after the left endpoint of the interval --- recall
  we split the follow-up in 6 month intervals).
  We need to have variables of the same type when we merge, so we must
  transform the sex variable in \texttt{M.dk} to a factor, and must
  for each follow-up interval in the \texttt{SL} data have an age and
  a period variable that can be used in merging with the population data. 
\begin{Schunk}
\begin{Sinput}
 str( SL )
 SL$Am <- floor( SL$A+0.25 )
 SL$Pm <- floor( SL$P+0.25 )
 data( M.dk )
 str( M.dk )
 M.dk <- transform( M.dk, Am = A,
                          Pm = P,
                         sex = factor( sex, labels=c("M","F") ) )
 str( M.dk )
\end{Sinput}
\end{Schunk}
Then match the rates from \texttt{M.dk} into \texttt{SL} ---
\texttt{sex}, \texttt{Am} and \texttt{Pm} are the common variables,
and therefore the match is on these variables:
\begin{Schunk}
\begin{Sinput}
 SLr <- merge( SL, M.dk[,c("sex","Am","Pm","rate")] )
 dim( SL )
 dim( SLr )
\end{Sinput}
\end{Schunk}
This merge (remember to \texttt{?merge}!) only takes rows that have
information from both datasets, hence the slightly fewer rows in
\texttt{SLr} than in \texttt{SL}.
\item Compute the expected number of deaths as the person-time
   multiplied by the corresponding population rate, and put it in a
   new variable, \texttt{E}, say (\texttt{E}xpected). Use \texttt{stat.table}
   to make a table of observed, expected and the ratio (SMR) by age
   (suitably grouped) and sex. 
\item Fit a poisson model with sex as the explanatory varable and
  log-expected as offset to derive the SMR (and c.i.).
  Some of the population mortality rates are 0, so you need to exclude
  those records from the analysis.
\begin{Schunk}
\begin{Sinput}
 msmr <- glm( (lex.Xst=="Dead") ~ sex - 1 + offset(log(E)),
              family = poisson,
              data = subset(SLr,E>0) )
 ci.exp( msmr )
\end{Sinput}
\end{Schunk}
  Recogninse the numbers?
\end{enumerate}
\subsection*{Age-specific mortality}
\begin{enumerate}[resume]
\item Now estimate age-specific mortality curves for men and
  women separately, using splines as implemented in \texttt{gam}.
  We use \texttt{k=20} to be sure to catch any irregularities by age.
\begin{Schunk}
\begin{Sinput}
 r.m <- gam( (lex.Xst=="Dead") ~ s(A,k=20) + offset( log(lex.dur) ),
             family = poisson,
             data = subset( SL, sex=="M" ) )
 gam.check( r.f )
\end{Sinput}
\end{Schunk}
Make sure you understand all the components on this modeling statement.
\item Now, extract the estimated rates by using the wrapper function
  \texttt{ci.pred} that computes predicted rates and confidence
  limits for these.
  Note that \texttt{lex.dur} is a covariate in the context of
  prediction; by putting this to 1000 in the prediction dataset we get
  the rates in units of deaths per 1000 PY:
\begin{Schunk}
\begin{Sinput}
 nd <-  data.frame( A = seq(10,90,0.5),
              lex.dur = 1000)
 p.m <- ci.pred( r.m, newdata = nd )
 str( p.m )
\end{Sinput}
\end{Schunk}
\item Plot the predicted rates for men and women together - using for
  example \texttt{matplot}.
\begin{Schunk}
\begin{Sinput}
 p.f <- ci.pred( r.f, newdata = nd )
 matplot( nd$A, cbind(p.m,p.f),
          type="l", col=rep(c("blue","red"),each=3), lwd=c(3,1,1), lty=1,
          log="y", xlab="Age", ylab="Mortality of DM ptt per 1000 PY")
\end{Sinput}
\end{Schunk}
\end{enumerate}
\subsection*{Period and duration effects}
\begin{enumerate}[resume]
\item We now want to model the mortality rates among diabetes patients
  also including current date and duration of diabetes, using penalized
  splines.  Use the argument \texttt{bs="cr"} to \texttt{s()} to get
  cubic splines indstead of thin plate (\texttt{"tp"}) splines which is
  ithe default, and check if you have a reasonable fit:
\begin{Schunk}
\begin{Sinput}
 Mtp <- gam( (lex.Xst=="Dead") ~ s(   A, bs="cr", k=10 ) +
                                 s(   P, bs="cr", k=10 ) +
                                 s( dur, bs="cr", k=10 ),
             offset = log( lex.dur/1000 ),
             family = poisson,
               data = subset( SL, sex=="M" ) )
 summary( Mcr )
 gam.check( Mcr )
\end{Sinput}
\end{Schunk}
Fit the same model for women as well. Are the models reasinable fitting?
\item Plot the estimated effects, using the default plot method for
  \texttt{gam} objects. Remember that there are three effects
  estimated, so it ise useful set up a multi-panel display, and for
  the sake of comparability to set ylim to the same for men and women:
\begin{Schunk}
\begin{Sinput}
 par( mfrow=c(2,3) )
 plot( Mcr, ylim=c(-3,3) )
 plot( Fcr, ylim=c(-3,3) )
\end{Sinput}
\end{Schunk}
\item Compare the fit of the naive model with just age and the
  three-factor models, using \texttt{anova}, e.g.:
\begin{Schunk}
\begin{Sinput}
 anova( Mcr, r.m, test="Chisq" )
\end{Sinput}
\end{Schunk}
\item The model we fitted has three time-scales: current age, current
  date and current duration of diabetes, so the effects that we report
  are not immediately interpretable, as they are (as in any kind of
  multiple regressions) to be interpreted as ``all else equal'' which
  they are not, as the three time scales advance simultaneously at the
  same pace.
  The reporting would therefore more naturally be \emph{only} on the
  mortality scale as a function of age, but showing the mortality
  for persons diagnosed in different ages, using separate displays
  for separate years of diagnosis.
  This is most easily done using the \texttt{ci.pred} function with
  the \texttt{newdata=} argument. So a person diagnosed in age 50 in
  1995 will have a mortality measured in cases per 1000 PY as:
\begin{Schunk}
\begin{Sinput}
 pts <- seq(0,20,1/2)
 nd <- data.frame( A =   50+pts,
                   P = 1995+pts,
                 dur =      pts,
             lex.dur = 1000 )
 m.pr <- ci.pred( Mcr, newdata=nd )
\end{Sinput}
\end{Schunk}
Note however, that if you have used the \texttt{offset=)} argument
in the mdel specification rather than the \texttt{+ offset()} in
the model fromula, the offset specification in \texttt{nd} will be
ignored, and prediction be made for the scale chosen in the model
specification.
  Now take a look at the result from the \texttt{ci.pred} statement and
  construct prediction of mortality for men and women diagnosed in a
  range of ages, say 50, 60, 70, and plot these together in the same
  graph:
\begin{Schunk}
\begin{Sinput}
 cbind( nd, ci.pred( Mcr, newdata=nd ) )
\end{Sinput}
\end{Schunk}
\item From figure \ref{fig:rates} it seems that the duration effect is
dramatically over-modeled, so we refit constraining the d.f. to 5:
\begin{Schunk}
\begin{Sinput}
 Mcr <- gam( (lex.Xst=="Dead") ~ s(   A, bs="cr", k=10 ) +
                                 s(   P, bs="cr", k=10 ) +
                                 s( dur, bs="cr", k=5 ) +
                            offset( log( lex.dur ) ),
             family = poisson,
               data = subset( SL, sex=="M" ) )
 Fcr <- update( Mcr, data = subset( SL, sex=="F" ) )
\end{Sinput}
\end{Schunk}
What do you conclude from the plots?
\end{enumerate}
\subsection{SMR modeling}
\begin{enumerate}[resume]
\item Now model the SMR using age and date of diagnosis and diabetes
  duration as explanatory variables, including the log-expected-number
  instead of the log-person-years as offset, using separate models for
  men and women. 
  You can re-use the code you used for fitting models for the rates,
  you only need to use the expedtd numbers instead of the
  person-years. But remember to exclude those units where no deaths
  in the population occur (that is where the rate is 0) --- an
  offset of $-\infty$ will crash \texttt{gam}.
  Plot the estimated smooth effects from both models using
  e.g. \texttt{plot.gam}. What do you see?
\item Plot the predicted SMRs from the models for men and women
  diagnosed in ages 50, 60 and 70 as you dif for the rates. What do
  you see?
\item Try to simplify the model to one with a simple sex effect,
  linear effects of age and date of follow-up, and a smooth effect of
  duration, giving an estimate of the change in SMR by age and
  calendar time. How much does SMR change by each year of age? And by
  each calendar year?
\item Use your previous code to plot the predicted mortality from this
  model too. Are the predicted SMR curves credible?
\item (\emph{optional}) We may deem the curves non-credible and
  ultimately resort to a brutal parametric assumption without any
  penalty. If we choose a natural spline for the duration with knost
  at 0,1,3,6 years we get a model with 3 parameters, try:
\begin{Schunk}
\begin{Sinput}
 dim( Ns(SLr$dur, knots=c(0,1,3,6) ) )
\end{Sinput}
\end{Schunk}
  Now fit the same model (also ignoring sex) as above using this:
\begin{Schunk}
\begin{Sinput}
 SMRglm <- glm( (lex.Xst=="Dead") ~ I(A-60) + 
                                    I(P-2000) + 
                                    Ns( dur, knots=c(0,1,4,8) ) +
                                    offset( log(E) ),
                family = poisson,
                  data = SLr )
\end{Sinput}
\end{Schunk}
  Plot the estimated SMRs from this model as before, and give a
  conclusion on SMR for diabetes patients.
\end{enumerate}
