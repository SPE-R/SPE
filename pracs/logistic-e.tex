

\section{Logistic regression (GLM)}

\subsection{Malignant melanoma in Denmark}

In the mid-80s a case-control study on
risk factors for malignant melanoma was conducted in Denmark
({\O}sterlind et al. The Danish case-control study of cutaneous
malignant melanoma I: Importance of host factors.
 \emph{Int J Cancer} 1988; 42: 200-206).


The cases were patients with skin melanoma
(excluding lentigo melanoma),
newly diagnosed from 1 Oct, 1982 to 31 March, 1985, aged 20-79, from 
East Denmark, and they were identified from the Danish Cancer Registry.

The controls (twice as many as cases) 
were drawn from the residents of East Denmark
in April, 1984, as a random sample stratified by
sex and age (within the same 5 year age group) to reflect the 
sex and age distribution of the cases. This is called group matching,
and in such a study, it is necessary to control for age and sex in the statistical analysis. 
(Yes indeed: In spite of the fact that
stratified sampling by sex and age removed the statistical association of these
variables with melanoma from the final case-control data set,
 the analysis must control for
variables which determine the probability of selecting 
 subjects from the base population to the study sample.)
 
The population of East Denmark is a dynamic one. Sampling the
controls only at one time point is a rough approximation of
{\it incidence density sampling}, which ideally would spread out over the whole study period. Hence
the exposure odds ratios calculable from the data are estimates
of the corresponding hazard rate ratios between the exposure groups. 

After exclusions, refusals etc., 
474 cases (92\% of eligible cases) 
and 926 controls (82\%) were interviewed. This was done face-to-face
with a structured questionnaire
by trained interviewers, who were not 
informed about the subject's case-control status.

For this  exercise we have selected a few host variables from the
study in an ascii-file, \texttt{melanoma.dat}. The variables are listed
in table \ref{tab:meldat}.

\begin{table}[h]
\begin{center}
\caption{\label{tab:meldat}
  \it Variables in the melanoma dataset.} 
\begin{tabular}{@{\extracolsep{1em}}llll}\\
{\bf Variable} & {\bf Units or Coding} & {\bf Type} & {\bf Name}\\
\\ \hline \\
 Case-control status &  1=case, 0=control & numeric & {\tt cc} \\
 Sex & 1=male, 2=female & numeric & {\tt sex} \\
 Age at interview & age in years & numeric & {\tt age} \\
 Skin complexion & 0=dark, 1=medium, 2=light & numeric & {\tt skin} \\
 Hair colour & 0=dark brown/black, 1=light brown, \\
             & 2=blonde, 3=red & numeric & {\tt hair} \\
 eye colour & 0=brown, 1=grey, green, 2=blue & numeric & {\tt eyes} \\
 Freckles & 1=many, 2=some, 3=none & numeric & {\tt freckles} \\
 Naevi, small & no. naevi $<5$mm & numeric & {\tt nvsmall} \\
 Naevi, largs & no. naevi $\ge 5$mm & numeric & {\tt nvlarge} \\ \\
\hline
\end{tabular}
\end{center}
\end{table}

\subsection{Reading the data}
Start R and
load the {\tt Epi} package using the function {\tt library()}. 
Read the data set from the file {\tt melanoma.dat} found in the course website
%% (this should be in the subdirectory data of your working directory) 
to a data frame with name {\tt mel} using the {\tt read.table()} function.
Remember to specify that missing values are coded ''{\tt .}'', and 
that variable names are in the first line of the file.
View the overall structure of the data frame, and list the first 20 rows of {\tt mel}.

\begin{Schunk}
\begin{Sinput}
> library(Epi)
> mel <- read.table("http://bendixcarstensen.com/SPE/data/melanoma.dat", header=TRUE, na.strings=".")
> str(mel)
> head(mel, n=20)
\end{Sinput}
\end{Schunk}

\subsection{House keeping}
The structure of the data frame {\tt mel} tells us that all the variables are numeric (integer), so first you need to do a bit of house keeping. For example the variables {\tt sex, skin, hair, eye} need to be converted to factors, with labels, and {\tt freckles} which is coded 4 for none down to 1 for many (not very intuitive) needs to be recoded, and relabelled.

To avoid too much typing and to leave plenty of time to think about the analysis, these house keeping commands are in a script file called {\tt melanoma-house.r}. You should study this script carefully before running it.
The coding of {\tt freckles} can be reversed by subtracting the current codes from 4. Once recoded the variable needs to be converted to a factor with labels "none", etc. Age is currently a numeric variable recording age to the nearest year, and it will be convenient to group these values into (say) 10 year age groups, using {\tt cut}. In this case we choose to create a new variable, rather than change the original.

\begin{Schunk}
\begin{Sinput}
> source("http://bendixcarstensen.com/SPE/data/melanoma-house.r")
\end{Sinput}
\end{Schunk}

Look again at the structure of the data frame {\tt mel} and note the changes. Use the command {\tt summary(mel)} to look at the univariate distributions.


This is enough housekeeping for now - let's turn to something a bit more interesting.

\subsection{One variable at a time}
As a first step it is a good idea to start by looking at the numbers of cases and controls by each variable separately, ignoring age and sex. Try
\begin{Schunk}
\begin{Sinput}
> with(mel, table(cc,skin))
> stat.table(skin, contents=ratio(cc,1-cc), data=mel)
\end{Sinput}
\end{Schunk}
to see the numbers of cases and controls, as well as the 
odds of being a case by skin colour

Now use \texttt{effx()} to get crude estimates of the hazard ratios for the effect of skin colour. 
\begin{Schunk}
\begin{Sinput}
> effx(cc, type="binary", exposure=skin, data=mel)
\end{Sinput}
\end{Schunk}


\medskip
$\bullet$ Look at the crude effect estimates of {\tt hair}, {\tt eyes} and 
{\tt freckles} in the same way.

\subsection{Generalized linear models with binomial family and logit link}


The function {\tt effx()} is just a wrapper for the {\tt glm()} function, and you can show this by fitting the glm directly with
\begin{Schunk}
\begin{Sinput}
> mf <- glm(cc ~ freckles, family="binomial", data=mel)
> round(ci.exp( mf ),2)
> 
\end{Sinput}
\end{Schunk}
Comparison with the output from {\tt effx()} shows the results to be the same. 

Note that in {\tt effx()} the type of response is ``binary'' whereas in {\tt glm()} the family of probability distributions used to fit the model is ``binomial''. There is a 1-1 relationship between type of response and family:

\begin{tabular}{ll}
metric & gaussian \\
binary & binomial\\
failure/count & poisson
\end{tabular}
\subsection{Controlling for age and sex}
Because the probability that a control is selected into the study depends on age and sex it is necessary to control for age and sex. For example, the effect of freckles controlled for age and sex is obtained with
\begin{Schunk}
\begin{Sinput}
> effx(cc, typ="binary", exposure=freckles, control=list(age.cat,sex),data=mel)
\end{Sinput}
\end{Schunk}
or
\begin{Schunk}
\begin{Sinput}
> mfas <- glm(cc ~ freckles + age.cat + sex, family="binomial", data=mel)
> round(ci.exp(mfas), 2)
\end{Sinput}
\end{Schunk}
Do the adjusted estimates differ from the crude ones
 that you computed with \texttt{effx()}?

\subsection{Likelihood ratio tests}

There are 2 effects estimated for the 3 levels of {\tt freckles}, and {\tt glm()} provides a test for each effect separately, but to test for no effect at all of {\tt freckles} you need a likelihood ratio test. This involves fitting two models, one without {\tt freckles} and one with, and recording the change in deviance. Because there are some missing values for freckles it is necessary to restrict the first model to those subjects who have values for freckles.
\begin{Schunk}
\begin{Sinput}
> mas <- glm(cc ~ age.cat + sex, family="binomial", data=subset(mel, !is.na(freckles)) )
> anova(mas, mfas, test="Chisq")
\end{Sinput}
\end{Schunk}
The change in residual deviance is $1785.9-1737.1=48.8$ on $1389-1387=2$ degrees of freedom.
The $P$-value corresponding to this change is obtained from the upper tail of the
cumulative distribution of the $\chi^2$-distribution with 2 df:
\begin{Schunk}
\begin{Sinput}
> 1 - pchisq(48.786, 2)
\end{Sinput}
\end{Schunk}

\medskip
$\bullet$
There are 3 effects for the 4 levels of hair colour ({\tt hair}). 
To obtain adjusted estimates for the effect of hair colour and to test the pertinent null hypothesis fit the relevant models, print the and use {\tt anova} to test for no effects of hair colour.
Compare the estimates with the crude ones and assess the evidence against the null hypothesis. 

\subsection{Relevelling}

From the above you can see that subjects at each of the 3 levels light-brown, blonde, and red, are at greater risk than subjects with dark hair, with similar odds ratios. This suggests creating a new variable {\tt hair2} which has just two levels, dark and the other three. The {\tt Relevel()} function in \texttt{Epi} has been used for this in the house keeping script.

\medskip
$\bullet$ Use \texttt{effx()} to compute the odds-ratio of melanoma
  between persons with red, blonde or light brown hair versus those with dark  hair.
Reproduce these results by fitting an appropriate glm.
Use also a likelihood ratio test to test for the effect of \texttt{hair2}.

\subsection{Controlling for other variables}

When you control the effect of an exposure for some variable you are asking a question about what would the effect be if the variable is kept constant. For example, consider the effect of {\tt freckles} controlled for {\tt hair2}. We first stratify by {\tt hair2} with
\begin{Schunk}
\begin{Sinput}
> effx(cc, type="binary", exposure=freckles, 
+                   control=list(age.cat,sex), strata=hair2, data=mel)
\end{Sinput}
\end{Schunk}
The effect of freckles is still apparent in each of the two strata for hair colour. Use {\tt effx()} to control for {\tt hair2}, too, in addition to \texttt{age.cat} and \texttt{sex}.
\begin{Schunk}
\begin{Sinput}
> effx(cc, type="binary", exposure=freckles, 
+                   control=list(age.cat,sex,hair2), data=mel)
\end{Sinput}
\end{Schunk}
It is tempting to control for variables without thinking about the question you are thereby asking. This can lead to nonsense.

\subsection{Stratification using \texttt{glm()} }
We shall reproduce the output from 
\begin{Schunk}
\begin{Sinput}
> effx(cc, type="binary", exposure=freckles, 
+            control=list(age.cat,sex), strata=hair2,data=mel)
\end{Sinput}
\end{Schunk}
using \texttt{glm()}. To do this requires a nested model formula:
\begin{Schunk}
\begin{Sinput}
> mfas.h <- glm(cc ~ hair2/freckles + age.cat + sex, family="binomial", data=mel)
> ci.exp(mfas.h )
\end{Sinput}
\end{Schunk}
In amongst all the other effects you can see the two effects of freckles for dark hair (1.61 and 2.84) and the two effects of freckles for other hair (1.42 and 3.15). 

\subsection{Naevi}
The distributions of {\tt nvsmall} and {\tt nvlarge} are very skew to the right. You can see this with
\begin{Schunk}
\begin{Sinput}
> with(mel, stem(nvsmall))
> with(mel, stem(nvlarge))
\end{Sinput}
\end{Schunk}
Because of this it is wise to categorize them into a few classes 
 \begin{itemize}
\item[--] small naevi into four: 0, 1, 2-4, and 5+; 
\item[--] large naevi into three: 0, 1, and 2+. 
 \end{itemize}
This has been done in
the house keeping script.

\medskip
$\bullet$
Look at the joint frequency distribution of these new variables
using {\tt with(mel, table( ))}. Are they strongly associated?

\medskip
$\bullet$ 
Compute the sex- and age-adjusted OR estimates (with 95\% CIs) associated with
the number of small naevi first by using {\tt effx()}, and 
then by fitting separate glms including {\tt sex}, {\tt age.cat} and {\tt nvsma4} in the model formula.

\medskip
$\bullet$
Do the same with large naevi {\tt nvlar3}.

\medskip
$\bullet$
Now fit a glm containing
{\tt age.cat, sex, nvsma4} and {\tt nvlar3}. 
What is the interpretation of the coefficients for {\tt nvsma4} and {\tt nvlar3}?

\subsection{Treating freckles as a numeric exposure}  

 The evidence for the effect of {\tt freckles} is already convincing. However,
 to demonstrate how it is done, we shall perform a linear trend test by treating freckles as a numeric exposure. 
\begin{Schunk}
\begin{Sinput}
> mel$fscore<-as.numeric(mel$freckles)
> effx(cc, type="binary", exposure=fscore, control=list(age.cat,sex), data=mel)
\end{Sinput}
\end{Schunk}

You can check for linearity of the log odds of being a case with {\tt fscore} by comparing the model containing {\tt freckles} as a factor with the model containing {\tt freckles} as numeric.
\begin{Schunk}
\begin{Sinput}
> m1 <- glm(cc ~ freckles + age.cat + sex, family="binomial", data=mel)
> m2 <- glm(cc ~ fscore + age.cat + sex, family="binomial", data=mel)
> anova(m2, m1, test="Chisq")
\end{Sinput}
\end{Schunk}
There is no evidence against linearity ($p=0.22$).

It is sometimes helpful to look at the linearity in more detail with
\begin{Schunk}
\begin{Sinput}
> m1 <- glm(cc ~ C(freckles, contr.cum) + age.cat + sex, family="binomial",data=mel)
> round(ci.exp(m1 ), 2)
> m2 <- glm(cc ~ fscore + age.cat + sex, family="binomial",data=mel)
> round(ci.exp(m2), 2)
\end{Sinput}
\end{Schunk}
The use of {\tt C(freckles, contr.cum)} makes each odds ratio to compare the odds at that
 level versus the previous level; not against the baseline (except for the 2nd level). 
If the log-odds are linear then these odds ratios should be the same
 (and the same as the odds ratio for {\tt fscore} in m2.

\subsection{Graphical displays}
   
  The odds ratios (with CIs) can be graphically displayed using
  function {\tt plotEst()} in {\tt Epi}. It uses the value
 of {\tt ci.lin()} evaluated on the  fitted model object.
 As the intercept and the effects of age and sex are of
 no interest, we shall drop the corresponding rows (the 7 first ones)
 from the matrix produced by {\tt ci.lin()}, and
  the plot is based just on the 1st, 5th and the 6th column
 of this matrix:

\begin{Schunk}
\begin{Sinput}
> m <- glm(cc ~ nvsma4 + nvlar3 + age.cat + sex, family="binomial",data=mel)
> plotEst( exp( ci.lin(m)[ 2:5, -(2:4)] ), xlog=T, vref=1 )
\end{Sinput}
\end{Schunk}
The {\tt xlog} argument makes the OR axis logarithmic.



