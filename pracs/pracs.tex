\documentclass[a4paper,dvipsnames,twoside,12pt]{report}

% ----------------------------------------------------------------------
% General information for the title page and the page headings
\newcommand{\Title}{Statistical Practice in Epidemiology\\ with
     \includegraphics[height=1em,keepaspectratio]{../adm/Rlogo}\\
      Computer exercises}
\newcommand{\Tit}{SPE practicals}
%\newcommand{\Version}{2}
\newcommand{\Dates}{14--20 June 2018}
\newcommand{\Where}{IARC Lyon}
\newcommand{\Homepage}{\url{http://bendixcarstensen.com/SPE} }
% This years draw:
% > sample( c("B","K","J","E","M") )
% [1] "J" "B" "E" "K" "M"
\newcommand{\Faculty}{\begin{tabular}{rl}
Janne Pitk\"{a}niemi
  & Finnish Cancer Registry, Helsinki, Finland\\
  & \texttt{janne.pitkaniemi@cancer.fi}\\[1ex]
Bendix Carstensen
  & Steno Diabetes Center Copenhagen, Gentofte, Denmark \\
  & \& Dept. of Biostatistics, University of Copenhagen, Denmark\\
  & \texttt{b@bxc.dk} \\
  & \url{http://BendixCarstensen.com}\\[1ex]
Esa L\"{a}\"{a}r\"{a}
  & Department of Mathematical Sciences, University of Oulu, Finland\\
  & \texttt{Esa.Laara@oulu.fi} \\
  & \url{http://www.oulu.fi/university/researcher/esa-laara}\\[1ex]
Krista Fischer
  & Estonian Genome Center, University of Tartu, Estonia. \\
  & \texttt{Krista.Fischer@ut.ee} \\[1ex]
Martyn Plummer
  & International Agency for Research on Cancer, Lyon, France\\
  & \texttt{plummerm@iarc.fr} \\[1ex]
\end{tabular}}

\setcounter{tocdepth}{1}
% Stuff to put in at the begiing of the document
%----------------------------------------------------------------------
% Packages
\usepackage[utf8]{inputenc}
\usepackage[T1]{fontenc}
\usepackage[english]{babel}
\usepackage[font=it,labelfont=normalfont]{caption}
\usepackage[colorlinks,urlcolor=blue,linkcolor=red,citecolor=Maroon]{hyperref}
% \usepackage[ae,hyper]{Rd}
\usepackage[dvipsnames]{xcolor}
\usepackage[super]{nth}
\usepackage[noae]{Sweave}
% \usepackage[noae]{C:/util/R/R-4.3.0/share/texmf/tex/latex/Sweave}
\usepackage{makeidx,floatflt,amsmath,amsfonts,amsbsy,enumitem,dcolumn,needspace}
\usepackage{ifthen,calc,eso-pic,everyshi}
\usepackage{booktabs,longtable,rotating,graphicx,subfig}
\usepackage{pdfpages,verbatim,fancyhdr,datetime,afterpage}
\usepackage[abspath]{currfile}
\renewcommand{\textfraction}{0.0}
\renewcommand{\topfraction}{1.0}
\renewcommand{\bottomfraction}{1.0}
\renewcommand{\floatpagefraction}{0.9}
\definecolor{blaa}{RGB}{99,99,255}
\DeclareGraphicsExtensions{.png,.pdf,.jpg}
% Make the Sweave output nicer
\DefineVerbatimEnvironment{Sinput}{Verbatim}{fontsize=\small,fontshape=sl,formatcom=\color{BlueViolet}}
\DefineVerbatimEnvironment{Soutput}{Verbatim}{fontsize=\small,formatcom=\color{BrickRed},xleftmargin=0em}
\DefineVerbatimEnvironment{Scode}{Verbatim}{fontsize=\small}
\fvset{listparameters={\setlength{\topsep}{-0.1ex}}}
\renewenvironment{Schunk}%
{\renewcommand{\baselinestretch}{0.85} \vspace{\topsep}}%
{\renewcommand{\baselinestretch}{1.00} \vspace{\topsep}}
% \renewenvironment{knitrout}
% {\renewcommand{\baselinestretch}{0.85}}
% {\renewcommand{\baselinestretch}{1.00}}

%----------------------------------------------------------------------
% The usual usefuls
% This is a file of useful extra commands snatched from
% Michael Hills, David Clayton, Bendix Carstensen & Esa Laara.
%

% Commands to draw observation lines on follow-up diagrams
%
% Horizontal lines
%
\providecommand{\hfail}[1]{\begin{picture}(250,5)
      \put(0,0){\line(1,0){#1}}
      \put(#1,0){\circle*{5}}
   \end{picture}}

\providecommand{\hcens}[1]{\begin{picture}(250,5)
      \put(0,0){\line(1,0){#1}}
      \put(#1,0){\line(0,1){2.5}}
      \put(#1,0){\line(0,-1){2.5}}
   \end{picture}}

%
% Diagonals for Lexis diagrams
%
\providecommand{\dfail}[1]{\begin{picture}(250,250)
      \put(0,0){\line(1,1){#1}}
      \put(#1,#1){\circle*{5}}
   \end{picture}}

\providecommand{\dcens}[1]{\begin{picture}(250,250)
      \put(0,0){\line(1,1){#1}}
%      \put(#1,#1){\line(0,1){2.5}}
%      \put(#1,#1){\line(0,-1){2.5}}
% BxC Changed this to an open circle instead of a line
      \put(#1,#1){\circle{5}}
   \end{picture}}

%
% Horizontal range diagrams
%
\providecommand{\hrange}[1]{\begin{picture}(200,5)
     \put(0,0){\circle*{5}}
     \put(0,0){\line(1,0){#1}}
     \put(0,0){\line(-1,0){#1}}
   \end{picture}}

%
% Tree drawing
%
\providecommand{\Tree}[3]{\setlength{\unitlength}{#1\unitlength}\begin{picture}(0,0)
   \put(0,0){\line(3, 2){1}}
   \put(0,0){\line(3,-2){1}}
   \put(0.81, 0.54){\makebox(0,0)[br]{\footnotesize #2\ }}
   \put(0.81,-0.54){\makebox(0,0)[tr]{\footnotesize #3\ }}
\end{picture}}

\providecommand{\Wtree}[3]{\setlength{\unitlength}{#1\unitlength}\begin{picture}(0,0)
   \put(0,0){\line(1, 1){1}}
   \put(0,0){\line(1,-1){1}}
   \put(0.8,0.8){\makebox(0,0)[br]{\footnotesize #2\ }}
   \put(0.8,-0.8){\makebox(0,0)[tr]{\footnotesize #3\ }}
\end{picture}}

\providecommand{\Ntree}[3]{\setlength{\unitlength}{#1\unitlength}\begin{picture}(0,0)
   \put(0,0){\line(2, 1){1}}
   \put(0,0){\line(2,-1){1}}
   \put(0.8,0.4){\makebox(0,0)[br]{\footnotesize #2\ }}
   \put(0.8,-0.4){\makebox(0,0)[tr]{\footnotesize #3\ }}
\end{picture}}

\providecommand{\Nutree}[3]{\setlength{\unitlength}{#1\unitlength}\begin{picture}(0,0)
   \put(0,0){\line(2, 1){#1}}
   \put(0,0){\line(2,-1){#1}}
   \put(0.8,0.4){\makebox(0,0)[br]{#2\ }}
   \put(0.8,-0.4){\makebox(0,0)[tr]{#3\ }}
\end{picture}}

%
% Tree drawing
%
\providecommand{\tree}[3]{\setlength{\unitlength}{#1}\begin{picture}(0,0)
   \put(0,0){\line(3,2){1}}
   \put(0,0){\line(3,-2){1}}
   \put(0.81,0.54){\makebox(0,0)[br]{\footnotesize #2\ }}
   \put(0.81,-0.54){\makebox(0,0)[tr]{\footnotesize #3\ }}
\end{picture}}

\providecommand{\wtree}[3]{\setlength{\unitlength}{#1}\begin{picture}(0,0)
   \put(0,0){\line(1,1){1}}
   \put(0,0){\line(1,-1){1}}
   \put(0.8,0.8){\makebox(0,0)[br]{\footnotesize #2\ }}
   \put(0.8,-0.8){\makebox(0,0)[tr]{\footnotesize #3\ }}
\end{picture}}

\providecommand{\ntree}[3]{\setlength{\unitlength}{#1}\begin{picture}(0,0)
   \put(0,0){\line(2,1){1}}
   \put(0,0){\line(2,-1){1}}
   \put(0.8,0.4){\makebox(0,0)[br]{\footnotesize #2\ }}
   \put(0.8,-0.4){\makebox(0,0)[tr]{\footnotesize #3\ }}
\end{picture}}

\providecommand{\nutree}[3]{\begin{picture}(0,0)
   \put(0,0){\line(2,1){#1}}
   \put(0,0){\line(2,-1){#1}}
   \put(0.8,0.4){\makebox(0,0)[br]{#2\ }}
   \put(0.8,-0.4){\makebox(0,0)[tr]{#3\ }}
\end{picture}}

%
% Other commands
%
\providecommand{\prob}[0]{\text{\rm Pr}}
\providecommand{\nhy}[0]{_{\oslash}}
\providecommand{\true}[0]{_{\text{\rm \tiny T}}}
\providecommand{\hyp}[0]{_{\text{\rm \tiny H}}}
% \providecommand{\mpydiv}[0]{\stackrel{\textstyle \times}{\div}}
% Changed to slightly smaller symbols
\providecommand{\mpydiv}[0]{\stackrel{\times}{\scriptstyle \div}}
\providecommand{\mie}[1]{{\it #1}}
\providecommand{\mycircle}[0]{\circle*{5}}
\providecommand{\smcircle}[0]{\circle*{1}}
\providecommand{\corner}[0]{_{\text{\rm \tiny C}}}
\providecommand{\ind}[0]{\hspace{10pt}}
\providecommand{\gap}[0]{\\[5pt]}
\renewcommand{\S}[0]{section~}
\providecommand{\blank}[0]{$\;\,$}
\providecommand{\vone}{\vspace{1cm}}
\providecommand{\ljust}[1]{\multicolumn{1}{l}{#1}}
\providecommand{\cjust}[1]{\multicolumn{1}{c}{#1}}
\providecommand{\Var}{\text{\rm var}}
\providecommand{\cov}{\text{\rm cov}}
\providecommand{\corr}{\text{\rm corr}}
\providecommand{\mean}{\text{\rm mean}}
\providecommand{\median}{\text{\rm median}}
\providecommand{\transpose}{^{\text{\sf T}}}
\providecommand{\histog}[5]{\rule{1mm}{#1mm}\,\rule{1mm}{#2mm}\,\rule{1mm}{#3mm}\,\rule{1mm}{#4mm}\,\rule{1mm}{#5mm}}
\providecommand{\pmiss}{P_{\mbox{\tiny miss}}}

% Below is BxCs commands inserted

\providecommand{\bc}{\begin{center}}
\providecommand{\ec}{\end{center}}
\providecommand{\bd}{\begin{description}}
\providecommand{\ed}{\end{description}}
\providecommand{\bi}{\begin{itemize}}
\providecommand{\ei}{\end{itemize}}
\providecommand{\bn}{\begin{equation}}
\providecommand{\en}{\end{equation}}
\providecommand{\be}{\begin{enumerate}}
\providecommand{\ee}{\end{enumerate}}
\providecommand{\bes}{\begin{eqnarray*}}
\providecommand{\ees}{\end{eqnarray*}}

\providecommand{\p}{{\mathrm p}}
\providecommand{\e}{{\mathrm e}}
\providecommand{\D}{{\mathrm D}}
\providecommand{\dif}{{\,\mathrm d}}
\providecommand{\pmat}[1]{\text{\rm P}\left\{#1\right\}}
\providecommand{\ptxt}[1]{\text{\rm P}\left\{\text{#1}\right\}}
\providecommand{\E}{\text{\rm E}}
\providecommand{\V}{\text{\rm V}}
\providecommand{\BLUP}{\text{\rm BLUP}}
\providecommand{\se}{\text{\rm s.e.}}
\providecommand{\sem}{\text{\rm s.e.m.}}
\providecommand{\std}{\text{\rm std}}
\providecommand{\sd}{\text{\rm s.d.}}
\providecommand{\cv}{\text{\rm c.v.}}
\providecommand{\erf}{\text{\rm erf}}
\providecommand{\ef}{\text{\rm ef}}
\providecommand{\SSD}{\text{\rm SSD}}
\providecommand{\SPD}{\text{\rm SPD}}
\providecommand{\odds}{\text{\rm odds}}
\providecommand{\bin}{\text{\rm binom}}
\providecommand{\half}{\frac{1}{2}}
% \providecommand{\td}[0]{\stackrel{\textstyle \times}{\div}}
% Changed to slightly smaller symbols
\providecommand{\td}[0]{\stackrel{\scriptstyle \times}{\scriptstyle \div}}
\providecommand{\dt}[0]{\stackrel{\scriptstyle \div}{\scriptstyle \times}}
\providecommand{\diag}{\text{\rm diag}}
\providecommand{\spcol}{\text{\rm span}}
\providecommand{\logit}{\text{\rm logit}}
% \providecommand{\link}{\text{\rm link}}
\providecommand{\spn}{\text{\rm span}}
\providecommand{\CI}{\text{\rm CI}}
\providecommand{\IP}{\text{\rm IP}}
\providecommand{\OR}{\text{\rm OR}}
\providecommand{\RR}{\text{\rm RR}}
\providecommand{\ER}{\text{\rm ER}}
\providecommand{\EM}{\text{\rm EM}}
\providecommand{\EF}{\text{\rm EF}}
\providecommand{\RD}{\text{\rm RD}}
\providecommand{\AC}{\text{\rm AC}}
\providecommand{\AF}{\text{\rm AF}}
\providecommand{\PAF}{\text{\rm PAF}}
\providecommand{\AR}{\text{\rm AR}}
\providecommand{\CR}{\text{\rm CR}}
\providecommand{\PAR}{\text{\rm PAR}}
\providecommand{\SD}{\text{\rm SD}}
\providecommand{\SE}{\text{\rm SE}}
\providecommand{\SEM}{\text{\rm SEM}}
\providecommand{\SR}{\text{\rm SR}}
\providecommand{\SMR}{\text{\rm SMR}}
\providecommand{\RSR}{\text{\rm RSR}}
\providecommand{\CMF}{\text{\rm CMF}}
\providecommand{\pvp}{\text{\rm PV$+$}}
\providecommand{\pvn}{\text{\rm PV$-$}}
\providecommand{\R}{\textsf{R}}
\providecommand{\sas}{\textsl{\textbf{SAS}}}
\providecommand{\SAS}{\textsl{\textbf{SAS}}}
%\providecommand{\gap}[0]{\\[5pt]}
%\providecommand{\blank}[0]{$\;\,$}
% Conditional independence sign from Philip Dawid
\providecommand{\cip}{\mbox{$\perp\!\!\!\perp$}}

%%% Commands to comment out parts of the text
\providecommand{\GLEM}[1]{}
\providecommand{\FORGETIT}[1]{}
\providecommand{\OMIT}[1]{}

%%% Insert output from program in small text
%%% (requires package verbatim)
\providecommand{\insout}[1]{
 \scriptsize
 \renewcommand{\baselinestretch}{0.8}
 \verbatiminput{#1}
 \renewcommand{\baselinestretch}{1.0}
 \normalsize
}
\providecommand{\insouttiny}[1]{
\tiny
\renewcommand{\baselinestretch}{0.8}
\verbatiminput{#1}
\renewcommand{\baselinestretch}{1.0}
\normalsize
}

% From Esa:
\providecommand{\T}{\text{\rm \small{T}}}
\providecommand{\id}{\text{\rm id}}
\providecommand{\Dev}{\text{\rm Dev}}
\providecommand{\Bin}{\text{\rm Bin}}
\providecommand{\probit}{\text{\rm probit}}
\providecommand{\cloglog}{\text{\rm cloglog}}

% Special commands to include output from R, Bugs and Stata

\providecommand{\Rin}[2]{
\subsection{\texttt{#1.R}}
#2

\insout{./R/#1.Rout}

}

\providecommand{\Statain}[2]{
\subsection{\texttt{#1.do}}
#2

\insout{./do/#1.log}

}

\providecommand{\Bugsin}[2]{
\subsection{\texttt{#1.bug}}
#2

\insout{./bugs/#1.bug}

}

\newlength{\wdth}
\providecommand{\fxbl}[1]{\settowidth{\wdth}{#1} \makebox[\wdth]{}}

%%% Local Variables:
%%% mode: latex
%%% TeX-master: t
%%% End:

\newcommand{\code}[1]{\textcolor{ForestGreen}{\texttt{#1}}}

%----------------------------------------------------------------------
% Set up layout of pages
\oddsidemargin -5mm
\evensidemargin -5mm
\topmargin -10mm
\headheight 8mm
\headsep 5mm
\textheight 240mm
\textwidth 170mm
%\footheight 5mm
\footskip 15mm
\renewcommand{\topfraction}{0.9}
\renewcommand{\bottomfraction}{0.9}
\renewcommand{\textfraction}{0.1}
\renewcommand{\floatpagefraction}{0.9}
\renewcommand{\headrulewidth}{0.1pt}
\setcounter{secnumdepth}{4}
% \setcounter{tocdepth}{2}

%----------------------------------------------------------------------
% How to insert a figure in a .rnw file
\newcommand{\rwpre}{./graph/gr}
\newcommand{\insfig}[3]{
\begin{figure}[h]
  \centering
  \includegraphics[width=#2\textwidth]{\rwpre-#1}
  \caption{#3}
  \label{fig:#1}
% \afterpage{\clearpage}
\end{figure}}
\newcommand{\linput}[1]{
% \clearpage 
\afterpage{\hfill \ldots input from \texttt{#1.tex}} 
\fancyfoot[OR,EL]{\footnotesize \texttt{#1.tex}} 
\input{#1}}

%----------------------------------------------------------------------
% Here is the document starting with the titlepage
\begin{document}

%----------------------------------------------------------------------
% The title page
\setcounter{page}{1}
\pagenumbering{roman}
\pagestyle{plain}
\thispagestyle{empty}
% \vspace*{0.05\textheight}
\flushright
% The blank below here is necessary in order not to muck up the
% linespacing in title if it has more than 2 lines
{\Huge \bfseries \Title

}\ \\[-1.5ex]
\noindent\textcolor{blaa}{\rule[-1ex]{\textwidth}{5pt}}\\[2.5ex]
\large
\Where \\
\Dates \\
\Homepage \\
%\Version \\[1em]
\normalsize
Compiled \today,\ \currenttime\\
from: \texttt{\currfileabspath}\\[1em]
% \input{wordcount}
\normalsize
\vfill
\Faculty
% End of titlepage
% \newpage

%----------------------------------------------------------------------
% Table of contents
% \listoftables
% \listoffigures
\tableofcontents
\clearpage

%----------------------------------------------------------------------
% General text layout
\raggedright
\parindent 1em
\parskip 0ex
\cleardoublepage

%----------------------------------------------------------------------
% General page style
\pagenumbering{arabic}
\setcounter{page}{1}
\pagestyle{fancy}
\renewcommand{\chaptermark}[1]{\markboth{\textsl{#1}}{}}
\renewcommand{\sectionmark}[1]{\markright{\thesection\ \textsl{#1}}{}}
\fancyhead[EL]{\bf \thepage \quad \rm \rightmark}
\fancyhead[ER]{\rm \Tit}
\fancyhead[OL]{\rm \leftmark}
\fancyhead[OR]{\rm \rightmark \quad \bf \thepage}
\fancyfoot{}

\renewcommand{\rwpre}{./prefix}
% command to include an exercise/solution with an initial pagethrow etc.
\newcommand{\pput}[2]{\clearpage
                   \fancyhead[ER]{\sl SPE: \ExcSol}
                   \fancyhead[OL]{\sl IARC, 2018}
                   \fancyhead[OR]{\sl \rightmark \quad \bf \thepage}
                   \fancyhead[EL]{\bf \thepage \quad \sl \rightmark}
                   % \texttt{#1} ## write the name of the input file
                   \input{#1}}

\newpage
\addcontentsline{toc}{section}{Program}

\section*{Program}
\noindent
\begin{tabular}{r@{ -- }rp{13cm}}
\multicolumn{3}{l}{\bf Daily timetable} \\
 9:00 &  9:30 & Recap of yesterday's practicals \\
 9:30 & 10:30 & Lecture \\
10:30 & 10:50 & Coffee break \\
10:50 & 12:50 & Practical \\
12:50 & 14:00 & Lunch \\
14:00 & 14:30 & Recap of morning's practical \\
14:30 & 15:30 & Lecture \\
15:30 & 16:00 & Tea break \\
16:00 & 18:00 & Practical \\[2em]
\end{tabular}

\noindent
\begin{tabular}{r@{ -- }rp{12cm}}
\multicolumn{3}{l}{\bf Thursday 1 June} \\
 9:00 &  9:15 & Welcome (KF) \\
 9:15 & 10:30 & Introduction to R language and commands reading data (MP) \\
10:30 & 10:50 & Coffeee break \\
10:50 & 12:50 & Practical:
                Practice with basic \R \newline
                Simple reading and data input \\
12:50 & 14:00 & Lunch \\
14:00 & 14:30 & Recap of morning practical \\
14:30 & 15:30 & Language, indexing,
                {\tt subset()}, {\tt ifelse()},
                \texttt{attach(), detach()},
                \texttt{search}. Simple simulation. Simple graphics. (KF)\\
15:30 & 16:00 & Tea break\\
16:00 & 18:00 & Practical: Simple simulation \newline
                Tabulation\newline
                Introduction to graphs in R \\
18:00 & 19:00 & Tour of the genome center before the \\
19:00 & 21:00 & Welcome reception at the \newline
\href{https://www.google.dk/maps/place/Riia+23b,+51010+Tartu,+Estonia/@58.3728901,26.7157088,17z/data=!3m1!4b1!4m5!3m4!1s0x46eb371f8605d0f9:0x66688818ca0e3156!8m2!3d58.3728873!4d26.7178975?hl=en}%
{Estonian Genome Center (Eesti Geenivaramu), Riia 23b.}\\[1em]
\end{tabular}

\noindent
\begin{tabular}{r@{ -- }rp{13cm}}
\multicolumn{3}{l}{\bf Friday 2 June} \\
 9:00 &  9:30 & Recap of yesterday's practicals. \\
 9:30 & 10:30 & Poisson regression for follow-up studies ---
                likelihood for a constant rate \newline
                Logistic regression for cc-studies (JP) \\
10:30 & 10:50 & Coffeee break \\
10:50 & 12:50 & Practical: Rates, rate ratio and rate difference with \texttt{glm}\newline
                Logistic regression with \texttt{glm} \\
12:50 & 14:00 & Lunch \\
14:00 & 14:30 & Recap of morning practical \\
14:30 & 15:45 & Linear and generalized linear models (EL) \newline
                All you ever wanted to know about splines (MP) \\
15:45 & 16:15 & Tea break\\
16:15 & 18:00 & Practical: Simple estimation of effects \newline
                Estimation and reporting of linear and curved effects \\
\end{tabular}

\noindent
\begin{tabular}{r@{ -- }rp{13cm}}
 \multicolumn{3}{l}{\bf Saturday 3 June} \\
 9:00 &  9:30 & Recap of yesterday's practicals \\
 9:30 & 10:30 & More advanced graphics in R, including \texttt{ggplot2} (MP)\\
10:30 & 10:50 & Coffee break. \\
10:50 & 12:50 & Practical: Graphical meccano \\
12:50 & 14:00 & Lunch\\
\multicolumn{2}{l}{Afternoon}
              & Orienteering and visit at the 
\href{https://www.google.dk/maps/place/Estonian+National+Museum/@58.395294,26.7355243,15z/data=!4m5!3m4!1s0x46eb371e0069d991:0x3b484674469f1cea!8m2!3d58.395294!4d26.744279?hl=en}{Estonian National Museum} (optional)\\[2em]
\end{tabular}

\noindent
\begin{tabular}{r@{ -- }rp{13cm}}
\multicolumn{3}{l}{\bf Sunday 4 June} \\
 9:00 &  9:30 & Recap of yesterday's practicals \\
 9:30 & 10:30 & Survival analysis: Kaplan Meier \& simple
                Cox-model. Simple competing risks and relative
                survival. (JP)\\
10:30 & 10:50 & Tea break\\
10:50 & 12:50 & Practical: Survival and competing risks in oral\
                cancer. Relative survival.\\
12:50 & 14:00 & Lunch \\
14:00 & 14:30 & Recap of morning practical \\
14:30 & 15:30 & Dates in R; follow up representation in \texttt{Lexis} objects,
                time-splitting, multistate model and SMR. (BxC)\\
15:30 & 16:00 & Coffee break. \\
16:00 & 18:00 & Practical: Time-splitting and SMR (Danish diabetes patients)\\[1em]
\end{tabular}

\noindent
\begin{tabular}{r@{ -- }rp{13cm}}
 \multicolumn{3}{l}{\bf Monday 5 June} \\
 9:00 &  9:30 & Recap of yesterday's practicals \\
 9:30 & 10:30 & Nested and matched cc-studies \& Case-cohort studies (EL) \\
10:30 & 10:50 & Coffee break. \\
10:50 & 12:50 & Practical: CC study: Risk factors for Coronary heart disease\\
12:50 & 14:00 & Lunch \\
14:00 & 14:30 & Recap of morning practical \\
14:30 & 15:30 & Causal inference. (KF)\\
15:30 & 16:00 & Coffee break. \\
16:00 & 18:00 & Practical: Simulation and causal inference\\
19:00 &       & Course dinner at %
\href{https://www.google.com/maps/place/Vilde+lokaal+ja+kohvik,+catering/@58.3786583,26.7195608,17.25z/data=!4m5!3m4!1s0x46eb36e097301cdb:0x32b45e5712e44ff!8m2!3d58.3781843!4d26.7228678}{Wilde}\\[1em]
\end{tabular}

\noindent
\begin{tabular}{r@{ -- }rp{13cm}}
\multicolumn{3}{l}{\bf Tuesday 6 June} \\
 9:00 &  9:30 & Recap of yesterday's practicals \\
 9:30 & 10:30 & Multistate models, Poisson models for rates and
                simulation of \texttt{Lexis} objects (BxC)\\
10:30 & 10:50 & Coffee break. \\
10:50 & 12:30 & Practical: Multistate-model: Renal complications\\
12:30 & 13:00 & Recap of morning practical \\
13:00 & 13:15 & Wrap-up and farewell.\\
13:15 & 14:15 & Lunch \\
% \multicolumn{2}{l}{Afternoon} & Post-mortem (Faculty only). \\
% \multicolumn{2}{l}{Evening}   & Faculty dinner.\\
\end{tabular}
\vfill
\noindent
Further material will appear at this year's course website:\\ 
\url{http://bendixcarstensen.com/SPE/2017}

\cleardoublepage
\pagenumbering{arabic}
\pagestyle{fancy}
\renewcommand{\sectionmark}[1]{\markboth{\thesection #1}{\thesection \ #1}}
\renewcommand{\headrulewidth}{0.1pt}
\setcounter{section}{0}
\setcounter{page}{1}

%% \end{document}

\chapter{Exercises}
\newcommand{\ExcSol}{Exercises}

\addcontentsline{toc}{section}{Introduction to practicals}
% \section*{Introduction to practicals}

\subsection*{Data sets}

Datasets for the practicals in this course, as well as some useful \R
scripts, will be available on the course material dowloading page at \url{https://spe-r.github.io/} and on Bendix website \url{http://bendixcarstensen.com/SPE}. In addition to the data we will use during the course, you will
also find the ``housekeeping'' scripts created to save you long typing.

We advise to get all files in one go, download the zip
file \url{https://github.com/SPE-R/SPE/raw/gh-spe-material/SPE-all-material.zip}.

\subsection*{Graphical User Interfaces to R}

When running the exercises it is a good idea to use a text editor
instead of typing your commands directly at the R prompt. On Windows
and macOS, R comes with a basic graphical user interface including a
built-in text editor. Many people like to use the RStudio interface to
R, which includes a very powerful syntax-highlighting editor.

\subsection*{Keyboard shortcuts}

In the past we have found that some participants have had difficulty
finding keys for symbols that are commonly used in the R language.  In
particular, the tilde symbol \verb+~+ is used in all modelling
functions but not directly available on some keyboards. If this
affects you then please consult the Wikipedia
page: \url{http://en.wikipedia.org/wiki/Tilde#Keyboards} for advice on
the combination of key presses you will need to get tilde.

\subsection*{Recaps}
The R-scripts used during the course for the recaps will be available
in \url{http://BendixCarstensen.com/SPE/recap}.

\subsection*{Ask for help}

The faculty are here to help you. Ask them for help.




% Thursday 1
\pput{basic-e}{Practice with basic \R}
\pput{dinput-e}{Simple reading and data input}
\pput{tab-e}{Tabulation}
\pput{graph-intro-e}{Introduction to graphs in R}
\pput{simulation-e}{Simple simulation}

% Friday 2
\pput{rates-rrrd-e}{Rates, rate ratio and rate difference with \texttt{glm}}  % BxC / EL
\pput{logistic-e}{Logistic regression with \texttt{glm}}  % EL
\pput{effects-e}{Simple estimation of effects}
\pput{cont-eff-e}{Estimation and reporting of linear and curved effects}  % BxC

% Saturday 3
\pput{graphics-e}{Graphical meccano}                 % MP

% Sunday 4
\pput{oral-e}{Survival and competing risks in oral cancer}  % EL
\pput{DMDK-e}{Time-splitting and SMR (Danish diabetes patients)} % BxC

% Monday 5
\pput{causal-e}{Simulation and causal inference}     % KF
\pput{occoh-caco-e}{Nested case-control and case-cohort study} % EL

% Tuesday 6
\pput{renal-e}{Multistate-model: Renal complications} % BxC

%% check if withoutsolutions variable has been pass when compiling 
%% and stop or continue the compilation accordingly
\ifdefined\withsolutions
  
\else
  \end{document}
\fi

% \end{document}

\chapter{Solutions}
\renewcommand{\ExcSol}{Solutions}

There is a chapter for each of the exercises used at the
course. This is either a printout of the R-program that
performs the analyses, as well as the graphs produced by the
programs, or output from an R-weave solution file with a bit more
elaborate text.

The code and the output from these programs are also available from
the course homepage in \url{http://BendixCarstensen/SPE/R}; they are
called \texttt{xxx-s.R}; just before each chapter you will find a line
with the text \texttt{xxx-s}, indicating that the name of the script
will be \texttt{xxx-s.R}.

% Note that exercsises without sultion are replaced by an \addtocounter
% in order to syncronize the numbering between the exercise chapter
% and the solution chapter

% Thursday 1
\addtocounter{section}{1}
% \pput{basic-s}{Practice with basic \R}
\addtocounter{section}{1}
% \pput{dinput-s}{Simple reading and data input}
\pput{tab-s}{Tabulation}
\pput{graph-intro-s}{Introduction to graphs in R}  % KF
\pput{simulation-s}{Simple simulation}  % KF
% 
% Friday 2
\pput{rates-rrrd-s}{Rates, rate ratio and rate difference with \texttt{glm}}  % JP
\pput{logistic-s}{Logistic regression with \texttt{glm}}  % JP
\pput{effects-s}{Simple estimation of effects}  % EL
\pput{cont-eff-s}{Estimation and reporting of linear and curved effects}  % EL

% Saturday 3
\addtocounter{section}{1}
% \pput{graphics-s}{Graphical meccano} % MP

% Sunday 4
\pput{oral-s}{Survival and competing risks in oral cancer}  % EL
\pput{DMDK-s}{Time-splitting and SMR (Danish diabetes patients)} % BxC

% Monday 5
\pput{causal-s}{Simulation and causal inference}     % KF
\pput{occoh-caco-s}{CC study: Risk factors for Coronary heart disease} % EL

% Tuesday 6
\pput{renal-s}{Multistate-model: Renal complications} % BxC

\end{document}
