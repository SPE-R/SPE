% \section*{Introduction to practicals}

\subsection*{Data sets}

Datasets for the practicals in this course, as well as some useful R
scripts, will be available on the course homepage, in
\url{http://BendixCarstensen.com/SPE/data}. This is where you will
also find the ``housekeeping'' scripts designed to save you typing.

To get all files in one go, download the zip
file \url{http://spe-r.github.io/data.zip}.

\subsection*{Graphical User Interfaces to R}

When running the exercises it is a good idea to use a text editor
instead of typing your commands directly at the R prompt. On Windows
and macOS, R comes with a basic graphical user interface including a
built-in text editor. Many people like to use the RStudio interface to
R, which includes a very powerful syntax-highlighting editor.

\subsection*{Keyboard shortcuts}

In the past we have found that some participants have had difficulty
finding keys for symbols that are commonly used in the R language.  In
particular, the tilde symbol \verb+~+ is used in all modelling
functions but not directly available on some keyboards. If this
affects you then please consult the Wikipedia
page: \url{http://en.wikipedia.org/wiki/Tilde#Keyboards} for advice on
the combination of key presses you will need to get tilde.

\subsection*{Recaps}
The R-scripts used during the course for the recaps will be available
in \url{http://BendixCarstensen.com/SPE/recap}.

\subsection*{Ask for help}

The faculty are here to help you. Ask them for help.


