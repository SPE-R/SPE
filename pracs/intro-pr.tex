% \section*{Introduction to practicals}
 Datasets for the practicals in this course will be available on the
 local machines and on the course homepage, in
 \url{http://BendixCarstensen.com/SPE/data}. This is where you will
 also find the ``housekeeping'' scripts designed to save you typing.

 The R-scripts used during the course for the recaps in the morning
 will be available in \url{http://BendixCarstensen.com/SPE/recap}.

 The general convention is that when \R-functions are mentioned in the
 text they will normally not be explained in any great detail. Hence
 you should get into the habit of consulting the help page for any
 function that you are not entirely familiar with by typing one of
\begin{verbatim}
?Lexis
args( Lexis )
\end{verbatim}
The first form brings up a help-page and the second just a listing of
the function arguments with their defaults (without any explanation).

At the end of each help-page is (normally) an example showing some
aspects of the use of the function. This example can be run in your
\R-session by typing:
\begin{verbatim}
example( Lexis )
\end{verbatim}
This has the advantage that you can play around with the function,
because the data structures used for illustration will be available in
your \R-session.

When running the exercises it is a good idea to use a text editor
instead of typing your commands directly at the R prompt. On Windows
and macOS, R comes with a basic graphical user interface including a
built-in text editor. Many people like to use the RStudio interface to
R, which includes a very powerful syntax-highlighting editor.

